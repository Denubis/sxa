\setvariables[article][shortauthor={Johnson}, date={May 2016}, issue={1}, DOI={10.7916/D84B31DN}]

\setupinteraction[title={A Review of \quote{Two Plantations}},author={Jessica Marie Johnson}, date={May 2016}, subtitle={Review}]
\environment env_journal


\starttext


\startchapter[title={A Review of \quote{Two Plantations}}
, marking={Review}
, bookmark={A Review of \quote{Two Plantations}}]


\startlines
{\bf
Jessica Marie Johnson
}
\stoplines


Based on research compiled by Richard Dunn over more than forty years, the website \useURL[url1][http://www.twoplantations.com][][{\em Two Plantations}]\from[url1] expands on his work published in print and e-book form as {\em A Tale of Two Plantations: Slave Life and Labor in Jamaica and Virginia}.\footnote{{\em Two Plantations}, \useURL[url2][http://www.twoplantations.com]\from[url2]; Richard Dunn, {\em A Tale of Two Plantations: Slave Life and Labor in Jamaica and Virginia} (Cambridge, MA: Harvard University Press, 2014).} Dunn's research on seven multigenerational families was culled from plantation ledgers and other source material held by the Barham Papers in the Bodleian Library at the University of Oxford and the Tayloe Papers at the Virginia Historical Society in Richmond. Creation of the site itself, however, was spearheaded by Vincent Brown and is a project of the History Design Studio.

{\em Two Plantations} leads users horizontally and vertically into representations of the lives of 431 enslaved people at two plantations. The landing page features a short description of the site with a present-day image of a building surrounded by lush greenery, a stately structure presumably on the site of either Mesopotamia (Jamaica) or Mount Airy (Virginia) and rendered in shades of blue. Slight, nearly undetectable animation causes the sentence beneath the description to glow. In italics, it reads, \quotation{This is their story.} Clicking an arrow icon slides the user to the right, but not immediately into the names database or the handwritten family tree (painstakingly digitally rendered, since the family tree in paper form is over five feet in length). Instead, the user is advanced to a paragraph length introduction to Dunn's research set against a blue scale rendering of a page from a plantation ledger. The words \quotation{Doctors Hall,} the name of a farm quarter owned by Mount Airy's proprietors, loom especially large against the zoomed-in page. As a composition, these design choices acquaint the user with the work of a present-day historian of slavery juxtaposed against the work of the slave-owning past as preserved by archivists, endeavored upon by slave owners, and engaged in by the enslaved themselves---men with names like James, Gerard, Isaac.

As the user clicks and is guided through the site from left to right (following the logic of a book), such multilayered and intertextual experiences continue. The next slide, against a verdant blue-scale landscape that is neither clearly Jamaica nor Virginia, includes a query as to why studying the lives of enslaved families across two different plantations, in two different slaveholding societies, is important. In Jamaica, enslaved death rates exceeded birth rates; in Virginia, slaves built multigenerational families. On both plantations, however, those in bondage suffered. {\em Two Plantations} thus outlines some of the deep historiographical debates it wishes to illustrate. In comparative slave studies, compiling death and birth rates, manumission records, slave trade flows, types of crop, and other statistics has been crucial to how historians have organized data and analyzed the violent nature of bondage. In the past, historians have wielded quantitative data to compare brutality in and among slaveholding societies, or the extent to which varied European authorities destroyed the social and political lives of Africans in the Americas. {\em Two Plantations} gestures to these quantitative studies, but also to more nuanced, qualitative analyses of slave life across the Americas by suggesting that death rates, high as they were, should not be the only data used to determine the suffering of the enslaved.

As the user is moved through the next two slides---one for Mesopotamia and one for Mount Airy---information about the plantations is foregrounded. Short paragraphs summarize demographics of the enslaved population at each location. The three enslaved families recorded at Mesopotamia are described as \quotation{continually stunted by death.} In contrast, the four enslaved families recorded at Mount Airy reproduced, but \quotation{were routinely broken up by movement and sale.} Text divulges the names of the owners of each plantation and the user is introduced to Sally Thurston, a woman enslaved at Mount Airy Plantation, Virginia.

More on Sally Thurston emerges as the next button diverges from the horizontal flow and directs users downward, drawing them deeper into the history itself. Beneath the fold, the first document is a digital rendering of Sally Thurston's family tree as handwritten by Dunn. The structure of the tree is familiar---hierarchical and patronymic---and broad, requiring the user to either utilize the scrolling function in the header or to swipe left and right to view it in full. Sally is the focus, her mothering and motherhood. In the upper right hand corner, Dunn provides a short tally on her children with her first husband, Amphy Thurston: thirteen children, forty-two grandchildren, and twenty-four great-grandchildren. The tree itself, however, is shallow, either because available records stop in 1833 or as a result of the pressure movement, sale, and death placed on kinship networks. The visual impact of the truncated tree brings this pressure, the loss and lost history, into sharp relief.

Still beneath the fold, the user is guided to the right to documentation on Mesopotamia and Sarah Affir (Affy) and a digitized family tree of her descendants. Sarah is the head of this tree, and her children are described using race and color terminology as articulated by Jamaican slaveowners. Names like Mulatto Ann and Quadroon Jane proliferate. The difference in racial nomenclature makes explicit what was implicit at Mount Airy---that sex across the color line and all of its attendant relations of power, violence, and property were very much a part of enslaved women's lives crucial to understanding enslaved family dynamics and central to any historiographical debate on slavery in Jamaica, Virginia or beyond. The text of Sarah's family tree is searchable, and with a few clicks the user discovers that the family trees of all seven families are also available. Clicking on a name opens a dialog box with biographical information on the individual, and the graphical user interface makes navigating family lines easy and interesting.

The next set of documents is in database form. The members of all seven families are listed in twin databases with identical fields---name, birth year, family head, relationship, description, and death. This is no small victory. When designing databases, overcoming differences between archival documents and translating them into uniform data sets is a serious challenge. The {\em Two Plantations} website resolves these differences by organizing records by name, birth and death year, head of the family tree, and family relationship, and then adding a description box for more detailed commentary. Users can filter using these same variables or toggle for individuals who appear in the archive but not in the 1870 US census (the 1870 census was the first to count the formerly enslaved).

The {\em Two Plantations} project tackles in digital form what slavery studies attempts in analog texts---breaking the archive of its commitment to mastery and obsession with property and therefore paternity. A final analysis slide asks the user to consider major historiographic and methodological questions on what it means to interpret documents related to slavery and narrate the lived reality of the enslaved. In interviews, Brown has described the aim of the project as encouraging users to read/view/interact against the grain as well as with the grain.\footnote{Vincent Brown, \quotation{Two Plantations: Enslaved Families in Virginia and Jamaica,} in the presentation {\em The Caribbean Digital 2: Histories, Cartographies, Narratives}, 4 December 2015, Columbia University, New York.} The geography of the site maps out this imperative. As with all digital projects, users ultimately choose how, when, and where they will enter into the site itself. However, the experience of moving along the site's projected route is quite literally a journey along two axes. In walking horizontally \quotation{against the grain} of slavery's archive, the user is exposed to post-1960s historiography of slavery, comparing bondage across slave owners and plantation societies, and into the post-1990s political and social turn toward the everyday lives of the enslaved. Especially important, however, is the way {\em Two Plantations} moves the user vertically, \quotation{with the grain} and into the archive itself. This is not as simple as it seems. With {\em Two Plantations}, users engage with the enslaved themselves by learning their names, the names of their children, and the consistency of their family networks. At the same time, {\em Two Plantations} outlines the archive created by Richard Dunn, revealing something of the work engaged in by historians and the art of their craft. This point may be of special interest to graduate students or to K--12 teachers and students in their classrooms.

Opportunities to discuss historians and their craft, the texture and textual nature of the manuscript documents, the methodological implications of data and databases for the study of slavery, might require more explicit foregrounding to best impact audiences beyond historians of slavery already well versed in these tensions and debates. For instance, a discussion of Dunn's encounter with the ledgers and the information available or unavailable in them, would be of interest to a wide range of users like archivists and librarians, scholars and students, community members and families, as well as digital humanists and designers.

As far as the design (which is gorgeous) is concerned, the visual impact of the shallow family tree shapes viewers' understanding of the brutality of slavery for families on either plantation: regardless of birth and death rates, enslaved peoples continued to struggle to retain ties to loved ones. However, this is also where a discussion of the documents themselves, for which Dunn offers his own interpretative layer (the family tree), might work well to produce a conversation within the site or among users visiting about who and what is {\em missing.} Why does the archive of slavery still continue to cause so much pain? It is both what is present in the history and what is not that resonates with users---and the same can be said of what is and is not present in {\em Two Plantations}.

It is an important choice to begin the family trees begin with the women, a reality of the time period and of the logics of reproduction. However, what of the political project of a family tree that does not take the form of the hierarchical and vertical tree diagram? Would multidimensional data visualizations or polysingular (network analysis using multiple pathways and nodes of intersection and interaction) renderings of the Thurston or Affir family trees better capture the dense networks of kin, mutuality, and precarity at play in bondage? Would enslaved women continue to have a central role in these new models or is \quotation{head of household} a political project that cannot be avoided, a reality of the archive we do not dare ignore? Is it possible for digital history to depict the enslaved in networks of kin united in a practice of labor on behalf of and for each other as well as in biological or fictive units tied to institutions?

{\em Two Plantations} digitizes an important and significant archive in an extremely well-conceived intellectual digital project while offering scholars a delicious design challenge. Historians of slavery designing digital projects do battle with inherited assumptions of difference and deviance with every keystroke. By grappling with the historiographical debates around comparative slavery and the documentation on enslaved families---and doing so in digital form---{\em Two Plantations} asks provocative questions that leave users with more questions than answers. This, in so many ways, is what good digital histories of difficult pasts ought to do.

\thinrule

\page
\subsection{Jessica Marie Johnson}

Jessica Marie Johnson is Assistant Professor of History at Michigan State University. Her work has appeared in {\em Slavery & Abolition}, {\em The Black Scholar}, {\em Meridians: Feminism, Race and Transnationalism}, and {\em Debates in the Digital Humanities}. As a digital humanist, Johnson explores the way digital and social media disseminate and create historical narratives, in particular, comparative histories of slavery and people of African descent.

\stopchapter
\stoptext