\setvariables[article][shortauthor={Murray-Román}, date={May 2016}, issue={1}, DOI={10.7916/D8S46S1S}]

\setupinteraction[title={Twitter's and @douenislands' Ambiguous Paths},author={Jeannine Murray-Román}, date={May 2016}, subtitle={Ambiguous Paths}]
\environment env_journal


\starttext


\startchapter[title={Twitter's and @douenislands' Ambiguous Paths}
, marking={Ambiguous Paths}
, bookmark={Twitter's and @douenislands' Ambiguous Paths}]


\startlines
{\bf
Jeannine Murray-Román
}
\stoplines


{\startnarrower\it The Douen Islands project was cofounded in 2013 by a group of Trinidadian artists, including poet Andre Bagoo, graphic designer Kriston Chen, and artist Rodell Warner, among others. Its name refers to the folk figures of douen, whose backward-pointing feet create backward paths. This essay uses a Twitter-focused analysis to explore the representations of the douen in two Douen Islands projects: first, considering the social mediatization of the text and the performance event the artists created, and, second, exploring how a backward-reading practice impacts possible interpretations of the douen's persona as tweeted in these two projects. \stopnarrower}

\blank[2*line]
\blackrule[width=\textwidth,height=.01pt]
\blank[2*line]

In describing his aims in conceiving of a collaboratively written, destined-for-Internet-publication book of poetry, Andre Bagoo states, \quotation{I wanted to use the internet as a forum for sharing this particular work. I wanted to do something compelling online, in a way that we might not expect, given the subject matter, or given our idea of Caribbean poetry.}\footnote{Andre Bagoo, quoted in Nicholas Laughlin, \quotation{{\em Douen Islands} and the Art of Collaboration,} {\em Caribbean Review of Books}, 4 November 2013, \useURL[url1][http://caribbeanreviewofbooks.com/2013/11/04/douen-islands-and-the-art-of-collaboration]\from[url1].} The result of this inspiration is Douen Islands, an artists' collective based in Port of Spain, Trinidad and Tobago, and spearheaded by poet and journalist Bagoo (1983--), with graphic designer and visual artist Kriston Chen (1982--).The folkloric figure that anchors this project is the {\em douen}, a figure that, as the spirit of abandoned children, represents social crisis and a cycle of violent neglect on a national scale. This essay explores the transformations that the notoriously mysterious figure of the douen undergoes when Douen Islands creates an identity for it on the social media platforms of Twitter and Tumblr. This conjunction of private and social is salient in both of the projects Douen Islands has produced as of March 2016. In the first manifestation, the e-book \useURL[url2][http://www.dropbox.com/s/1kopn1kb0st781s/DouenIslands.pdf][][{\em Douen Islands Issue 001: Tomorrow Please God}]\from[url2], the collective, which included Brianna McCarthy (illustrator), Sharda Patasar (sitarist), and Rodell Warner (artist), published a downloadable PDF on the official Douen Islands Tumblr page, released on 31 October 2013.\footnote{Douen Islands, {\em Douen Islands Issue 001: Tomorrow Please God}, with poems by Andre Bagoo; hereafter cited in the text. The PDF can be accessed at \useURL[url3][http://www.dropbox.com/s/1kopn1kb0st781s/DouenIslands.pdf]\from[url3]. A link to the PDF was on \useURL[url4][http://www.douenislands.tumblr.com]\from[url4] from October 2013 to January 2016.} Douen Islands's second manifestation was an experimental poetry reading at the Alice Yard performance space in Port of Spain, which took place on 12 April 2014.\footnote{Located at 80 Roberts Street, Woodbrook, Port of Spain, Trinidad and Tobago, Alice Yard is a nonprofit gallery and performance space.} Named after one of Bagoo's poems in {\em Tomorrow Please God}, \quotation{In Forest and Wild Skies,} the event featured readings by Bagoo, Sharon Millar, and Shivanee Ramlochan, as well as live tweets by Warner. Both the e-book and the poetry reading were organized around the project's goals to welcome the douen in from its social marginalization. I argue that the douen, when articulated through Twitter as part of these projects, uses the writing conventions of the social media platform to reassert its location in the margins and, paradoxically, remain ambiguous to the point of unreadability. The gesture to social media within these texts and performances therefore creates a dynamic counterpoint that retains the douen's characteristic powers of deception through twenty-first-century tools.

Twitter, like other microblogging platforms, is most known for short but constantly updated descriptions of users' quotidian observations and experiences. Douen Islands draws on this incremental and short-form performance of identity, which is based on ongoing disclosures through \quotation{a staccato of one-liners} that are, as Michele Zappavigna writes, meant to affiliate users to one another.\footnote{Michele Zappavigna, {\em Discourse of Twitter and Social Media} (London: Continuum, 2012), 38.} The collective also relies on the platform's formal elements, notably, the discontinuity of tweets, Twitter's reverse chronological organization, and a volume of production that creates effective ephemerality. First, while tweets can be linked together with \quotation{cont.} or \quotation{1 out of x number {[}of tweets{]},} generally they are assumed to be discontinuous, recording separate ideas and located in separate contexts. As such, they lend themselves to the work of de- and recontextualization to which Chen puts Warner's tweets when curating them for {\em Tomorrow Please God}. Second, reverse chronology sidesteps the traditional beginning-to-end reading experience of most print matter by placing the most recently published tweets at the top of the feed.\footnote{In February 2016, Twitter announced that Timelines would change from reverse chronology to include an algorithmic mode of organization.} In general, this practice stresses the importance of immediacy, but when tweeted from the user handle @douenislands and invoking a folk figure whose signature act is to create backward paths, I propose that the reference to reading on Twitter signals a social media platform that has acculturated us to reading in reverse.

This reverse chronology and temporal backward organization is the primary link to the douen figure, who has a backward practice of its own. The douen is the spirit of a child who died before being baptized and then becomes a kind of bogeyman who, at dusk, tries to lure neglected children away from their homes and into the forest. Some aspects of the douen's appearance are variable: the douen sometimes has no face or sometimes has a small round mouth with which it feeds on water crab, and yet in other instances, the face is covered by its large, floppy straw hat. Its only absolutely fixed characteristic is that its feet are on backward.\footnote{A figure that shares a similar origin story and the physical characteristic of the backward feet with Trinidad and Tobago's douen can be found throughout Latin America as {\em el duende}.} In the project description on his website, Chen remarks that he was inspired by Alfred Codallo's representation of Trinidad's mythology in the 1958 watercolor \quotation{Folklore.}\footnote{Kriston Chen, \quotation{Douen Islands: In Forest and Wild Skies,} 1 August 2014, \useURL[url5][http://www.notsirk.com/]\from[url5] {[}Page not found as of September 2017{]}.} In this painting, which art critic and artist Kenwyn Chrichlow identifies as providing, in an art historical context, \quotation{a reflection of concerns expressed during the decolonization of Trinidad and Tobago in the years before political Independence in 1962,} the douen are on the margins of the frame or on the edges of the river, their awkward feet requiring a double-take in an already fantastical painting.\footnote{See Kenwyn Chrichlow, \quotation{Figures at the Forest's Edge,} {\em Kaieteur News Online}, 6 June 2010, \useURL[url6][http://www.kaieteurnewsonline.com/2010/06/06/\%E2\%80\%98figures-trapped-at-the-forest\%E2\%80\%99s-edge\%E2\%80\%99][][http://www.kaieteurnewsonline.com/2010/06/06/\letterpercent{}E2\letterpercent{}80\letterpercent{}98figures-trapped-at-the-forest\letterpercent{}E2\letterpercent{}80\letterpercent{}99s-edge\letterpercent{}E2\letterpercent{}80\letterpercent{}99]\from[url6].} As filtered through this painting, the folk figure has a timeless quality at the same time that it is also historically anchored in processes of nation building. In recontextualizing the douen for early-twenty-first-century concerns, Douen Islands mines the douen's characteristic environment to explore marginalization: dusk as a liminal moment, the forest as a space of ominous mystery, and the paths left by the douen's backward feet that misdirect anyone who might be chasing after stolen children. It is from the starting point of the douen's role as providing a dangerous reminder of the possible consequences of social neglect that I consider Chen's incorporation of Warner's tweets into {\em Tomorrow Please God} as well as Warner's live-tweeting during Douen Islands's Alice Yard event.

Douen Islands has used Tumblr as the primary organ for publishing and advertising its works and events, including its first manifestation, {\em Tomorrow Please God}, which Chen designed with InDesign for screenic reading as a PDF. Digital tools are important to two aspects of the project: first, Bagoo and Chen used Tumblr to develop a collaborative creative model, and, second, online publication creates its own circuits for dissemination. When first developing a working relationship, Bagoo and Chen began with a cowritten Tumblr page on which Chen would post visual responses to Bagoo's poems, and they would each critique the work that had resulted from their exchanges. Tumblr's aptness for publishing images was suited to placing Bagoo's poetry and Chen's image-based vocabulary in an evenly matched conversation. {\em Tomorrow Please God} was therefore created through a series of dialogues between Chen and Bagoo and in relation to Warner's Twitter feed. Bagoo, Chen, and Warner had been working together for some time before the creation of {\em Tomorrow Please God}, and when Chen was designing the e-book, he identified tweets from Warner's feed that resonated with Bagoo's poems. As Chen recollects, \quotation{Nothing was decided, no instructions were given to Rodell. Just him doing what he was already doing, and if anything struck me, then we'd use it.}\footnote{Kriston Chen, e-mail message to author, 8 July 2015.} Chen therefore curated the serendipitous resonances between Warner's microblogging and Bagoo's poetry with evocative results for thinking about the interaction between social media and digital art.

Both the process and product of {\em Tomorrow Please God} are grounded in artistic exchanges facilitated by text first produced for social media platforms. {\em Tomorrow Please God} was published as a downloadable file in Dropbox and accessible on the Douen Islands Tumblr site from October 2013 to January 2016. As Bagoo comments, \quotation{The Internet, while it often is seen to be in animus with publishing, might also be an opportunity for post-colonial countries like ours, to publish our own stories in our own ways, using cyberspace's breadth of tools and its reach.}\footnote{Andre Bagoo, quoted in Laughlin, \quotation{{\em Douen Islands} and the Art of Collaboration.}} Elsewhere, Bagoo has spoken about the difficulties presented by the absence of publishers or institutional support for the arts in Trinidad and Togabo.\footnote{See Geosi Gyasi, \quotation{Interview with Trinidadian Poet, Andre Bagoo,} 25 April 2015, \useURL[url7][http://geosireads.wordpress.com/2015/04/25/interview-with-trinidadian-poet-andre-bagoo]\from[url7].} Bagoo's insistence on the local---\quotation{our own}---can find logistical support in the Internet's capacity to make the text available locally and immediately, without needing to send the text abroad and subject it to potentially homogenizing influences before returning it to readers. While the World Wide Web often alludes to the capability to connect readers and writers without regard for distance, in this case the breadth of cyberspace's reach is focused on building institutions for poetry in Port of Spain. By conjoining the traditional figure of the douen with social media platforms, Bagoo, Chen, and their collaborators are able to both look back to Trinidad's folkloric roots and to look forward to emerging digital writing practices.

Douen Islands harnesses digital tools to advertise and disseminate their work, to develop a collaborative poetic practice, and to experiment with the genres of microblogging. The use of Twitter at the Alice Yard event calls attention to the douen's cutting and sardonic voice while purposefully re-asserting the douen's marginalization, and the embedding of tweets in the e-book evokes the possibility for polyvocal counter-narratives. I argue that the tweets incorporated in both events sidestep the project's goal to bring the douen out of the margins by normalizing it: these tweets are a site in which the douen retains its devious potential for misdirection. As such, my readings concentrate primarily on the relationship between the douen as represented through the tweets in Douen Islands's two manifestations and the formal properties of Twitter. Because an exploration of Twitter's functionality undergirds my theorization of how Douen Islands both uses and gestures to the platform, I also proceed in reverse chronology alongside Twitter feeds and the douen's misdirecting footprints: I begin with the Alice Yard reading before turning to Douen Islands's inaugurating event, the e-book {\em Tomorrow Please God}, in order to explore the potential for amplifying, remixing, and transforming the folk figure of the douen as a tweeting subject.

\subsection[title={\quotation{In Forest and Wild Skies}: The Backchannel and the Bathroom},reference={in-forest-and-wild-skies-the-backchannel-and-the-bathroom}]

\quotation{In Forest and Wild Skies} was \quotation{an experimental evening} of poetry at Alice Yard, Port of Spain, on 12 April 2014. Andre Bagoo, Sharon Millar, and Shivanee Ramlochan did not give their readings from lecterns, facing the audience; the poets were deliberately separated from their audiences, \useURL[url8][https://www.instagram.com/p/m9BmgSGxJS/][][reading]\from[url8] from a side room and through a slotted window or from behind a screen and with the trailer video for the event projected behind them or, in Ramlochan's case, silently \useURL[url9][https://www.instagram.com/p/m_SDXEGxJn/][][writing]\from[url9] on the walls in chalk.\footnote{See \useURL[url10][http://www.instagram.com/p/m9BmgSGxJS/][][http://www.instagram.com/p/m9BmgSGxJS]\from[url10]; and \useURL[url11][http://www.instagram.com/p/m_SDXEGxJn/]\from[url11]. See also performance stills included in Ramlochan's review of the event, \quotation{An Evening of Alice Yard Douens,} 14 April 2014, \useURL[url12][http://www.bocaslitfest.com/2014/an-evening-of-alice-yard-douens]\from[url12].} As part of the performance, Kriston Chen engineered a soundtrack based on recordings of Sharda Patasar's sitar performances, now housed on the \useURL[url13][http://soundcloud.com/douenislands/douen-islands-in-forests-wild-skies-soundtrack-feat-sharda-patasar][][Douen Islands Soundcloud page]\from[url13], and \useURL[url14][http://www.instagram.com/p/m89WgVGxCB/][][Chen's visual work]\from[url14] was projected throughout the space, inviting audience members to interact with both text and image.\footnote{See \useURL[url15][http://soundcloud.com/douenislands/douen-islands-in-forests-wild-skies-soundtrack-feat-sharda-patasar]\from[url15]; and \useURL[url16][http://www.instagram.com/p/m89WgVGxCB/]\from[url16].} Within this event dedicated to douen voices and experimenting with constrained spaces, staging new relationships between speakers and audiences, Rodell Warner contributed live tweets at the event, which were projected into the bathroom and dubbed \quotation{Live-Tweet Stalls.} The Live-Tweet Stalls doubled down on the douen's marginalization and called attention to the kind of articulations that are most possible in spaces that are on the edge of the social. The Live-Tweet Stalls both dwelled in the subversive intimacy of the bathroom and transformed that private space into a public one by hybridizing the live and the digital. The specific circuit in which a tweet was published from @douenislands, projected into the bathroom, and redigitized as an image of a tweet in the Live-Tweet Stalls highlighted the tension between the closely guarded privacy and entry into the domain of the social that the tweeting douen must manage.

For Warner, a photographer and conceptual artist based in Port of Spain, this interaction between the private and the social is a long-held interest. One of his early installation pieces, \quotation{Photobooth,} generated this dynamic. An interactive photography installation first assembled during the 2009 Erotic Art Week Port of Spain at Alice Yard,\footnote{The 2009 version of this work was installed and exhibited at Alice Yard; its 2010 and 2011 Erotic Art Week installations took place at the Bohemia Gallery in Port of Spain.} Warner created what Dave Williams describes as a \quotation{tiny white cocoon, \ldots{} shrouded off from the rest of the small Alice Yard annex by draping yards of white fabric from the ceiling,} in which participants were invited to stage their full-body portraits with as much of their bodies playfully and explicitly revealed or concealed according to their own choosing; \useURL[url17][http://www.rodellwarner.com/index.php?/photobooth][][these photographs]\from[url17] were then exhibited during the nine-day run.\footnote{See \useURL[url18][http://www.rodellwarner.com/index.php?/photobooth]\from[url18]; and Dave Williams, \quotation{Losing Your Head in the Photobooth,} {\em sx salon}, 9 November 2009.} This conjunction of intimate interaction and exhibition in \quotation{Photobooth} created a nexus between the private and the social, the cocoon and the digital photo. The installation was structured by its intermediality, a term that, as Marcela Fuentes explains, identifies how the interaction of live performance practices and digital practices can constitute a specific experiential space.\footnote{Marcela Fuentes, \quotation{Performance Constellations: Memory and Event in Digitally Enabled Protests in the Americas,} {\em Text and Performance Quarterly} 31, no. 1 (2015): 34.} Warner's writing of live tweets for \quotation{In Forest and Wild Skies} shares a similar dynamic engagement between the digital and the live. During the event, Warner's writing appeared in two spaces: on the Douen Islands's feed on Twitter and the bathroom walls at Alice Yard onto which the feed was projected. The Live-Tweet Stalls combined the privacy of a bathroom with the broadcasting function of the social media platform. This form of intermediality relied on the site of the bathroom as symbol of what the body (politic) eliminates and relegates to the out of sight, reinforcing the douen's isolation, while also publicizing the douen's presence by placing it into social media circulation.

Warner's participation in \quotation{In Forest and Wild Skies} used Twitter to shift the douen's potentially subversive voice from the margins into an event's broader discourse. Because of the critical and even mocking twist that the live tweets brought to the core questions of social neglect raised by the figure of the douen, Warner's participation in the event corresponds to that of backchanneling. Originating in the field of linguistics, the backchannel refers to how listeners indicate their participation in live conversation through nonverbal and phatic communications that acknowledge their reception of the speakers' speech. On Twitter and other social media, backchanneling has come to mean verbal contributions to the main conversation as audience-participants voice, record, and publish their responses to an event, adding their running commentary during a live event in real time in addition to reactive tokens and nonverbal assents in in-person interactions. The most common uses of Twitter backchannels are to disseminate notes of the event to create an archive, to signal boost an event in real time, or to record questions to give the presenter a heads-up for the direction the question-and-answer period might take.\footnote{See Cliff Atkinson, {\em The Backchannel: How Audiences Are Using Twitter and Social Media and Changing Presentations Forever} (Berkeley, CA: New Riders, 2010). Using the formulation \quotation{an audience with an audience} (5) to describe the backchannel, Atkinson notes the double-edged potential of a backchannel from the primary speaker's viewpoint: \quotation{A backchannel is a line of communication created by people in an audience to connect with others inside or outside the room, with or without the knowledge of the speaker at the front of the room. Usually facilitated by Internet technologies, it is spontaneous, self-directed, and limited in time to the duration of a live event. A backchannel can be constructive when it enhances and extends helpful information and relationships, and it can be destructive when it articulates and amplifies counterproductive emotions and sentiments} (17). What is constructive and destructive---and to whom---is debatable, given the relationship between the back-channeler and the dominant voice.} The genre can also create alternate programming accompanying the actual event: sites of instant critique, parody, mockery, and even derailments of the official presenters' aims reshape the conversation so that it speaks to the audience's concerns. During \quotation{In Forest and Wild Skies,} because Warner tweeted from @douenislands, his participation was both evidently coordinated with the event, and simultaneously added a sardonic edge to the evocations of the douen taking place at Alice Yard.

Outside the terminological coincidence of the douen's backward walk and the backchannel, what links the two practices? Both the douen and the backchanneler, existing on the margins of social engagement, draw the attention of the dominant conversation to its marginal site through physical or rhetorical violence: the douen, when children are drawn from their homes; the backchanneler, when pointed commentary derails the official presentation. The screenic projection of Warner's tweets in the Live-Tweet Stalls offered a visualization of how a backchanneler interacts with the main event: the grainy texture of the vinyl screen or paper sheet onto which the tweets were projected roughened up the smoothness of the phone or computer screen.

\placefigure{Screen capture of the Douen Islands Instagram account}{\externalfigure[images/douenislands.jpg]}
The image in figure 1, with the edges of the Twitter feed curling up, reveals how the change in screens from digital to analog distorts its initial mode of publication. Even when redigitized in order to post on the @douenislands Instagram page, the waviness and pock marks of the Live-Tweet Stalls' intermedial environment retains the roughness of how Warner backchannels the marginal voice of the douen tweeting at the most private and perverse edges of social space.

Twitter feeds reveal both how a subject establishes his or her personal voice and, through mentions of other tweeters and retweets of others' comments, how he or she forms a community on the social media platform. Retweeting is most commonly meant to disseminate information to one's own followers but can serve as a mode of citation, a way of bookmarking a tweet for later reference, and, on occasion, can form a loosely and temporarily held community around a common issue.\footnote{Alex Maireder and Stephan Schlögl term these temporary communities formed on social networks \quotation{encounter-publics,} often organized around timely and relevant hashtag that allows \quotation{less conventional voiced {[}to{]} make themselves heard and sometimes heavily interact with the centre.} Alex Maireder and Stephan Schlögl, \quotation{24 Hours of an \#outcry: The Networked Publics of a Socio-political Debate,} {\em European Journal of Communication} 29, no. 6 (2014): 689.} From 5:53 p.m. to 6:11 p.m., Warner shifted from writing tweets to retweeting. However, rather than draw from a wide circle of other Twitter users followed by @douenislands, Warner retweeted from what was his primary Twitter account at the time, @wanderroller, restricting the tweeting douen from a well-connected social ecology. The douen's own identity and tweeting voice developed over the course of the evening: the tone of Warner's tweets for @douenislands varied widely, including scatological puns, overt challenges to readers, melancholy, and, occasionally, a flat and unreadable affect---all of which were voiced in a demotic register. The identity performed through Warner's fifty-four tweets is that of an irreverent child-spirit, with a cutting and succinct, even laconic, wit---as is necessary for the 140-character limitation of the tweeted utterance. For example, when playing off the title of Douen Islands's e-book, {\em Tomorrow Please God}, Warner as @douenislands inverted the subject of address from God to a curse: \quotation{\useURL[url19][https://twitter.com/douenislands/status/455141412678684672][][tomorrow please muh muddacunt]\from[url19]} (5:32 p.m.).\footnote{\useURL[url20][http://twitter.com/douenislands/status/455141412678684672]\from[url20].} The specific colloquialisms and verbal constructions of English as spoken in the Caribbean was Warner's quickest route to roughening up everything from the national mottos of Trinidad and Tobago to the key phrases that anchor Douen Islands's own projects.

Because @douenislands is not a heavily used Twitter account, Warner's tweets remain available for analysis as an archive as of April 2016, creating new dynamics separate from how the individual live tweets functioned as part of the evening's performance event and permitting a retrospective reading of how the tweeting of douen's descriptions of self and address to readers shifted over the course of the evening. The first twelve tweets assert self-definitions and ask questions that also identify the douen's properties, as in Warner's first tweet of the evening as @douenislands, for example:

\placefigure{Screencapture of @douenislands Twitter feed.}{\externalfigure[images/douenislands-twitter.jpg]}
Warner registers a linguistic ping to signal the presence of the douen.\footnote{Michele Zappavigna defines the \quotation{linguistic ping} as a genre of tweets that register \quotation{I'm still here} ({\em Discourse of Twitter}, 29).} It also can serve as a challenge to readers with normalized physiology to recognize the difference between themselves and the douen or as a gesture of solidarity for other readers whose feet likewise do not face front. In part owing to the absence of punctuation---a convention of electronically mediated writing---this tweet can also take on a flat and ambiguous tone.\footnote{Naomi Baron notes that in social media and SMS texting, punctuation is often used only when it carries necessary discourse information. Naomi Baron, \quotation{Language Use in Online and Mobile Communication,} in Marie-Laure Ryan, Lori Emerson, and Benjamin Robertson, eds., {\em The Johns Hopkins Guide to Digital Media} (Baltimore, MD: Johns Hopkins University Press, 2014), 311.} This ambiguity present in each individual tweet can preserve the douen's footprints' characteristic unreadability in its electronic traces. But as the queries accumulate, the tweets develop a melancholic tenor. Among the first series of tweets are two tweets that punctuate the questions with acute pain and recrimination: \quotation{\useURL[url21][https://twitter.com/douenislands/status/455139629243842560][][plenty outta allyuh never loss]\from[url21]} (5:25 p.m.) and \quotation{\useURL[url22][https://twitter.com/douenislands/status/455140625386835969][][plenty times loss]\from[url22]} (5:29 p.m.).\footnote{\useURL[url23][http://twitter.com/douenislands/status/455139629243842560]\from[url23]; \useURL[url24][http://twitter.com/douenislands/status/455140625386835969]\from[url24].} Warner voices the douen's participation on Twitter as a dynamic between accusation toward others and isolation of the self. Considering the live tweets' archive in the aggregate, Warner oscillates between the introspection of self-definition and conversational engagement with readers.

Projected onto a screen in a public place, the tweets are in dialogue with the other happenings in that place, and, because they are only available for a few minutes to the public, the tweets do not constitute a stable and consistently accessible text in its screenic iteration as part of the Alice Yard performance. As visible in figure 1, the \useURL[url25][http://www.instagram.com/p/m8-KI0GxDb/][][Live-Tweet Stalls]\from[url25] featured one tweet at a time rather than the entire feed. Warner's live-tweeting, which began at 5:13 p.m. and ended at 6:49 p.m., includes fifty-four tweets. With a rate of roughly one tweet every two minutes, the text that was created over the course of the evening was aimed more at aleatory transience than the creation of a coherent archive---even if such an archive remains available on the Douen Islands Twitter feed as of February 2016. In a discussion of art installations in which texts are likewise available for reading only momentarily, Rita Raley says of such cases, \quotation{There is no stable text that one can look {\em at} for a meaningful period of time. They are not texts but text effects.}\footnote{Rita Raley, \quotation{TXTual Practice,} in N. Katherine Hayles and Jessica Pressman, eds., {\em Comparative Textual Media: Transforming the Humanities in the Postprint Era} (Minneapolis: University of Minnesota Press, 2013), 20.} Moreover, in heavily used social media accounts, the volume at which tweets are produced and the limited number of tweets that comprise a feed means that tweets generally have a very short time period of availability for reading. The posting and responding that takes place on active feeds is so rapid that older material is pushed off of the page and becomes inaccessible to users who are not following in real time, as is the case for Warner's personal accounts, for example.\footnote{While all tweets are stored in giant corporate databases and are being archived by the Library of Congress back to 2006, they are inaccessible to the lay user once they have been displaced by new tweets. Otherwise, commercial social media monitoring tools such as Gnip PowerTrack and Datasift exist specifically to allow companies to store and dredge information from users' tweets that are no longer viewable by the general public. See Dave Lee, \quotation{Twitter Partners with Datasift to Unlock Tweet Archive,} 28 February 2012, \useURL[url26][http://www.bbc.com/news/technology-17178022]\from[url26].} For performance artists such as Man Bartlett, who comments that \quotation{seeing a version of time suspended isn't nearly as interesting to {[}him{]}, no matter how it's archived,} this intangibility and the impossibility of preserving the work is a reason to produce work on Twitter.\footnote{Man Bartlett, quoted in Benjamin Sutton, \quotation{The Many Uses of Rhizome's New Social Media Preservation Tool,} {\em Hyperallergic}, 21 October 2014, \useURL[url27][http://hyperallergic.com/157039/the-many-uses-of-rhizomes-new-social-media-perservation-tool/]\from[url27].} For Warner's tweeting douen, the instability of the text and the inaccessibility of tweets creates an experience of effective ephemerality and a reading experience that is as intractably difficult as the douen's backward paths.

The douen's key physical feature is its backward feet, but it is unclear how to interpret the social ramifications derived from the capacity of throwing those who would follow them off track. Some descriptions of Caribbean folklore indicate that the douen plays pranks on children because they are the easiest to trap into following the douen into the forest where the douen abandons them. This version assumes that the childhood home is an unequivocal site of protection.\footnote{See \quotation{Our Folklore Is Predominantly of African Origin,} {\em TriniView}, \useURL[url28][http://www.triniview.com/TnT/Folklore.htm]\from[url28].} But the story included in Gérard Besson's essay, which both Chen and Bagoo read, discussed, and cited in an interview with Nicholas Laughlin, calls into question the implication that the douen target children out of caprice or malevolence. Instead, Besson's source notes that \quotation{the Duenn is haunt the parent} who has no time or attention for her child.\footnote{Gérard Besson, \quotation{Mermaids, Imps, and Goddesses: The Folklore of Trinidad and Tobago,} in {\em Trinidad and Tobago: Fifty Years of Independence} (London: First Magazine, 2012), 54.} Here, the douen is attracted to the homes where children are not protected, perhaps going so far as to acting out of a sense of retributive justice by removing the children from their homes in order to care for them in the forest. Angelo Bissessarsingh recalls a story told about a douen in the 1940s near Siparia, where a four-year-old child had been lost in the forest after his parents turned away from him, and when he was found alive and healthy months later, his well-being was credited to douens, the beings the child called \quotation{his friends.}\footnote{Angelo Bissessarsingh, \quotation{Douens and Other Folkore,} {\em Guardian}, 30 June 2013, \useURL[url29][http://www.guardian.co.tt/entertainment/2013-06-30/douens-and-other-folklore]\from[url29].} Reading back through Besson's and Bissessarsingh's stories, the first version wrongly shifts blame for children's abandonment onto the douen.

Turning away from the question of blame altogether, Laughlin proposes that the douen as represented in Douen Islands can serve as an allegory for the sociopolitical conditions of abandonment and neglect, one that both creates the douen and is experienced similarly by the contemporary Trinidadian reader: \quotation{Invited in from the wilderness and dark, with supernatural deformities erased, the douen looks more and more like any Trinidadian of the post-Independence generation: mischievous but bewildered, uncertain of his social birthright, possibly hapless, possibly not helpless.}\footnote{Laughlin, \quotation{{\em Douen Islands} and the Art of Collaboration.}} Drawing similarities between douens and contemporary Trinidadians, as Laughlin does and as Bagoo and Chen do in their interview with Laughlin about Douen Islands, highlights the extent to which the refusal of responsibility that characterizes the explanations surrounding whether or not douens pose a threat to children has become a national phenomenon. However, when the douen's persona is vocalized through Twitter, both in Warner's live tweets during the Alice Yard event and in the tweets drawn from Warner's Twitter feed to suggest the presence of a douen in {\em Tomorrow Please God}, it is the douen's differences rather than their commonality with Trinidadian citizens that surges to the fore. First, the disappearance of its traces resonates with the effective ephemerality of Twitter feeds. Second, the douen's spatially reversed paths are akin to Twitter's temporally reverse chronology by creating similar backward reading practices that the feed's organization demands. As Ann González comments about another Caribbean folk figure that is known for backward-pointing feet, \quotation{Walking in reverse becomes a metaphor, like reading in reverse, for a form of duplicity that offers a space for resistance inside the dominant codes.}\footnote{See Ann González's discussion of {\em ciguapas} in the Dominican Republic in {\em Resistance and Survival: Children's Narrative from Central America and the Caribbean} (Tucson: University of Arizona Press, 2009), 8.} When the backward paths of the traditional douen are paired with the genre conventions of writing on Twitter, the figure of the douen plays with the possibility of rejecting the gesture of welcome into the social sphere offered by Douen Islands projects and reasserts its place in the margins where it is protected by its unreadability.

\subsection[title={{\em Tomorrow Please God}: Ambiguous Readings of Half-Backward Feet},reference={tomorrow-please-god-ambiguous-readings-of-half-backward-feet}]

Facing the authors' page of {\em Tomorrow Please God} is a list of goals regarding the intervention into the figure of the douen, goals that are rearticulated in various interviews and publicity materials: \quotation{Douen Islands is a devious remixing of traditional Douen culture. (a) Remove the straw hats. (b) Invite them inside. (c) Straighten their feet. (d) All of the above. This is an open collaboration to incite anxiety, provoke beauty} (n.p.). Douen Islands suggests looking at the douen's erased faces instead of using straw hats to avoid confronting their horror and to bring them into the body politic. Undergirding this list of actions is an impetus to care for the douen by attempting to repair its damage: straightening out the douen's feet would make its traces legible and endow it with the capacity for direct communication with others. But this process of healing the douen by making its feet face front also makes the douen vulnerable to capture. So while this action welcomes the douen into social belonging, since it would pressure the douen conform to social norms, it is also a violent gesture that would neutralize the very deviousness on which Douen Islands relies in order to perform its cultural remix.

The tweeting persona of the douen as well as the visual representation of the douen's feet in {\em Tomorrow Please God} works against the e-book's stated desire to straighten out the douen's feet. For the douen's tweets imply a reverse chronological order for reading, confirm the disappearance of the douen's footfalls, and reassert capacity for cutting social commentary. Moreover, the invocation of Twitter offers the potential for reading graphic elements of {\em Tomorrow Please God} as a resistant backchannel that has been published within the e-book itself. My analysis of {\em Tomorrow Please God} focuses on the image vocabulary that Chen establishes for the douen, the multiple ways of reading the incorporation of digitally manipulated tweets, and the ways the PDF text is social-mediatized by its gestures to Twitter and its site of publication on Tumblr.

The Tumblr account on which Douen Islands disseminates its projects is itself also subject to creative experimentation with social media practices, in that Douen Islands draws on the affects most associated with the platform and also plays with the expectations for how the platform is best used. Particularly for young users, Tumblr is known as a social network in which, instead of the curated performance of effortless perfection most associated with Facebook and Instagram, users, as Elspeth Reeve writes, \quotation{confess their most vulnerable moments of social mortification} and dwell in the melancholy of heartbreak and social exclusion.\footnote{Elspeth Reeve, \quotation{The Secret Lives of Tumblr Teens,} {\em New Republic}, 17 February 2016, \useURL[url30][http://newrepublic.com/article/129002/secret-lives-tumblr-teens]\from[url30].} The douen, in some key ways, corresponds to the persona of the young, anonymous, and depressed Tumblr user: the douen has no fixed gender, wears a straw hat covering its face and its long hair plaited in the juvenile style of two braids. In some descriptions it is said to call children by name, in others, to have a hollow, owlish nonverbal \quotation{whoop} sound that sometimes devolves into cries and whimpers.\footnote{See {\em Dictionary of the English/Creole of Trinidad & Tobago: On Historical Principles}, ed. Lise Winer, s.v. \quotation{douen, duende douaine, doune, dwen, duegne} (Montreal: McGill-Queen's University Press, 2009).} While the affective tenor of this vein of Tumblr usage might therefore suit the douen's soft hooting sound that echoes throughout the forest, as in all blogging platforms a user's identity and its shades develop over the accumulation of posts. Douen Islands, however, does not generate a constant stream of posts that would permit a personality to emerge over time. Instead, the Tumblr is a standalone advertisement for its most recent artistic manifestation.

In this regard, Douen Islands also does not engage one of the uses for which Tumblr is built: the curation of images either gathered from around the Web or uploaded to the Tumblr site that all speak to one specific concept. Tumblr evolved to include text posts of any length as well as music, videos, and links; but because it was first developed for users to post one image per entry, the medium is particularly suited for aggregating visual collections, \quotation{sets of images organized through individual curatorial work {[}that{]} through collective reblogging draw attention to complex realms of identity, experience, and power relations,} as Marty Fink and Quinn Miller discuss with regards to sexual self-representation on Tumblr.\footnote{Marty Fink and Quinn Miller, \quotation{Trans Media Moments: Tumblr, 2011--2013,} {\em Television and New Media} 15, no. 7 (2013): 621.} This common use for Tumblr could also be salient to the project of representing and intervening in the figure of the douen. However, the Douen Islands Tumblr site is not a repository or archive. In its impermanence, it resembles more a site for performance in that the materials associated with the performance event disappear along with the performance once the event has taken place. When {\em Tomorrow Please God} was first published in October 2013, the Tumblr page was red, and in silhouette, flying away through a circle, were four scarlet ibises, Trinidad and Tobago's national bird---the same image on the inside of the cover of {\em Tomorrow Please God}, where the project statement goals are printed. Below the image on Tumblr was an invitation to discover the text by clicking the download link.

\placefigure{Screencapture of douenislands.tumblr.com, taken 27 July 2015.}{\externalfigure[images/douenislands-tumblr.jpg]}
In the run-up to the Alice Yard event, Douen Islands uploaded invitations and videos advertising \quotation{In Forest and Wild Skies}; the \useURL[url31][https://vimeo.com/78231839][][videos]\from[url31] were then used as part of Bagoo's poetry reading:\footnote{See \useURL[url32][http://vimeo.com/78231839]\from[url32].}

\placefigure{Screencapture of 'In Forest and Wild Skies' invitation posted on douenislands.tumblr.com, taken 19 May 2014.}{\externalfigure[images/douenislands-invitation.jpg]}
But rather than develop Tumblr's archival potential, the materials associated with the Alice Yard event were taken down from the site shortly after the event was over, and the page reverted to the initial post with the link to the e-book. After a year and a half, that was replaced with new materials looking forward to an 9 April 2016 BocasLitFest event.

\placefigure{Screencapture of Kis-Ka-Dee announcement posted on douenislands.tumblr.com, taken 29 February 2016.}{\externalfigure[images/kis-ka-dee.jpg]}
While Douen Islands maintains two \useURL[url33][http://www.facebook.com/douen.islands][][Facebook pages]\from[url33] in addition to their \useURL[url34][http://www.instagram.com/douenislands][][Instagram]\from[url34], \useURL[url35][file:///C:\Users\Kelly\Documents\sx\%20archipelagos\%20May\%202016\returned\twitter.com\douenislands][][Twitter]\from[url35], and \useURL[url36][file:///C:\Users\Kelly\Documents\sx\%20archipelagos\%20May\%202016\returned\douenislands.tumblr.com][][Tumblr]\from[url36] accounts, none of them has many followers as of March 2016 or hosts the conversational exchanges that make these media social.\footnote{See \useURL[url37][http://www.facebook.com/douen.islands]\from[url37]; \useURL[url38][http://www.instagram.com/douenislands]\from[url38]; \useURL[url39][http://twitter.com/douenislands]\from[url39]; \useURL[url40][http://douenislands.tumblr.com]\from[url40].} Because none of the Douen Islands accounts have the kind of traffic that would create effective ephemerality on their pages, these pages are more presentational and reminiscent of Web 1.0 pages. The manual intervention of loading up and taking down material on the Douen Islands Tumblr site gestures to the presence and disappearance of posts on heavily used Twitter accounts and, through this Twitterization of Tumblr, allows the Tumblr site to take on the effective ephemerality that characterizes Twitter feeds.

As an e-book published on Tumblr and incorporating the images of tweets, {\em Tomorrow Please God} skirts the boundaries of an example of digital art and Net art. As Robert Simanowski explains the terms, digital art \quotation{depends on digital technology not only for distribution but also for aesthetic effect,} whereas \quotation{connectedness provided by the internet} defines Net art.\footnote{Roberto Simanowski, \quotation{Digital and Net Art,} in Marie-Laure Ryan, Lori Emerson, and Benjamin Robertson, eds., {\em The Johns Hopkins Guide to Digital Media} (Baltimore, MD: Johns Hopkins University Press, 2014), 134.} The digital technology marking the production values of the PDF is particularly evident throughout the text, for example, in the full-spread images of the sea placed at two locations in the text (1, 24--25): the smooth sheen of the screen on which the PDF is meant to be viewed heightens the light glinting off of each ripple and wave of the sea. The most prominent aesthetic effect of the digital technology is the invocation of social media. Drawing from Twitter for a PDF that is then disseminated on Tumblr offers an example of how content can shift across different digital platforms and indicates the social-mediatization of {\em Tomorrow Please God}.\footnote{In their discussion of how Occupy Movement activists moved content from YouTube to Twitter, Kjerstin Thorson et. al.~show that this content shift transforms one platform into a \quotation{communal good} as a repository of resources as well as a \quotation{connective good, linking members of a public together and giving them at least the possibility of communication.} Kjerstin Thorson et al., \quotation{YouTube, Twitter, and the Occupy Movement,} {\em Information, Communication, and Society} 16, no. 3 (2013): 441.} To recall Raley's discussion of how a text can be transformed into text effect through \quotation{the shift from static pages to real-time streaming data,} here the text is the gesture to real-time streaming data within the static pages of a PDF that produces the transient and conversational experience of text effects within a text. {\em Tomorrow Please God} extends the possibility of communication even though the social ecology of these tweets is cut off through their very inclusion in an un-networked e-book.

{\em Tomorrow Please God} explores the project's aim to heal and integrate the douen, a project that itself is in process, as we see in the present progressive of Bagoo's discussion of the project's trajectory: \quotation{{\em Douen Islands} is about growing up in a world while coming to terms with injustice in all its forms: violence and crime, racism, homophobia, religious bigotry, classism, stigmatisation. It is about moving from a place of blind rage to a place approaching knowledge.}\footnote{Bagoo, quoted in Laughlin, \quotation{{\em Douen Islands} and the Art of Collaboration.}}

Beginning from the douen's social position of being subjected to injustice, Douen Islands explores how to pivot away from the douen's fearful and angry compulsions to create paths that are not limited to the detours formed by backward feet. But by evoking the backchannel, the tweets included in {\em Tomorrow Please God} hold at bay the douen's conversion from marginalized outcast to social actor, and the images of footprints and feet infer that the douen retains its devious potential for misdirection. Chen's focus on the formation of the body insists on the douen's fantastical physicality and the ambivalent traces it produces. Drawings of the skeletal and muscular structure of heads, spines, legs, and especially feet and footprints recur in the margins and in the background and are occasionally centered on the page. While the prevalence of these visual elements reminds the reader of the specificities of the douen's body, the douen's physical differences are not made an object of horrified fascination. One full-page spread, for example, places a diagram demonstrating how a child's back straightens out over the course of two years next to the drawing of the skeleton of a foot:

\placefigure{Tomorrow Please God, 22--23.}{\externalfigure[images/tomorrow-please-god.jpg]}
Because the heads are anatomical drawings in profile where the skin is not shown, and because the feet are not connected to the rest of a body, it is impossible to know if this foot is on backward or forward. Referring to a part of the body that does straighten itself out without intervention in juxtaposition with an isolated foot could mean that the foot is the next to be straightened out, or it could place them in contrast with one another.

The difficulty of ambiguous reading of the feet throughout {\em Tomorrow Please God} is encapsulated in the use of the Douen Islands's avatar in two separate instances in the e-book (7, 13). With each foot facing the exact opposite direction, the process of straightening out the feet seems halfway accomplished and at an interpretive deadlock. This image, with which the entire project identifies itself in all of its social media, raises the questions: Which foot is straight and which is still backward? Which foot will choose the direction for the douen's path? And how might a tracker read the traces that this set of half-straight, half-backward pair of feet will leave behind?

\placefigure{Tomorrow Please God, 12--13.}{\externalfigure[images/tomorrow-please-god-2.jpg]}
The two shadowboxes, with the brain on the left side and the ambi-directional pair of footprints on the right, are elevated above a background of David LeVay's {\em Anatomy, Second Edition}, discussing \quotation{deviated} feet.\footnote{Deviated feet is a birth defect that results in the infant's foot being shaped like a kidney bean, and in most cases can be resolved within a few months, through passive stretching exercises. See Ursula Geisen, \quotation{Common Foot Deformities Draft 2,}} The placement allows the two framed images to borrow the background page's arrow, literally drawing a causal connection from the brain's cognition to the feet's unreadable directionality.

Inspired by the fraught directionality of the avatar's feet, I read the tweets included in the text both front to back, as indicated by their inclusion in a book, and in reverse chronology, as corresponds to Twitter's formal properties. Chen culled four tweets from @wanderroller, Warner's 2013 Twitter feed. Reading these tweets backward I argue that the tweets constitute a counternarrative to the poems within {\em Tomorrow Please God}, a narrative that speaks for the first time on page 22 and ends inside the front cover, creating the sequence:

\startblockquote
raise me (21)

imagine giving birth to me (20)

love is eyes (facing the table of contents)

you know me (inside the cover page)
\stopblockquote

A brief bildungsroman, the tweets begin with the douen's birth in the reader's imagination and offer the reader a set of instructions for how to interact with the douen in order to counter its folk origins as a figure of abandonment. Consider, however, reading the tweets in the same order as the rest of the text. The narrative in that order appears: \quotation{you know me} (inside the cover page), \quotation{love is eyes} (facing the table of contents), \quotation{imagine giving birth to me} (20), \quotation{raise me} (21). To be introduced to a character with the phrase \quotation{you know me} before even knowing that the douen is the subject of the e-book and without having had the chance to explore the douen's remixed identity through the text seems bewildering, if not menacing---a tone that is very in keeping with the fear the douen can induce. Similar to the profile image of feet facing in two directions, it is the combination of narratives created by reading in both directions simultaneously that creates an ambiguous counternarrative to the project's aims to resolve the douen's rage and neglect.

These four tweets exist within {\em Tomorrow Please God} as a potential counternarrative and they also function as part of Chen's visual vocabulary because of the digital manipulation that Chen performed: he erased the dates from the tweets and did not include the number of retweets or favorites, thereby lifting the tweets out of the context of the stream that originally constituted the tweets' citational ecosystem by not expanding them, then copying each three times:

\placefigure{Tomorrow Please God, 20--21.}{\externalfigure[images/tomorrow-please-god-3.jpg]}
The inability to place the douen, as the result of Chen's digital manipulation, operates on both a spatial and a temporal level: by cropping out the time-stamps that are normally marked on each tweet, next to the name and handle, the tweets can be read as being outside of the immediacy that usually governs the relevance and accessibility of utterances on Twitter. Moreover, repeating the phrases is meant, according to Chen, to \quotation{amp up the feeling of being in a forest with douens.}\footnote{Chen, e-mail message to the author.} Coming from multiple locations within the darkened forest, the source of their multiple and demanding reiterations would not be identifiable. In this sense, the douen's melancholy and echoing hooting never loses its timeliness, and the tweets function as constantly present demands. So while the unlocatability of the douen is usually because of its backward paths that make its location unreadable, here, the use of Twitter combined with the removal of time-stamps produces a temporal unknowability, and the absence of temporal fixity for the tweets allows the tweets to remain legible backward, forward, and on endless loop.

To invite marginalized subjects to participate in broader conversations about their exclusion is a fraught, if necessary, project: What is the price of entry for these subjects? In the case of the douen, as implied in the Douen Islands project goals, it might seem to require giving up one's primary mode of self-protection, to straighten out its deviations, and to produce paths that are easily legible. In {\em Tomorrow Please God}, Bagoo, Chen, and their collaborators present poems and images that uncover the douen's face and invite the douen inside, but they also include a tweeting douen that continues to assert its unreadability---to keep at least one foot on backward and retain the potential for subversively ambivalent readings. In order to manage this dynamic set of voices that move in different directions, Douen Islands mobilizes the particular mixture of the private and social that is characteristic of microblogging platforms. Evoking the traditional folklore figure of the douen by connecting it with Twitter, a social media platform famous in part for its use by marginalized groups to organize themselves, allows Douen Islands to experiment with how a douen might begin to craft a persona and writing voice. And it is in exploring the writing practices and functionalities of Twitter that, despite the desire for the douen to progress toward normalization, working with the representation of the douen retains its capacity for devious and deviant paths.

\thinrule

\page
\subsection{Jeannine Murray-Román}

Jeannine Murray-Román (PhD Comparative Literature, UCLA) is Assistant Professor of French and Spanish in the Department of Modern Languages and Linguistics at Florida State University. Her areas of specialization are comparative Caribbean literatures and Cultures with an interest in digital humanities and postcolonial, transnational, and performance studies. Her book, {\em Performance and Personhood in Caribbean Literature: From Alexis to the Digital Age} was published by University of Virginia Press in 2016.

\stopchapter
\stoptext