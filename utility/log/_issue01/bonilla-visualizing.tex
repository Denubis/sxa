\setvariables[article][shortauthor={Bonilla, Hantel}, date={May 2016}, issue={1}, DOI={10.7916/D8CV4HTJ}]

\setupinteraction[title={Visualizing Sovereignty: Cartographic Queries for the Digital Age},author={Yarimar Bonilla, Max Hantel}, date={May 2016}, subtitle={Visualizing Sovereignty}]
\environment env_journal


\starttext


\startchapter[title={Visualizing Sovereignty: Cartographic Queries for the Digital Age}
, marking={Visualizing Sovereignty}
, bookmark={Visualizing Sovereignty: Cartographic Queries for the Digital Age}]


\startlines
{\bf
Yarimar Bonilla
Max Hantel
}
\stoplines


{\startnarrower\it This essay asks how visual representations of the postcolonial Caribbean are shaped by, and in turn could reshape, the political imaginary of sovereignty. Describing several different experiments with form---from conventional maps to temporal charts to animation---it argues that visualizing sovereignty is a first step in retheorizing the meaning of sovereignty itself beyond the regulatory limits of insular, nation-state autonomy. The authors call for collaborative efforts to create \quote{prophetic cartographies} attuned to alternative political currents and the possibility of imagining the Caribbean otherwise. \stopnarrower}

\blank[2*line]
\blackrule[width=\textwidth,height=.01pt]
\blank[2*line]

The genesis for this project arose during a fit of last-minute teaching prep for one of the authors. While putting the final touches on a lecture for a Caribbean Perspectives class, Yarimar struggled with how to visually represent the political terrain of the Caribbean and specifically how to convey the predominance of nonsovereign societies in the region. She wanted to convey that places like Guadeloupe, Curacao, Puerto Rico, and the Cayman Islands did not constitute exceptions to Caribbean political history but rather emblematic examples of the political forms that characterize the region as a whole. When Google Image proved useless, and with little time to spare before class, she settled on altering a basic map of the Caribbean by using multicolored arrows to mark the different societies that were not independent nation-states. The year was 2010 and the full political effects of the Haitian earthquake were still unknown, with news emerging that a supranational body might replace the Haitian government, so before rushing off to class Yarimar impulsively added a question mark at the center of the map, representing not just the political uncertainties faced by one Caribbean society but also the larger question of what exactly sovereignty has meant for the region as a whole and how one might seek to visually represent it.

\placefigure{Map of the Caribbean}{\externalfigure[images/yarimar-map.jpg]}
In the years that followed, this initial cartographic impulse has led to a series of projects that have sought to disrupt how we both {\em visualize} and {\em theorize} Caribbean sovereignty. In this essay we discuss this trajectory and how it led to the creation of an animated map of the Caribbean, developed by the two authors, that seeks to foreground the analytical importance and predominance of nonindependent societies while also rethinking how sovereignty itself can be imagined, conceptualized, theorized, and visualized beyond the constraints of Western cartography.

\subsection[reference={the-map-and-the-territory},
bookmark={The Map and the Territory},
title={The Map and the Territory}]

Classic political theory has traditionally represented sovereignty as supreme, absolute, territorially confined, and vertically rooted in the apparatus of the state. The traditional philosophy of sovereignty, rooted in the treaty of Westphalia in 1648, poses that national governments hold supreme authority over their internal affairs and that other states should not intervene under exception of threat or obligation of alliance.\footnote{See Robert Jackson, {\em Sovereignty: The Evolution of an Idea} (Cambridge: Polity, 2007).} Recent scholarship has sought to complicate this notion by demonstrating the extent to which formally independent nations (mostly in the Global South) have been palpably shaped by outside interests, supranational organizations, and both internal and external nongovernmental actors. These scholars have emphasized how nonstate, trans-state, and suprastate actors (such as the United Nations, the International Monetary Fund, international nongovernmental organizations, military contractors, lending firms, and others agents) are increasingly taking up what were long thought to be the privileged duties of state governments.\footnote{See Arjun Appadurai, \quotation{Sovereignty without Territoriality: Notes for a Postnational Geography,} in Patricia Yaeger, ed., {\em The Geography of Identity} (Ann Arbor: University of Michigan Press, 1996), 40--58; Wendy Brown, {\em Walled States, Waning Sovereignty} (New York: Zone Books, 2010); John L. Comaroff and Jean Comaroff, \quotation{Reflections on the Anthropology of Law, Governance, and Sovereignty,} in Franz von Benda-Beckmann et al., eds., {\em Rules of Law and Laws of Ruling: On the Governance of Law} (Surrey, UK: Ashgate, 2009), 31--60; James Ferguson, {\em Global Shadows: Africa in the Neoliberal World Order} (Durham, NC: Duke University Press, 2006); Thomas Blom Hansen and Finn Stepputat, \quotation{Sovereignty Revisited,} {\em Annual Review of Anthropology} 35 (2006): 295--315; Thomas Blom Hansen, {\em Sovereign Bodies: Citizens, Migrants, and States in the Postcolonial World} (Princeton, NJ: Princeton University Press, 2005); Carolyn Nordstrom, \quotation{Shadows and Sovereigns,} {\em Theory, Culture, and Society} 17, no. 4 (2000): 35--54; and Aihwa Ong, {\em Neoliberalism as Exception: Mutations in Citizenship and Sovereignty} (Durham, NC: Duke University Press, 2006).} The perceived \quotation{newness} of these processes is, however, often overstated. For, as historians of empire have shown, claims to sovereignty have always been fractured, layered, negotiated, and contested.\footnote{See Lauren A. Benton, {\em A Search for Sovereignty: Law and Geography in European Empires, 1400--1900} (Cambridge: Cambridge University Press, 2010); Jack P. Greene, {\em Negotiated Authorities: Essays in Colonial Political and Constitutional History} (Charlottesville: University Press of Virginia, 1994); and Philip J. Stern, {\em The Company-State: Corporate Sovereignty and the Early Modern Foundations of the British Empire in India} (Oxford: Oxford University Press, 2012).}

This is nowhere more evident than in the Caribbean, where the history of postcolonial sovereignty has unfolded as a story of sovereignty challenged, contested, disavowed, and undermined. From the moment of Haiti's declaration of independence in 1804, Caribbean sovereignty has been an open question. The political authority of the first black republic posed a challenge to the international system of states at the time, requiring the international community to create new protocols for trade and diplomacy with which to engage Haiti economically while skirting the political and ontological challenges posed by its revolution.\footnote{See Michel-Rolph Trouillot, {\em Silencing the Past: Power and the Production of History} (Boston: Beacon, 1995); and Julia Gaffield, {\em Haitian Connections in the Atlantic World: Recognition after Revolution} (Chapel Hill: University of North Carolina Press, 2015).} Indeed, it was only after Haiti's economic sovereignty had been fully undermined through the imposition of crippling debt that its political sovereignty was nominally recognized. Similarly, Cuba's independence in 1898 was conditional on the securing of a permanent relationship with the United States. In the wake of its war of independence with Spain, Cuba remained under US military occupation until 1901, at which point the US-authored Platt Amendment was incorporated into the Cuban constitution, guaranteeing permanent use of the Guantanamo military base, authorizing future US interventions, and barring Cuba from entering into foreign trade with other foreign powers. In the immediate aftermath of the formation of the United Nations in the 1940s, the Caribbean continued to serve as a site of political experimentation with the development of alternative formulas of decolonization that offered limited self-governance, such as the formation of the Puerto Rican commonwealth, and stratified forms of inclusion, as with the creation of the French overseas departments. This period also led to significant exploration with models of federation, including the short lived West Indies Federation (1958--62).

This history of fractured, uneven, contested, and negotiated sovereignty continues to shape the region as a whole, and at present the {\em majority} of societies in the Caribbean are not independent nation-states but rather protectorates, territories, departments, and commonwealths (see \useURL[url1][\%7B\%7Bsite.baseurl\%7D\%7D/assets/extras/issue01-bonilla-appendix.pdf][][appendix]\from[url1]). In addition, the Caribbean also holds a large number of nonsovereign enclaves: military bases, privately owned islands, semiautonomous tourist resorts, free-trade zones, tax havens, wildlife preserves, satellite launching stations, detention centers, penal colonies, floating data centers, and other spaces of suspended, subcontracted, usurped, or imposed foreign jurisdiction that challenge the principles of bounded territorial authority associated with the Westphalian order. Moreover, even the nominally independent nations of the Caribbean have repeatedly had their political and economic sovereignty challenged through military invasions, electoral interference, security legislation, and the multiple barriers placed on international trade throughout the global South. As a result, sovereignty in the Caribbean is best understood as a contested claim and an imposed ideal rather than an actually existing condition. It is for this reason that we suggest that the Caribbean is best understood (borrowing from Antonio Benítez-Rojo) as a {\em nonsovereign archipelago}, where patterns of constrained and challenged sovereignty can be said to repeat themselves.\footnote{Antonio Benìtez-Rojo, {\em The Repeating Island: The Caribbean and the Postmodern Perspective} (Durham, NC: Duke University Press, 1996), 2. 
}

However, rather than representing the Caribbean as a site of problematic sovereignty, we would like to emphasize how Caribbean history brings into question the notion of sovereignty itself. We believe that {\em sovereignty} needs to be understood as part of that family of words that Michel-Rolph Trouillot describes as \quotation{North Atlantic universals.}\footnote{Michel-Rolph Trouillot, \quotation{North Atlantic Universals: Analytical Fictions, 1942--1945,} {\em South Atlantic Quarterly} 101, no. 4 (2002): 848.} These concepts do not seek to merely describe the world but to constrain it possibilities. In other words, these are the native categories of the West, as a project not a place. In describing the Caribbean as a site of nonsovereignty, we are thus suggesting that it is not simply a site where \quotation{ordinary sovereignty} wanes and fails but also a fertile site from which to contest, disrupt, and reimagine notions of sovereignty, autonomy, freedom, liberty, and self-determination beyond the canon of political theory.\footnote{Yarimar Bonilla, \quotation{Ordinary Sovereignty,} {\em Small Axe}, no. 42 (November 2013): 152--65.}

Cartography plays a constitutive role in this process, as it has historically reproduced the idea of the sovereign nation-state as a bounded entity and naturalized it as the site of proper politics. Geographic histories reveal that the visualizing power of the map preceded the formation of sovereign states and created the conditions of possibility for colonial expansion. As Jordan Branch argues, maps reshaped our perceptions of legitimate political authority and organization, paving the way for the shift from medieval to modern political structures.\footnote{Jordan Branch, \quotation{Mapping the Sovereign State: Technology, Authority, and Systemic Change,} {\em International Organization} 65, no. 1 (2011): 2; see also Philip E. Steinberg, \quotation{Insularity, Sovereignty, and Statehood: The Representation of Islands on Portolan Charts and the Construction of the Territorial State,} {\em Geografiska Annaler} 87, no. 4 (2005): 2.}~Of course, this sense of~{\em insular}~sovereignty casts the production of European modernity as an internal, autarkic shift. Yet, the global power of mapmaking in the production of sovereign states took hold through colonial circuits of exchange and violence. The map is \quotation{a technology of possession,} as Anne McClintock argues, \quotation{promising that those with the capacity to make such perfect representations must also have the right of territorial control.}\footnote{Anne McClintock, {\em Imperial Leather: Race, Gender, and Sexuality in the Colonial Contest} (New York: Routledge, 1995), 28.} Again, the map reifies the truth of what it represents, promising and delivering virgin lands and nonsovereign territories in need of discovery, settlement, borders, and territorial authority.

If the philosophical and material projects of modernity and colonialism took shape through a geographic reordering of sovereignty, our contemporary moment suggests another emerging shift. Sylvia Wynter argues that the various imaginaries of decolonization made possible \quotation{a new opening---that of the collective challenge made to . . . symbolic representational systems,} and hence, the possibility of a \quotation{new world view.}\footnote{Sylvia Wynter, \quotation{1492: A New World View,} in Vera Lawrence Hyatt and Rex Nettleford, eds., {\em Race, Discourse, and the Origin of the Americas} (Washington, DC: Smithsonian Institution Press, 1995), 50.} The crucial question then becomes, What spatiotemporal models can grasp and even utilize such an opening? The problem, Wynter suggests, is that much contemporary political and geopolitical theory \quotation{mistakes the map for the territory,} by working within the ideological coordinates of Eurocentric, Western thought rather than investigating the historical processes that bequeathed those mores to the present.\footnote{See Sylvia Wynter, \quotation{On How We Mistook the Map for the Territory, and Re-Imprisoned Ourselves in Our Unbearable Wrongness of Being, of Desetre: Black Studies Toward the Human Project,} in Lewis R. Gordon and Jane Anna Gordon, eds., {\em Not Only the Master's Tools: African-American Studies in Theory and Practice}, (Boulder, CO: Paradigm, 2006), 107--69.} Orienting struggles according to certain maps attends to representations of political space {\em as if} these representations faithfully and transparently described contemporary horizons of possibility. Yet, the landscape of possibility always exceeds the limits of representation. Moreover, the map is itself a function of a foundational set of codes concerning who controls visual representation and what counts as representable in the first place. Attending to those codes themselves, rather than to simply the maps they generate, profoundly disrupts the cartographic gaze and its imposed limits.

One such code, we argue, is the evolutionary discourse of political development, which produces a naturalized view of the nation-state that then renders nonindependent and nonsovereign Caribbean islands as exceptional, paradoxical, and even pathogenic.\footnote{See Yarimar Bonilla, {\em Non-sovereign Futures: French Caribbean Politics in the Wake of Disenchantment} (Chicago: University of Chicago Press, 2015).} Confined to a map of the properly sovereign Caribbean, those \quotation{paradoxical} societies that articulate freedom without the grammar of national independence, or those nominally independent countries that fail to achieve bounded autonomy, are found wanting or inexplicable. If we instead begin our work from the space between the map and the territory---and the political processes that render them transferable---it becomes possible to see those \quotation{exceptional} cases as not always already failed but as generative sites for alternative visions of sovereignty itself.

Wynter's deployment of the map as the key symbol for this epistemic mistake is certainly intentional. Cartography spatially produces and reproduces political-economic arrangements while retroactively naturalizing them.\footnote{John Pickles, {\em A History of Spaces: Cartographic Reason, Mapping, and the Geo-Coded World} (New York: Routledge, 2003); for a discussion of mapping and the \quotation{social reproduction of space} in Caribbean political formations, see Michaeline Crichlow and Patricia Northover, {\em Globalization and the Post-creole Imagination: Notes on Fleeing the Plantation} (Durham, NC: Duke University Press, 2009), 15--41.} Through this visual power, a recursive relationship develops between the boundary projects of modernity and the limits of identity and expression, thus reifying the political and epistemological perspective the map purports to describe. Charles de Gaulle's infamous declaration, made upon looking at a transatlantic map---\quotation{Between Europe and America I see only specks of dust}---indexes this cartographic effect. Edouard Glissant uses this quotation as the epigraph to {\em Caribbean Discourse}, making it the point of departure for reimagining the geographic meaning of the Caribbean and \quotation{the future of small countries.}\footnote{Edouard Glissant, {\em Caribbean Discourse: Selected Essays}, trans. J. Michael Dash (Charlottesville: University Press of Virginia, 1989), 3; originally published as {\em Le discours antillais} (1981; repr., Paris: Gallimard, 1997).} Those little bits of dust, he hopes, might help produce alternative spatiotemporal models of identity and collective life beyond the fortress island of national sovereignty or the \quotation{flat world} of globalization.\footnote{Max Hantel, \quotation{Rhizomes and the Space of Translation: On Edouard Glissant's Spiral Retelling,} {\em Small Axe}, no. 42 (November 2013): 100--12.}

If the visual representations we use to capture questions of political sovereignty are inextricably intertwined with how and what we imagine sovereignty to be, then the question becomes how to visually represent this process, where the production of space is both the precondition and the result of imagining sovereignty itself. In some sense, this question poses a visual correlate to the longstanding concern in Caribbean thought with the displacement of universal history, such as Glissant's experiments with alter-chronologies or Trouillot's interrogations of Western historiography. Cartographic experimentation also questions any neat division between spatial and temporal categories by demonstrating how the production of space through a \quotation{mapping of the present} entails a specific mode of articulating the past and future.\footnote{Crichlow and Northover, {\em Globalization and the Post-creole Imagination}, 22--23.}

Building on these ideas, we thus consider alternative cartographic and geographic approaches that begin from the tense and tenuous relationship between the map and the territory and which are grounded in the specificity of Caribbean political history. Our goal, then, is not simply to update previous textual or analogue modes of engagement but to trace and enact their transformation through digital subjectivities and new cartographic technologies. We thus ask, What could visual representations of the Caribbean become if no longer anchored by political sovereignty as a regulatory ideal of postcolonial independence or economic development? Faced with de Gaulle's dismissal of Caribbean specks of dust as inferior to the point of irrelevance, Glissant demanded a \quotation{prophetic vision of the past} beyond \quotation{schematic chronology} or \quotation{nostalgic lament.}\footnote{Glissant, {\em Caribbean Discourse}, 64.} The Caribbean novelist had to reach into the past without the teleological comfort of Western history to invent futures from the traces, the scraps, the cinders of collective memory and everyday life. While his interlocutors often limit this discussion to the novel and the word, we take the insistence on vision as a point of departure: a prophetic cartography of the postcolonial Caribbean.

\subsection[reference={cartographies-for-the-digital-age},
bookmark={Cartographies for the Digital Age},
title={Cartographies for the Digital Age}]

Until now most cartographic representations of the Caribbean have failed to reveal the complexity of the region's political topography and how it challenges normative understandings of political sovereignty. The fact is that the complexity of the region is difficultly captured through traditional cartographic forms, which tend to flatten out political histories. We would like to suggest that through the use of digital technologies more complex and sophisticated forms of visual argumentation are increasingly becoming possible. Through the use of time lapse maps, scholars are able to convey shifts and changes over time, unsettling our views of contemporary borders and political relationships. For example, in his project {\em The Invasion of America: How the United States Took Over an Eighth of the World}, Claudio Saunt documents the land cessions through which the United States slowly expanded Westward, thus showing the slow and violent reduction of Native homelands into scattered reservations.\footnote{Claudio Saunt, {\em Invasion of America: How the United States Took Over an Eighth of the World}, video, 1:27, published 2 June 2014 by eHistory.org, \useURL[url2][http://youtu.be/pJxrTzfG2bo]\from[url2]. For an interactive map, see \useURL[url3][http://invasionofamerica.ehistory.org]\from[url3].} In addition to the time-lapse display, which offers a sense of how this process unfolded, the website also features an interactive source map that allows users to see detailed information on how individual cessions were disputed and challenged, thus offering both a sense of the larger process and textured detail of its constitutive events.

\placefigure{Invasion of America -- Video url: https://www.youtube.com/embed/pJxrTzfG2bo}{\externalfigure[images/invasion-of-america.jpg]}
Vincent Brown has developed a similarly rich visual argument in his online animated map {\em Slave Revolt in Jamaica, 1760--1761: A Cartographic Narrative}, in which he documents a previously misunderstood revolt showcasing the tactical maneuvers made by the rebels. By combining geographic data on slave uprisings culled from various forms of data (witness accounts, legal documents, literary texts, and others), he is able to reveal a spatial logic to slave resistance that was obscured in colonial accounts, suggesting a greater degree of strategic organization among insurgents than previously imagined.\footnote{Vincent Brown, {\em Slave Revolt in Jamaica, 1760--1761: A Cartographic Narrative}, \useURL[url4][http://revolt.axismaps.com]\from[url4]. See also Vincent Brown, \quotation{Mapping a Slave Revolt: Visualizing Spatial History through the Archives of Slavery,} {\em Social Text}, no. 125 (December 2015): 134--41.}

\placefigure{Slave Revolt in Jamaica, 1760--1761: A Cartographic Narrative}{\externalfigure[images/brown-revolt.jpg]}
Brown argues that if we are to take seriously the opportunities afforded by digital forms of scholarship, we must remain attentive to how {\em design} and {\em interface} constitute modes of scholarly argumentation.\footnote{Vincent Brown, \quotation{Narrative Interface for New Media History: Slave Revolt in Jamaica, 1760--1761,} {\em American Historical Review} 121, no. 1 (2016): 176--86.} Whether we engage with the symbolic traditions of maps, charts, and diagrams or new forms of data visualization, we must thus think carefully about how representational choices in the use of color, sound, and format constitute rhetorical strategies. Building on, rather than abandoning, previous debates about textual representation, digital scholars must therefore remain attentive to how their narratives are plotted through the use of various representational formats and symbolic orders---even as they challenge them. For example, on the website \useURL[url5][http://www.twoplantations.com][][{\em Two Plantations}]\from[url5], produced by Harvard University's History Design Studio, the social history of enslaved families is diagramed through the use of family trees.\footnote{{\em Two Plantations: Enslaved Families in Virginia and Jamaica}, \useURL[url6][http://www.twoplantations.com]\from[url6].} The website shows both the possibilities and the limitations of adopting this graphic form to represent the lives of families whose structures of kinship were violently molded and could never be based solely on biological descent.\footnote{Brown, \quotation{Narrative Interface,} 178.} As Vincent Brown suggests, even when they reach their limits these representational experiments can open up new questions about the most appropriate representational strategies for the material in question and can generate new questions by uncovering patterns and revealing challenges that were previously unperceived.

\subsection[reference={mapping-the-postcolonial-caribbean},
bookmark={Mapping the Postcolonial Caribbean},
title={Mapping the Postcolonial Caribbean}]

Our own efforts at mapping the Caribbean political landscape have yielded similar insights. After her initial experiment attempting to alter pre-existing maps, Yarimar enlisted the help of her research assistant at the time, Landon Yarrington, to create a map that would foreground the nonindependent societies of the Caribbean region (see fig. 4). The goal was to visually depict the large number of Caribbean societies (listed in table 1) that were not independent nation-states and to use color to portray the diversity of political forms in the region.

\placefigure{Map created by Landon Yarrington in 2011.}{\externalfigure[images/landon-map.jpg]}

The creation of this map quickly revealed some of the challenges of depicting the Caribbean in cartographic form. First, choices had to be made about how to include societies in the Southern continent. Although politically and historically the coast of South America is often included in definitions of the region, visually the inclusion of the South American coast results in a dwarfing of the smaller Caribbean islands. Given our interest in representing the smaller islands of the Caribbean, this posed a significant problem. In addition, the presence of the large landmass of the continent and negative oceanic space had a similar rhetorical effect to the oft-cited metaphor that the nonindependent Caribbean represented little more than \quotation{specks of dust.} The first challenge thus became how to fit the story we wished to tell into the narrative form of the geographic map.

Working with cartographer Jeff Blossom at the Harvard Center for Geographic analysis, we developed a new map that sought to tackle these cartographic challenges (see fig. 5). In this map we used an inset to facilitate the view of the Eastern Caribbean, and we distorted the size of the ABC islands (Aruba, Bonaire, and Curacao), to bring them into better view. We also erased the continental landmass, portraying only those South American societies most commonly imagined as part of the Caribbean (Belize, Guyana, Suriname, and Guiana). Once again, we used color to visually demarcate political jurisdictions.

\placefigure{Map created by Jeff Blossom and Yarimar Bonilla in 2015.}{\externalfigure[images/blossom-map.jpg]}

Although we played with size and scale, the map retains a sense of territorial positioning, allowing one to view how political affiliations are distributed geographically. The erasure of the Southern continent aids in focusing the eye on the Caribbean societies in question, though it implicitly reinforces underlying assumptions about the boundaries of the Caribbean region. This occludes historical and cultural continuities across the South and Central American region and fails to indicate the ways places like Panama, Columbia, Venezuela, Mexico, and even Brazil form part of the framework of the Caribbean.\footnote{See Edouard Glissant, \quotation{Creolization in the Making of the Americas,} {\em Caribbean Quarterly} 54, nos. 1--2 (2008): 81--89; and Roman de la Campa, {\em Latin Americanism} (Minneapolis: University of Minnesota Press, 1999).} Moreover, an understanding of the complex history of the southern Caribbean islands, and of both the economic and social structures of places like Trinidad and Curacao, is almost impossible without an understanding of their close ties to the Latin American coast. These absences are particularly significant in the context of an explicitly cartographic interface that implies a fidelity to the geographic terrain. The southern coast is then not just absent but visually silenced in this form of representation. Moreover, the use of the cartographic form to convey an argument that is not solely or even primarily geographic makes it difficult for the viewer to comprehend the larger argument about the shifting and contested nature of political sovereignty.

In order to address these issues we decided to create a temporal map of the postcolonial Caribbean (see fig. 6). More of an infographic than a map, it features Caribbean societies ordered by the year in which they shifted away from a colonial status---either by becoming independent or by entering into a \quotation{postcolonial} arrangement with their colonial centers.\footnote{For most societies, we chose the date they were removed from the United Nations list of non-self-governing societies; for those that remain on the list, we chose the date they obtained the status they have today. We understand these arrangements as \quotation{postcolonial,} not in the sense of a formal rupture with colonialism but in terms of a shift in how empire and colonialism itself figure in the political imagination.}

\placefigure{Temporal map of the postcolonial Caribbean created by Jeff Blossom and Yarimar Bonilla in 2015.}{\externalfigure[images/temporal-map.jpg]}

Organizationally, the temporal map poses several alternatives to a traditional map. First, by playing with color and scale, we disrupt the \quotation{realist} effect of traditional geographic maps, making it clear that the map is not the territory. Second, although we retain the shape and outline of different Caribbean societies, we render the {\em scale} of each island equivalent, thus flattening out the differences in size between \quotation{large} islands like Cuba and Haiti and \quotation{small} islands like Martinique or Dominica. This move dislodges the equation of smallness with inferiority and the analytical irrelevance usually attributed to those residing in a \quotation{small place.}\footnote{As Jamaica Kincaid suggests, \quotation{The people in a small place cannot see themselves in a larger picture, they cannot see that they might be part of a chain of something, anything.} Jamaica Kincaid, {\em A Small Place} (New York: Farrar, Straus, and Giroux, 2000), 52.} In addition, the use of color departs from the traditional distribution of color on political maps---which reinforces the idea of bounded states--- and instead suggests political configurations that surpass geographic lines.

The temporal map foregrounds, in other words, what geographer Neil Smith calls \quotation{the production of scale,} demonstrating that debates over scale express social as much as geographic concerns over boundaries, locations, and sites of experience.\footnote{Neil Smith, {\em Uneven Development: Nature, Capital, and the Production of Space} (Athens: University of Georgia Press, 2008), 229.} If scale is constructed, then not only do different power formations shift scales at different strategic moments---as, for example, in the reorganization of nation-states into supranational organizations---but alternative political imaginaries can also emerge by \quotation{jumping scale.}\footnote{Neil Smith, \quotation{Homeless/Global: Scaling Places,} in Jon Bird et al., eds., {\em Mapping the Futures: Local Cultures, Global Change} (London: Routledge, 1993), 90.}

Indeed, by shifting scales and placement this visualization makes visible several patterns. First, at a quick glance one can see that there were two distinct moments of political independence in the region. The first is ushered in by the Haitian Revolution in 1804 and culminates with the Spanish American War in 1898, which led to the granting of independence to Cuba. This is followed by a period of political experimentation in which various formulas of inclusion were developed throughout the French, Dutch, US, and British possessions in the Caribbean. It is only after 1962 (with the dissolution of the West Indies Federation) that national sovereignty (now under the auspices of the United Nations postcolonial sovereignty project) becomes an imperative once again, but only for some of the former British islands.

Plotting the Caribbean through postcolonial time disrupts traditional groupings through geography and size (Windward/Leeward, Greater/Lesser Antilles, etc.) and through linguistic/colonial history (hispanophone, francophone, etc.). It offers by contrast three distinct groupings that cut across these traditional boundaries: (1) Caribbean societies that experienced nineteenth-century nation-building projects, (2) Caribbean societies that experimented with nonsovereign forms of decolonization, and (3) Caribbean societies that experienced the postwar decolonization project. This grouping disrupts traditional arguments about postcolonial sovereignty in the region---particularly the idea of the mid-twentieth-century postwar project of decolonization as the norm against which other exceptions are cast. Now postwar independence appears as a contingent outcome rather than a naturalized telos or evolutionary hierarchy.

Of course, our goal here is not to offer a new typology for Caribbean societies but rather to create {\em a tool of inquiry}. That is, by revealing patterns and opening up new questions about those patterns, images such as these could ideally spur new thinking about political sovereignty and its shifts over time in relation to a larger set of questions about what connects or diffracts \quotation{the Caribbean} itself as a cartographic object: physical geography, linguistic groupings, palimpsestic colonial histories, connections to other regional groupings like Latin America or the US South, and so forth. In addition, the suggestion that these images constitute tools of inquiry also implies that they are not final products, but perpetual works in progress. In many ways we create these images in order to supersede them as we come to new understandings and new visions of the processes in question.

In this case, while helpfully bringing into relief alternative ways of imagining the region, it quickly became evident that the static nature of the image failed to fully convey the historical processes we sought to illuminate. By fixing each society at a particular point in time the image fails to represent the contested and fluctuating nature of these political arrangements and obscures political configurations that although temporary are key to understanding the contemporary terrain. For example, the map in figure 6 does not allow us to represent the temporary existence of the West Indies Federation, the failure of which sparked the wave of 1960s independence movements that followed. It also fails to show that Puerto Rico was an unincorporated US territory since 1917, which suggests that the era of experimentation with formulas of incorporation began much earlier than what is revealed on the final graph. Additionally, even as it plots through time, its neat separation of bounded societies makes it difficult, for instance, to understand the complications of political sovereignty at the border of Hispaniola and the ways in which Haiti's independence is both a condition and a complication of the Dominican Republic's. Lastly, the representation of independent nations as solidly and unwaveringly blue implies that political independence is a final endpoint and ultimate goal for Caribbean societies. It wrongly implies that once achieved, national sovereignty is not perpetually challenged through military occupations and other forms of political and economic intervention.

In order to better convey these relationships, we recruited the assistance of a professional animator to set the above image into motion. The final product retains the effects of the temporal map but challenges the static effect of the original by allowing us to represent movement, contingency, temporary arrangements, and the constant challenges and reorganizations that have characterized the region.\footnote{\useURL[url7][https://vimeo.com/169690419]\from[url7]}

\placefigure{Visualizing Sovereignty in the Postcolonial Caribbean, 2016; animated video designed by Yarimar Bonilla and Max Hantel and produced by Kindea Labs. -- Video url: https://player.vimeo.com/video/169690419?autoplay=1}{\externalfigure[images/bonilla-video.jpg]}
The video begins with a traditional geographic map of the Caribbean but then releases Caribbean societies from their static positioning in order to open up the possibility of new organizing principles. Indeed, the point of departure for the animation is to question how notions of sovereignty have traditionally ordered our understanding of the Caribbean and to explore how we might imagine disaggregating and aggregating the Caribbean differently.

We begin with a colonial grouping based on the various European powers staking claims in the region at the dawn of the nineteenth century. Haiti's independence in 1804 then ushers in the postcolonial Caribbean, spurring the movement of each island from their colonial grouping to a charted date when their \quotation{postcolonial sovereignty} qualitatively shifts. Although the end point is visually similar to the temporal map, the animated video presents these as an uneven and nonlinear process rather than as a finalized outcome. Indeed, during the collaboration with the animators, we realized how projecting the final frame in advance---as a predetermined outcome for the video---ended up constraining the questions we could visually pose. Once we gave up on knowing with certainty where each movement might lead, or even which events properly counted as \quotation{relevant,} the animation became a much messier and sometimes difficult process of bricolage---closely resembling the twists and turns of postcolonial sovereignty.

Polities now move within and between different statuses, not only vis-à-vis colonial arrangements (i.e., from Spanish to US control) but relative to other postcolonial nations. Take, for instance, the Dominican Republic, which initially appears on the chart as part of a unified Hispaniola in 1822. It wins independence from Haiti in 1844, now visually represented as a separate blue territory. Already, a more complex view of the Dominican Republic's relationship to sovereignty emerges as their formal independence is simultaneously conditioned by the Haitian revolution and generated in opposition to it. Today, the Dominican Republic's national celebration of independence day takes place on 27 February, marking independence {\em from Haiti}. The animation points to, moreover, the short period of Spanish restoration when Spain re-annexed the Dominican Republic in 1861 only to retreat again in 1865 after a two-year war with rebels. A finalized map of the postcolonial Caribbean might omit this brief interlude because it failed to retroactively alter the formal event of independence in the Dominican Republic. However, by insisting on the unevenness and contingency of sovereign territoriality, the animation helps mark the constitutive importance of the War of Restoration (a decolonial struggle in which Haitians and Dominicans fought alongside one another) in the historical emergence of a divisible Hispaniola and the seemingly antagonistic political imaginaries dividing it.\footnote{See Pedro L. San Miguel, {\em The Imagined Island: History, Identity, and Utopia in Hispaniola} (Chapel Hill: University of North Carolina Press, 2006).}

Along those lines, the video brings into relief a finer set of gradations of postcolonial sovereignty that remain at the center of many contemporary political challenges in the Caribbean today. The creation in 1954, for example, of the Netherlands Antilles as one administrative unit in the Kingdom of Netherlands groups together almost the entirety of the Dutch Caribbean, excepting Suriname. This initial mode of organization sets the stage not only for Suriname's eventual independence in 1975 but for internal conflicts and struggles for autonomy {\em within} the Dutch Caribbean. Aruba breaks away in 1986, winning autonomy from the Netherlands Antilles but remaining a country within the Kingdom of the Netherlands. Although some of these struggles were initially articulated in the terms of eventual independence, Aruba has maintained its autonomous status while the Netherlands Antilles was dismantled in 2010, reopening the question of sovereignty in the Dutch Caribbean. Thus, with the animation, we hope to show the triangulated relationship between historical colonial groupings (i.e., the Dutch versus the Spanish) and postcolonial modes of organization (i.e., the Netherlands Antilles) in the emergence of contemporary conditions of nonsovereignty.

Second, the video also takes stock of temporary arrangements, fleeting revolutions, and alternative modes of governance to acknowledge not only the legacies of colonial rule but the ongoing experiments with postcolonial transformation. Most notable is the period of the West Indies Federation from 1958 to 1962, both for how it regrouped the collective spatial politics of the Caribbean and for how its eventual dissolution shaped the trajectories (toward independence or otherwise) of its constituent states. Similarly, at the scale of specific islands, revolutionary movements challenged the meaning of independence and the partial nature of sovereignty under capitalism. The New Jewel movement in Grenada led by Maurice Bishop, which took power from 1979 to 1983, for instance, and the 26th of July movement in Cuba launched by Fidel Castro in 1953 inaugurated the period of the Cuban Revolution still in effect today.

Of course, the temporal difference between the longevity of those two revolutionary moments in Grenada and Cuba, respectively, did not occur naturally. The third characteristic of the animated video is the way it highlights the importance of foreign military interventions in the formation and denial of postcolonial sovereignty. The US invasion and occupation of Grenada in 1983 deposed the New Jewel movement government. On the other hand, US incursions on Cuban sovereignty, such as the Bay of Pigs invasion, infamously failed, although it remains to be seen how the recent restoration of diplomatic relations between the United States and Cuba will effect the revolutionary government. Tracking military invasions, moreover, points to the emergence of the United States as a colonial power after the Spanish-American War in 1898, powerfully represented by both the acquisition of Puerto Rico and the US Virgin Islands, as well as the stubborn green dot staining the blue of independent Cuba: Guantanamo Bay.

Fourth and finally, the animation ends up moving toward a nonchronological plotting of time in which the dates of independence and post-colonial status spatially represent the inextricably woven nature of sovereignty as a regulatory ideal, a discursive formation, a historical imaginary, and a material experience. In 2003, for instance, St.~Barthélemy and St.~Martin emerge out of Guadeloupe and the powerful historical shadow of 1946. By keeping them spatially adjacent, the video holds in tension the continued importance of 1946 to the sense of citizenship in the French Caribbean with contemporary struggles for other modes of recognition and collectivity. Perhaps most clearly, the animation returns us to Haiti, the inaugurating country on the chart of the postcolonial Caribbean. Now, next to the constitutive date of 1804, a United Nations flag rests uneasily to mark the international peacekeeping mission that began in 2004 after the ousting of Jean-Bertrand Aristide. Haitian sovereignty and self-determination, ideals borne in the first black republic out of the profound power of the Haitian Revolution, thus remain subject to humanitarian occupation. After the tragic earthquake in 2010, for instance, the United States claimed control of Haitian airspace and airports, causing diplomatic fights with other countries involved in the United Nations mission, \quotation{exposing a brewing power struggle amid the global relief effort.}\footnote{Rory Carroll and Daniel Nasaw, \quotation{US Accused of Annexing Airport as Squabbling Hinders Aid Effort in Haiti,} {\em Guardian} (UK), January 2010, \useURL[url8][http://www.theguardian.com/world/2010/jan/17/us-accused-aid-effort-haiti]\from[url8].} In the haunting overlay of 1804 and 2004, it becomes clear that the political plotting of time slips away from the control of linear narrative and historical mastery.

\subsection[reference={toward-a-prophetic-cartography},
bookmark={Toward a Prophetic Cartography},
title={Toward a Prophetic Cartography}]

Despite our best efforts to create complex representations of the political landscape in the Caribbean, we repeatedly ran up against the problem of the reifying nature of Western cartography itself and the ways it reproduces normative understandings of political sovereignty. In both the static and the animated versions of our map, independence is still represented as a solid and equivalent state. This flattens out different histories of political independence, such as the experiences and implications of independence through violent struggle versus political agreement, and routine impingements on sovereignty by international actors, such as the International Monetary Fund. This limitation indexes the fundamental challenge to a cartographic endeavor in which the complexity and connectivity of the region continually outstrips the frames we use to make it intelligible. The political work of representation must account for inevitable omissions: the formation of the Panama Canal, for instance, or the fall of the Soviet Union go unmarked in the video, although each event could help tell the story of the postcolonial Caribbean. Indeed, the very act of naming something an event and committing it to the map entails a political decision. For example, we debated whether to include the emerging rapprochement between the United States and Cuba promised by the reopening of the US Embassy in Havana: What might a new era in US-Cuban relations mean for a mapping of the present and future horizons of the region? The recursive quality of these challenges suggest that the visualizations above, no matter how capacious they become, do not necessarily help us reimagine what sovereignty represents for Caribbean populations and how it is experienced differently across space and time.

After producing these visual tools, we are thus still left with a lingering question: How can we represent sovereignty as a historically contingent {\em claim} and culturally specific {\em value} rather than an ontological reality? That is, how do we not just visualize sovereignty but use visualizations to retheorize the meaning of sovereignty itself?

We believe that the first step in this process would involve unbundling the relationships that have been variously signified as the constituent components of political sovereignty. As scholars such as Lauren Benton have shown, sovereignty needs to be theorized as a \quotation{divisible quality,} that encompasses a broad set of relationships applied in various configurations across place and time.\footnote{Benton, {\em Search for Sovereignty}, specifically chap. 5, \quotation{Landlocked: Colonial Enclaves and the Problem of Quasi-Sovereignty,} 222--78.} Secondly, we must examine how sovereignty operates as a native category. Previous research has revealed that the concrete elements local populations associate with sovereignty, autonomy, and self-determination do not always align with how these values are defined by political theorists and policy experts. In her previous research about Guadeloupe, Yarimar found that local activists viewed the search for sovereignty not as the attainment of nation-state status but as the ability to exert control over elements that impacted daily life, such as the consumption of locally produced food, the availability of local employment opportunities (as opposed to having to migrate for work), increases in minimum wage and \quotation{purchasing power,} the use of vernacular language, attention to silenced histories of resistance, and the promotion of local arts and media.\footnote{See Bonilla, {\em Non-sovereign Futures}.} Third, we must develop alternative spatiotemporal models based on these reexaminations of sovereignty and specific, local negotiations of nationalism and globalization. The nonsovereign archipelago suggested earlier is one such possibility. In his work on political ecology in New Orleans and the Greater Caribbean, Max considers other possibilities from Afro-Caribbean thought that traverse the boundaries and scalar fixes of sovereign imaginaries, drawing together distinct bodies and landscapes without annihilating their difference.\footnote{See Max Hantel, \quotation{Errant Notes on a Caribbean Rhizome,} {\em Rhizomes: Cultural Studies in Emerging Knowledge} 24 (2012), \useURL[url9][http://www.rhizomes.net/issue24/hantel.html]\from[url9]; \quotation{Intergenerational Geographies of Race and Gender: Tracing the Confluence of Afro-Caribbean and Feminist Thought beyond the Word of Man} (PhD diss., Rutgers University, New Brunswick, NJ, 2015).}

Caribbean populations have long produced ideas of sovereignty, autonomy, self-determination, freedom, and liberty that are not properly captured by either the textual or the visual language of the international system of states. These nonsovereign projects push against the limits of what is conceivable under our current visual and conceptual repertoire. We believe that new kinds of speculative visualizations---employing both scholarly and artistic methods and practices---might help propel these projects by giving shape and form to an otherwise inchoate search for political alternatives. In other words, by creating new representations of Caribbean sovereignty---what, borrowing from Glissant, we describe as prophetic cartographies---new political visions for the region might become more readily imaginable. At a time when many across the region and beyond feel a profound disenchantment with the project of nation-state sovereignty, these new visions might help us become more attuned to alternative political forms that are not currently visible or fully imaginable.

\subsection[reference={acknowledgements},
bookmark={Acknowledgements},
title={Acknowledgements}]

Support for this project was provided by the National Science Foundation, the Rutgers School of Arts and Sciences, Rutgers Department of Latino and Caribbean Studies, Rutgers Advanced Institute for Critical Caribbean Studies, the Center for the Study of Social Difference at Columbia University, the Gender Research Institute at Dartmouth College, and the History Design Studio at Harvard University. For their assistance at various stages of this project, we would like to thank Landon Yarrington, Jonna Yarrington, Enmanuel Martinez, Catalina Martínez Sarmiento, Louis Philippe Römer, and Mario Mercado Diaz. For their advice and feedback on various drafts and versions, we would like to thank Vincent Brown, Glenda Carpio, the members of the Multimedia History and Literature seminar at the Charles Warren Center for Studies in American History at Harvard University, the participants in the Rutgers Advanced Institute for Critical Caribbean Studies seminar at Rutgers University, and the comrades of the Digital Black Atlantic Project working group at Barnard College and Columbia University.

\thinrule

\page
\subsection{Yarimar Bonilla}

Yarimar Bonilla is Associate Professor of anthropology and Latino & Caribbean Studies at Rutgers University, where she is on the executive board for the Advanced Institute for Critical Caribbean Studies. She teaches and writes about social movements, colonial legacies, and the politics of history in the Atlantic World. She is the author of {\em Non-Sovereign Futures: French Caribbean Politics in the Wake of Disenchantment} (2015) and is currently working on a book project about the statehood movement in Puerto Rico.

\subsection{Max Hantel}

Max Hantel is a postdoctoral fellow in the Gender Research Institute at Dartmouth College. He is currently completing his book project, Haunted Confluence: Intergenerational Geographies of Race and Sex in Afro-Caribbean Feminism. His work on Caribbean philosophy and political ecology has appeared in {\em Small Axe}, {\em The Journal of French and Francophone Philosophy}, and {\em Rhizomes}.

\stopchapter
\stoptext