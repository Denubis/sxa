\setvariables[article][shortauthor={Eller}, date={May 2016}, issue={1}, DOI={10.7916/D8833S3J}]

\setupinteraction[title={A Review of \quote{A Colony in Crisis}},author={Anne Eller}, date={May 2016}, subtitle={Review}]
\environment env_journal


\starttext


\startchapter[title={A Review of \quote{A Colony in Crisis}}
, marking={Review}
, bookmark={A Review of \quote{A Colony in Crisis}}]


\startlines
{\bf
Anne Eller
}
\stoplines


\useURL[url1][https://colonyincrisis.lib.umd.edu/][][{\em A Colony in Crisis}]\from[url1] is a digital exhibition that addresses a colonial flashpoint, the moment when failed grain harvests in France fueled escalating political tensions between white planters in Saint-Domingue and political elites in the metropole.\footnote{{\em Colony in Crisis: The Saint-Domingue Grain Shortage of 1789}, \useURL[url2][http://colonyincrisis.lib.umd.edu]\from[url2].} The site's authors, Abby Broughton, Kelsey Corlett-Rivera, and Nathan Dize, in consultation with an interdisciplinary team of experts, aim to present the relationship between France and Saint-Domingue in a more dialogic fashion than is traditionally depicted in French accounts. They highlight the fundamental conflicts between the financial interests of planters and the trade monopolies of the metropole, revealing positions that amount to discourses of permanent complaint.\footnote{See \quotation{The Project,} {\em Colony in Crisis}, \useURL[url3][http://colonyincrisis.lib.umd.edu/about]\from[url3].}

The site's authors refer to a \quotation{potential famine} and call the grain \quotation{much needed,} echoing the language of the 1780s planter supplicants who spoke of a \quotation{{\em fear} of a food shortage.}\footnote{\quotation{M. de Cocherel's reflections, Deputy of Saint-Domingue, on the report from the Comité des Six,} 10 November 1789, \useURL[url4][http://colonyincrisis.lib.umd.edu/1789/11/10/m-de-cocherels-reflections-deputy-of-saint-domingue-on-the-report-from-the-comite-des-six]\from[url4], paras. 3 and 1 of the introduction, para. 4 of the translation (emphasis mine). The Comte de Reynaud suggests an \quotation{eve . . . of famine.} \quotation{Motion from M. le Comte de Reynaud, Deputy of Saint-Domingue at the August 31 Session,} 31 August 189, \useURL[url5][http://colonyincrisis.lib.umd.edu/1789/08/31/motion-from-m-le-comte-de-reynaud-deputy-of-saint-domingue-at-the-august-31-session]\from[url5], para. 6 of the translation. The table of contents for issue 1.0 of the translations also echoes this theme, in its phrasing \quotation{the grain shortages in Saint-Domingue . . . threatened the planters with famine and malnutrition.} Annotation to \quotation{Ordinance concerning the introduction of foreign grain in the warehouse ports of the French section of the island of Saint-Domingue,} 27 May 1789, \useURL[url6][http://colonyincrisis.lib.umd.edu/category/translations/issue-1-0]\from[url6].} However, it is important to note that despite a production shortage in France, there was no grain crisis in the colony of Saint-Domingue. Any potential shortfall threatened exclusively the consumption of a small planter minority, and, in fact, the colony's intendant denied any threat to planter diets at all.\footnote{While his refutation is not in the English translation on {\em Colony in Crisis} (http://colonyincrisis.lib.umd.edu/1790/02/12/charges-of-fraud-the-obstinate-rebuttal-of-a-banished-intendant-concerning-the-grain-shortage-and-criminal-negligence), it is offered in the full text on the Internet Archive; see {\em Mémoire et observations du Sieur Barbé de Marbois, intendant des Isles-sous-le-vent en 1786, 1787, 1788, et 1789}, \useURL[url7][http://archive.org/stream/memoireetobserva00barb\#page/34/mode/2up]\from[url7], 35.} Of all the colony-side missives gathered in {\em Colony in Crisis}, only two address the question of food supplies for the enslaved, although a helpful linked article suggests that other petitions often did so, albeit in a nakedly utilitarian manner.\footnote{Joseph Horan, \quotation{The Colonial Famine Plot: Slavery, Free Trade, and Empire in the French Atlantic, 1763--1791,} {\em IRSH} 55 (2010): 103--21. Gouy d'Arsy and Reynaud invoked the diet of enslaved people in their petitions. \quotation{Summary given by M. le Marquis de Gouy d'Arsy,} 9 September 1789, \useURL[url8][http://colonyincrisis.lib.umd.edu/1789/09/09/summary-given-by-m-le-marquis-de-gouy-darsy]\from[url8]; \quotation{Motion from M. le Comte de Reynaud.}} These colony-side writers admit that the enslaved were not receiving these grain foodstuffs and suggest, in the course of their broader arguments, a meager allotment well out of proportion with their population.\footnote{\quotation{Summary given by M. le Marquis de Gouy d'Arsy.}} Pamphleteers based in France also take only a passing stab at addressing slave provisions. A single pamphlet offers (imprecise) estimates of the number of enslaved people working on coffee plantations, arguing (speciously) that food production there adequately supplied individuals enslaved on sugar plantations (a number they in turn minimized). The pamphleteers' vague numeracy serves only to mislead, and most of the elite writers whose letters are reproduced in {\em Colony in Crisis} avoid specifics altogether, not out of yet another instance of willful elision but because enslaved men, women, and children were not consuming these grains.\footnote{\quotation{Response from the Deputies of Production and Commerce of France: To the motions of Mm. de Cocherel & de Reynaud, Deputies from the Isle of Saint-Domingue to the National Assembly,} 13 September 1789, \useURL[url9][http://colonyincrisis.lib.umd.edu/1789/09/13/response-from-the-deputies-of-production-and-commerce-of-france]\from[url9].}

In sum, the crisis was not about Saint-Domingue's foodstuffs in any substantive way, even if the corruption, hoarding, and monopolies of high-level merchants and the intendant might have made local planters feel frustrated and even temporarily food-insecure for their own households. Instead, planters used France's grain crisis as a flashpoint to try to break the metropole's economic monopoly and engage in regional trade. The planters wanted not only to seek nearby supplies but also, crucially, to benefit from an expanded market for Saint-Domingue's cash crops. Free ports, opened through the guise of supply trade, would have unleashed Dominguan coffee and sugar to eager nearby markets, particularly the United States, and free ports would furthermore easily have concealed trade in many other products. American flour offered a Trojan vessel for reduced oversight. Deregulation would have had widespread effects, including diminished importance of France, as the pamphleteers acknowledged.\footnote{\quotation{M. de Cocherel's Reflections.}} The two elite groups speak past each other, highlighting intransigent differences over high-level policy. Authors in France invoke interimperial rivalry with Britain, even as colonial planters are far more concerned with engagement to their immediate north, with the United States.\footnote{\quotation{On the state of slaves regarding the prosperity of French colonies and their metropole; Address to the nation's representatives,} 17 March 1789, \useURL[url10][http://colonyincrisis.lib.umd.edu/1789/03/17/on-the-state-of-slaves]\from[url10].} Both elite groups were frustrated with each other, and the point-counterpoint structure of {\em Colony in Crisis} helps users understand the intransigence of each position, even a few months after revolutionary fighting began on French soil.\footnote{\quotation{Supplement to the counter argument from the Deputies of French Manufacturers and Commerce to the Deputies of Saint-Domingue, concerning the supply of provisions to the colony,} 11 November 1789, \useURL[url11][http://colonyincrisis.lib.umd.edu/1789/11/11/supplement-to-the-counter-argument-from-the-deputies-of-french-manufacturers-and-commerce-to-the-deputies-of-saint-domingue]\from[url11].}

Several documents selected for issue 2.0 of {\em Colony in Crisis} seek \quotation{to provide a larger, transnational context for . . . 1789 Saint-Domingue,}\footnote{\quotation{Welcome to A Colony in Crisis,} \useURL[url12][http://colonyincrisis.lib.umd.edu]\from[url12], para. 1, update.} and they succeed in highlighting promising new avenues of expansion for the site's collection. Two pamphlets in particular offer a regional view of French administrative questions in the Caribbean, pointing to new directions of inquiry: the functioning and life of regional ports, regional foodways, and intra-Caribbean traffic, more generally.\footnote{\quotation{Judgment from the State Council of the King, concerning foreign commerce in the French Isles of America,} 30 August 1784, \useURL[url13][http://colonyincrisis.lib.umd.edu/1784/08/30/judgment-from-the-state-council-of-the-king-concerning-foreign-commerce-in-the-french-isles-of-america-from-august-30-1784]\from[url13]; \quotation{Judgment from the State Council of the King,} 10 September 1789, \useURL[url14][http://colonyincrisis.lib.umd.edu/1786/09/10/judgment-from-the-state-council-of-the-king-september-10-1786]\from[url14].} The scholars aver that their initial goal for {\em Colony in Crisis} was to \quotation{speak to the conditions of the slaves, wealthy and poor planters, free people of color, and the reactions to the system of slavery in mainland France} and that this goal was unmanageably broad.\footnote{\quotation{Issue 1.0: Introduction,} {\em Colony in Crisis}, \useURL[url15][http://colonyincrisis.lib.umd.edu/2014/09/12/issue-1-0-introduction]\from[url15], para. 1.} In these transcolonial documents, however, the authors provide a scaffold for just such a growing dialogue.

University of Maryland pamphlets figure prominently in the first two issues of {\em Colony in Crisis}, and these documents have been curated, translated, and contextualized for a wide audience of academics, students, and the general public.\footnote{The university holds twelve thousand French-language pamphlets, of which nearly five hundred relate directly to France's Atlantic colonies. See \quotation{Revealing La Révolution Team,} colonyincrisis.lib.umd.edu/the-team/revealing-la-revolution-team. See also \quotation{Q&A with {\em A Colony in Crisis},} {\em Haitian History} (blog), \useURL[url16][http://haitianhistory.tumblr.com/post/136472549793/qa-with-a-colony-in-crisis]\from[url16].} An annotated table of contents (found under the tab for translations) offers helpful orientation, and each entry benefits from a longer introduction on its own page, where the French document is linked in its entirety from \useURL[url17][https://archive.org/index.php][][Internet Archive]\from[url17], and the authors offer a five-hundred- to seven-hundred-word English translation. The site is eminently accessible for and welcoming to new users and is also well designed, with built-in link redundancies for excellent navigation between documents, explanations, and the rest of the project.

{\em Colony in Crisis} calls attention to a wealth of printed resources currently housed in separate physical institutions, and it succeeds brilliantly in making them accessible to scholars of varying interests and levels of engagement. One can easily see how another pamphlet series might, for example, include \useURL[url18][https://archive.org/details/ideadelvalordel00valvgoog][][texts]\from[url18] from Saint-Domingue's immediate cattle-trading neighbor, Santo Domingo.\footnote{See, for example, Antonio Sánchez Valverde, {\em Idea del valor de la isla española de Santo Domingo} (1862), \useURL[url19][http://archive.org/details/ideadelvalordel00valvgoog]\from[url19].} Another challenge might be how resources from this \quotation{rich print culture} could be gathered alongside other more fragmentary text sources, oral histories, and sound studies.\footnote{\quotation{The Project,} para. 2 under 'The Goals." The \useURL[url20][http://sites.fhi.duke.edu/haitilab/][][Haiti Lab]\from[url20] at the John Hope Franklin Humanities Institute at Duke University gathers an important group of these projects; see \useURL[url21][https://web.archive.org/web/20160202032516/http://www.fhi.duke.edu/labs/haiti-lab/online-projects][][http://www.fhi.duke.edu/labs/haiti-lab/online-projects]\from[url21].} Or the impressive board of advisors for {\em Colony in Crisis} could bring context to a wide variety of sources and could also bring to new audiences the ongoing vibrant debates about the \useURL[url22][http://socialtext.dukejournals.org/content/33/4_125.toc][][archive]\from[url22] itself.\footnote{See, for example, \quotation{The Question of Recovery: Slavery, Freedom, and the Archive,} special issue, {\em Social Text}, no. 125 (December 2015).} These discussions would inevitably more thoroughly engage the lived experience of the enslaved men, women, and children who made up the majority of Saint-Domingue as the colony confronted various crises in the 1780s. That said, even at this present stage of the project's development, readers will certainly have a sense of the magnitude and value of the site, and I look forward to following the project as it grows.

\thinrule

\page
\subsection{Anne Eller}

Anne Eller is Assistant Professor of History at Yale University. Her first book, {\em We Dream Together: Dominican Independence, Haiti, and the Fight for Caribbean Freedom}, will appear in late 2016 with Duke University Press.

\stopchapter
\stoptext