\setvariables[article][shortauthor={Scott}, date={May 2016}, issue={1}, DOI={10.7916/D85B02J1}]

\setupinteraction[title={},author={David Scott}, date={May 2016}, subtitle={}]
\environment env_journal


\starttext


\startchapter[title={}
, marking={}
, bookmark={}]


\startlines
{\bf
David Scott
}
\stoplines


The ethos of the Small Axe Project, I've said, consists of a reflexive attitude of listening to, and engaging with, the challenges and provocations that grow out of the changing problem-spaces within which we contingently find ourselves as Caribbean intellectuals, and as scholars of the Caribbean. The ethos of the Small Axe Project, therefore, is essentially an ethos of learning, and of learning how to better learn---even to better learn what we think we already know. After all, the Caribbean won't stand still; and so the questions that constitute the Caribbean as an interpretive domain don't remain the same. Consequently, we in the Small Axe Project are constantly being called upon (and calling upon each other) to think, and to think again, about the forms and platforms that should best shape and drive our responsive hermeneutic endeavors.

In recent years the humanities and social sciences have been provoked to consider the implications of the \quotation{digital turn} for scholarly production and reception. The rise of new digital technologies, new modes of accessing and organizing intellectual resources, new ways of connecting and sharing various dimensions of understanding (and therefore of integrating our cognitive and affective capacities of appreciation and critical analysis), new questions about what should count as excellence in scholarship, has opened up new possibilities and new challenges for intellectual and artistic work. The Caribbean, we believe, in its persistent historicity and conscripted modernity, is a geopolitical zone that especially lends itself to the kinds of experiment and exploration that the digital turn obliges us to confront. {\em sx archipelagos} is our response to this momentous conjuncture.

{\bf David Scott}\letterbackslash{} Director, The Small Axe Project

\page
\subsection{David Scott}

David Scott is Professor of Anthropology at Columbia University. He is the author of {\em Formations of Ritual: Colonial and Anthropological Discourses on the Sinhala Yaktovil} (Minnesota 1994), {\em Refashioning Futures: Criticism After Postcoloniality} (Princeton UP 1999), {\em Conscripts of Modernity: The Tragedy of Colonial Enlightenment} (Duke UP 2004), and {\em Omens of Adversity: Tragedy, Time, Memory, and Justice} (Duke UP 2014), and co-editor of {\em Powers of the Secular Modern: Talal and His Interlocutors} (Stanford UP). He is the founder and director of the Small Axe Project.

\stopchapter
\stoptext