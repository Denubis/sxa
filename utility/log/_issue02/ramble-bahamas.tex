\setvariables[article][shortauthor={Dawson, Thompson}, date={July 2017}, issue={1}, DOI={doi:10.7916/D84T6WRF}]

\setupinteraction[title={Ramble Bahamas},author={Jessica Dawson, Tracey Thompson}, date={July 2017}, subtitle={Ramble Bahamas}]
\environment env_journal


\starttext


\startchapter[title={Ramble Bahamas}
, marking={Ramble Bahamas}
, bookmark={Ramble Bahamas}]


\startlines
{\bf
Jessica Dawson
Tracey Thompson
}
\stoplines


A Project by \quotation{From Dat Time}: The Oral and Public History Institute of the University of The Bahamas

\subsection[title={{\em sx archipelagos} Introduction},reference={sx-archipelagos-introduction}]

\useURL[url1][http://ramblebahamas.org][][Ramble Bahamas]\from[url1] is the publication engine for \quotation{From Dat Time}: The Oral and Public History Institute at the University of The Bahamas. Developed~by project director~Tracey Thompson and research and technology fellow Jessica Dawson, the site aims not only to promote the work of academic and independent researchers but also to affirm the contributions of community elders and to provide easily accessible resources for primary- and secondary-school teachers and, ultimately, for~the wider Bahamian community. Hewing close to the concept of rambling and all it connotes of meandering, discovery, and unhurriedness, the site invites multiple paths for exploration---through individual stories, curated wanderings, and interactions with a map of the island. The platform's primary thematic focus covers four areas:~Bahamian participation in World War II; national politics in the postwar era; the evolution of the nursing profession; and the development of the sportfishing industry. The site combines scholarly research with audio clips of oral history, archival images and documents, and contemporary photographs. Through this collage of source materials and insistence on place as historical frame, {\em Ramble Bahamas} promises users a gently guided but largely self-determined encounter with fascinating elements of Bahamian national history.

\subsection[title={Creators' Introduction},reference={creators-introduction}]

The digital humanities offer a unique vehicle for bridging the past and present. Interactive media formats encourage user engagement while maintaining the integrity of historical methodologies. Digital platforms enable audiences located far and wide to access information that is not easily available in print format. All these advantages carry special value for students, educators, and scholars who are investigating twentieth-century Bahamian history. Such audiences are met with a grave shortage of resources, whether in physical format or web-based format, that illuminate the Bahamian experience. The challenge of accessing resources confronts, in particular, audiences located within the Bahamian archipelago yet outside the central island of New Providence as well audiences located abroad. {\em Ramble Bahamas} seeks to remedy this deficit by providing a curated collection of easily accessible place-based exhibits in an innovative medium. Each geo-tagged exhibit includes a cohesive narrative that centers on the story of an historically significant site or object. Additional context is built through the inclusion of historical images, newspapers, other documents, and contemporary photographs. Select audio clips taken from oral history interviews with authoritative narrators are also featured within each exhibit to deepen the sense of place, further stimulate the sensory experience of the visitor, and extend each visitor's knowledge about events associated with the location and about circumstances prevailing during the era. Techniques for constructing the product include carrying out oral history interviews, conducting documentary and archival research, and performing audio-visual digitization and editing, as well as deploying and customizing the Omeka content management system powered by Curatescape. At the time of its launch in January 2017, Ramble Bahamas contained ten exhibits on subjects such as sociopolitical history, the maritime industry, and the history of nursing in The Bahamas. Site statistics attest to the local and international impact of the website. Such statistical data can be used to direct the site's growth, goals, and other characteristics of its future development.

\thinrule

\subsection[title={{\em sx archipelagos} review},reference={sx-archipelagos-review}]

\startblockquote
\quotation{to combine the best of oral history, Caribbean cultural heritage, and walking.}
\stopblockquote

First,\footnote{Editors' note: Because several updates have been made to the {\em Ramble Bahamas} website, prompted in part by the reading of this review, some content covered here is no longer available on the site. See Jessica Dawson and Tracey Thompson's \quotation{Review Response from the Creators of {\em Ramble Bahamas}} below for a discussion of changes made, not made, and anticipated.} we want to congratulate the creators on making such promising headway with such an ambitious agenda: to combine the best of oral history, Caribbean cultural heritage, and walking. It is there, in the last instance, where the project seems to want to make its major contribution: to ramble through the history of The Bahamas. It is also there where perhaps this review can help the most. {\em Ramble}, a word that can mean both a walk taken for pleasure or confusing speech, is fitting to describe the dilemmas and room for improvement of the project as it stands.

We trust that the intention of the creators is to provide a relatively random encounter with \quotation{Bahamian participation in World War II, national politics in the postwar era, the evolution of the nursing profession, and the development of the sportfishing industry,}\footnote{\quotation{About} (under \quotation{FDT}), {\em Ramble Bahamas}, \useURL[url2][http://web.archive.org/web/20170904035235/http://www.ramblebahamas.org/about][][www.ramblebahamas.org/about]\from[url2].} yet visitors are guided by the voices of the researchers, their interlocutors in the recordings, the images, and the site's navigation scheme (including its maps). In a very real sense, the current navigation scheme and web design can create a sense of disorientation, which falls outside the pleasures of rambling. What we hope to do in this review is suggest a series of minor tweaks that in the aggregate may allow the project to be both rambling and guided without necessarily being disorienting.

Let us clarify what we mean by \quotation{disorienting.} When we arrive at the site, the first thing that catches our attention is the small paragraph that invites us to explore. We have several options at this point: we can scroll down and click on the \quotation{Read more {\em About Us}} button or dive straight into the map on the right. The map is too appealing to ignore, so we click on it. It expands to replace the welcome message. We click again on the yellow circle over The Bahamas. A pin and a green circle appear as the viewport zooms in closer to the ground. We click again on the green circle. After a similar effect, we can now see the streets of Nassau. We finally click on a pin: we get an image and \quotation{The Contract.} Hovering the mouse over it, out of curiosity, we notice there is a link, so we click on that too and arrive at an article page. After reading a few sentences of the article, we realize we are already deep in history. How did we get here? Where are we? Scared we did something wrong, we return to the homepage.

Once we begin to get a sense of how the site works, everything starts making sense, but it takes a while to get there, and even then it takes a bit more time to see the site's conceptual coherence. Perhaps the few minor changes suggested below could help.

\subsubsection[title={The Homepage},reference={the-homepage}]

Much can be solved by doing most of the revision work on the homepage. Any good ramble needs a good preamble. Instead of the existing welcome message, which is too open an invitation, the user would be well served by some brief instructions to navigating the rest of the homepage, reconstructed for clarity. While the map is fun, it is currently positioned in too tempting a place to avoid disorientation. We recommend placing it at the bottom, with the preamble above helping site visitors understand what it is doing and perhaps even allowing the map to be the unguided place.

With a strong enough narrative preamble, there could be three sections below it: Trails, Recent Stories, and Maps.

Trails: \quotation{Ramble on} as an imperative is very catchy, but it needs clarification. {\em Ramble} is being used in too many senses here for it to imply simply \quotation{visit this trail.} We can think of two solutions to this problem: substitute {\em Trails} throughout with {\em Rambles} or simply add a \quotation{Visit This Trail} link at the button. We prefer the former, since it already starts giving a more precise meaning to capital-letter {\em Ramble} without interfering in the general sense of rambling, which the site authors want to hold on to, while also opening the door for them to provide some critical reflections on the act of rambling, through voices and streets, performed as a transmedia digital project---something we would ask for as reviewers, but more of that in our review of the \quotation{About} page.

Recent Stories: If the preamble above briefly explains what a story is, this can be left as is.

Maps: There is much to say about the three kinds of maps----the general one here; the one with individual pins, at the tops of the articles; and the one for Trails. The general one can stay as is, if placed at the bottom and preambled by a warning that the user must find landmarks without guidance in order to arrive at a more in-depth look at a given landmark.

\subsubsection[title={The \quotation{About} Page},reference={the-about-page}]

There is a great, but missed, opportunity here, since this page is expected to be straightforward prose. The site already does a fantastic job of packing much information into a few paragraphs. Our two suggestions for revisions are simple.

First, add a section reflecting on the design/structure of the site as founded in the idea of rambling and on the importance of rambling/walking in the Caribbean. We are thinking, for example, of Garnette Cadogan's \quotation{Walking while Black} (\useURL[url3][http://web.archive.org/web/20170904035212/http://lithub.com/walking-while-black/][][{\em lithub.com/walking-while-black}]\from[url3]), which the site authors would do well to consult. There are also other historians who have tried to reflect critically on walking and history. This would help mitigate the notion that the site is simply \quotation{where {[}the{]} research team publishes historical and cultural knowledge that {[}it{]} assemble{[}s{]} about Bahamians and about The Bahamas.}\footnote{\quotation{About,} {\em Ramble Bahamas} (as accessed March 2017).} To \quotation{publish} in such a formalized structure as a website/exhibit---different from a journal or a book---needs justification, especially when the form informs the ambitions of the project to this extent.

Second, break down the credits at the bottom into a list in which everyone's contributions and roles are noted, rather than simply their titles within the academic hierarchy.

\subsubsection[title={Browse Items},reference={browse-items}]

This page is almost perfect. One small confusion between {\em items} and {\em stories} can be clarified. As far as we can see, an item is a story. However, this is very different from the conception of {\em item} native to Omeka and familiar to Omeka users.\footnote{Omeka is the open-source web publishing platform used by {\em Ramble Bahamas}; see \useURL[url4][http://web.archive.org/web/20170904035248/http://omeka.org/about/][][omeka.org/about]\from[url4].} We would simply rebrand the headings to \quotation{Browse Stories} or \quotation{Stories.} This would go a long way to making this page clearer. To seal the deal, we would have a brief blurb above or below the map explaining what a {\em story} is.

{\em Tags:} Site authors should make the tags consistent and use them throughout (perhaps around the historical themes).

\subsubsection[title={Trails},reference={trails}]

As we suggested above, this is the place to do a lot of conceptual work around {\em Rambling}. The Trails, and this is crucial, are simply subcollections, in the Omeka sense of the word---selections from all the stories. In the future, site authors might encounter problems with Omeka's nonoverlapping collections, but maybe not. Currently, {\em Ramble Bahamas} makes very smart use of the collections feature. The site authors should be explicit, though, about the fact that these Rambles are coming from the general collection. More important, they should note that this is a specific, guided form of experiencing the project.

{\em The map:} As it stands, the map seems incomplete at some zoom levels (one of the pins overlaps on the other, depending on zoom), and it does not present itself as a Trail immediately. One simple tweak to show numbers instead of pins might help here. Although the numbers are visible on clicking, that is not immediately helpful. It would also help to clarify whether the numbers represent suggested paths on a walking tour or simply serve as indexes to the ordering below.

{\em The blurb:} \quotation{Making use of multimedia tools so as to interest audiences of all ages} is not enough. Those tools must be used in specific ways for different audiences. This is where prose helps enormously. The introduction to this section, and to all future Rambles/Trails, could help synthesize the selections and provide the necessary preamble for audiences of all ages to enjoy the individual stories more fruitfully.

\subsubsection[title={Individual Stories},reference={individual-stories}]

These are great, and they provide the substance for the site. A few small tweaks to the design can go a long way in making these pages even richer than they already are:

\startitemize[a,packed][stopper=)]
\item
  On the map pin, make the zoom be at the street level;
\item
  separate image from audio visually (different background color, dividing line, etc.), and place audio above images, since audio is richer; and
\item
  use a more legible font for the audio caption to make sure the user understands what they are about to listen to.
\stopitemize

In general, it is always best to err on the side of more narrative framework for multimedia content. We understand the delicate balance between letting documents speak for themselves and taking on the role of curator and guide, but we think a bit more framing can vastly improve the user experience here without detracting from the primary sources.

\subsection[title={Review Response from the Creators of {\em Ramble Bahamas}},reference={review-response-from-the-creators-of-ramble-bahamas}]

We would like to thank the reviewers for the constructive and insightful feedback that they have provided in their comprehensive review of \useURL[url5][http://www.ramblebahamas.org/][][{\em Ramble Bahamas}]\from[url5].\footnote{See \quotation{A Review of {\em Ramble Bahamas},} this issue of {\em sx archipelagos}. See also \useURL[url6][http://www.ramblebahamas.org][][www.ramblebahamas.org]\from[url6].} We are thrilled that the reviewers see potential in our ambitious undertaking to provide a platform combining oral history, active learning, and an exploration of Bahamian cultural heritage. We found the reviewers' detailed narration of their first encounter with the home page to be a perceptive critique that will help guide us to build a more user-friendly experience with {\em Ramble Bahamas}.

The reviewers provided a series of recommendations to enhance the ease of site navigation and relieve any disorientation that may be felt by a new user. The most substantial revisions concern the addition of explanatory prose and the utilization of consistent labels. We appreciate and have heeded the recommendation to revise the home page welcome message to include explanatory instructions for site navigation. In this space, we now provide definitions for our terms {\em stories} and {\em rambles}. We have also added explanatory prose to \quotation{The Majority Rule Heritage Trail} preamble to address how specific audiences, such as students and tourists, can engage with each of the stories as well as to provide a recommendation on how to interact with the walking trail. The suggestion to rebrand \quotation{Trails} as \quotation{Rambles} to both simplify navigation and nominalize the act of rambling was an excellent one. For the sake of clarity and ease, the reviewers keenly suggested the rebranding of \quotation{Browse Items} to \quotation{Stories.} We have made these recommended edits to the navigation bar.

The review offered a number of prudent suggestions related to the embedded maps. The placement of the map on the home page is a subject we have debated internally. The reviewers noted the placement as \quotation{too tempting a place to avoid disorientation.} While our site targets a broad audience, our primary users are Bahamian secondary school students. The placement of the map has resonated in a positive way with this younger audience, and for this reason we have chosen to leave it parallel to the welcome message. We hope that the balance between detailed welcoming instructions and an interactive map will prove to be user-friendly for all audiences. For individual stories, the suggestion to automatically default the map zoom to street level was a sensible one and has been implemented sitewide. For Trails (now Rambles), the reviewers offered some ideas to provide a more visually complete trail, including the utilization of numbered marker icons rather than the current marker icons. While we considered this suggestion, it is not an option available with the current geolocation plug-in utilized by the site. As the reviewers noted, the numbers are available when clicking on each marker icon. To make this clear, we have included notes in \quotation{The Majority Rule Heritage Trail} text explaining how the numbers can be activated. The reviewers also noted the overlap of two marker icons visible at more distant zoom levels. The proximity of these two icons is unfortunate and unavoidable, since two stories took place at the same locale. In future Rambles, we will be mindful of marker icon placement when selecting and mapping our locations to avoid this kind of overlap.

The review notes a missed opportunity on our \quotation{About} page and offers several suggestions to better explain the process behind the work we undertake as public historians and the importance of \quotation{rambling} to the site creators. Moreover, the reviewers suggested how this space could be utilized as an opportunity to reflect on how {\em rambling}, in Bahamian parlance, means both to speak at length on a topic and to wander in an unhurried and enjoyable fashion. Interestingly, when we present {\em Ramble Bahamas} to outside stakeholders and audiences, we are commonly asked questions regarding the meaning of the word {\em ramble} and why the site's name was chosen. Therefore, the reviewers had underscored a real and pressing opportunity to better explain a concept possibly foreign to non-Bahamians as well as to remind younger Bahamians of a pastime cherished by their elders. Per recommendation, we have taken Garnette Cadogan's powerful \quotation{\useURL[url7][http://web.archive.org/web/20170904035212/http://lithub.com/walking-while-black/][][Walking while Black]\from[url7]} essay to heart.\footnote{Garnette Cadogan, \quotation{Walking while Black,} 8 July 2016, \useURL[url8][http://web.archive.org/web/20170904035212/http://lithub.com/walking-while-black/][][lithub.com/walking-while-black]\from[url8].} This work and the reviewers' recommendation have inspired us to expand our \quotation{About} page to include a reflection on the meaning of \quotation{rambling} which illustrates why this is an appropriate term to employ throughout the site as it conveys the kind of intellectual experience the site has to offer.

Also concerning the \quotation{About} page, the reviewers encouraged us to break down the site credits by expanding on our team members' individual contributions and roles rather than focusing solely on their academic hierarchy. We appreciate this opportunity to elaborate on and recognize individual contributions. When finished, these personal details, along with the reflection on \quotation{rambling,} will provide an introspective look into the modus operandi that guides the work of our team. Over the coming months we will further flesh out the personal details and this reflection.

The recommendation to consistently use tags based on our research themes (national politics, nursing, education, sportfishing, and World War II and the postwar era) is welcomed and solves a problem relating to the overuse of this feature, which we have debated for some time. To effectively and consistently utilize the tags feature, we have decided to implement only tags related to our formal research foci. Commenting on the multimedia captions, the reviewers astutely pointed out the fine balance between \quotation{letting documents speak for themselves and taking on the role of {[}a{]} curator.} The recommendation to provide more framing to improve the user experience will be effective for upcoming stories. Finally, the reviewers suggested several changes to improve the design of the individual story pages. These edits (placement of the oral history audio clips, visual separation of the audio and image sections, and a more legible audio caption font) are sensible and will be integrated in future site updates.

\thinrule

\page
\subsection{Jessica Dawson}

Jessica Dawson has served most recently as Public History Fellow in Research & Technology for \quotation{From Dat Time}: The Oral & Public History Institute of the University of The Bahamas. In this role she acted as webmaster and curator for {\em Ramble Bahamas}. Prior to this, she has taught in the field of cultural anthropology and worked in historic preservation in Ohio state. She holds a B.A. in Anthropology from Washington State University and an M.A.~in American Studies & Public History from Youngstown State University.

\subsection{Tracey Thompson}

Tracey Thompson is the director of \quotation{From Dat Time}: The Oral and Public History Institute of the University of The Bahamas. She oversees the research program and administrative processes of the institute and has been involved in research, teaching, and administration at the University of The Bahamas for more than twenty-five years. Her research focuses principally on African and African diaspora history, the philosophy of history, oral history, and public history.

\stopchapter
\stoptext