\setvariables[article][shortauthor={Wagner}, date={July 2017}, issue={2}, DOI={doi:10.7916/D8J394ZN}]

\setupinteraction[title={*Nou toujou la!* The Digital (After-)Life of Radio Haïti-Inter},author={Laura Wagner}, date={July 2017}, subtitle={Nou toujou la!}]
\environment env_journal


\starttext


\startchapter[title={{\em Nou toujou la!} The Digital (After-)Life of Radio Haïti-Inter}
, marking={Nou toujou la!}
, bookmark={*Nou toujou la!* The Digital (After-)Life of Radio Haïti-Inter}]


\startlines
{\bf
Laura Wagner
}
\stoplines


{\startnarrower\it What happens when a major US academic library meets the single most important archive of twentieth-century Haitian history? What happens when traditional archival practice brushes up against a massive non-English audio collection? What happens when an institution with an emphasis on preservation and research vows to create a multilingual digital archive that is truly accessible, first and foremost, to Haitian people in Haiti? This essay explores the goals, methods, challenges, and meanings of the Radio Haiti archive project, currently underway at Duke University. From the early 1970s until 2003, Radio Haïti-Inter was the voice of the Haitian people, especially those who had been historically and systematically excluded from public discourse and power: rural farmers, grassroots organizers, Vodou adherents, and the urban poor. The director of Radio Haiti, agronomist-turned-journalist Jean Dominique, was a public intellectual with an unwavering commitment to social change, and under his leadership, Radio Haiti developed unparalleled democratic appeal and influence. Radio Haiti's journalists covered stories that no other media would or could cover, traveling to the far reaches of the country and urban no-go zones to report on human rights abuses, repression, and popular resistance. Today, Radio Haiti's archive is an irreplaceable tool in the fight against forgetting, erasure, and impunity; creating this archive and bringing it home to Haiti is {\em devoir de mémoire} (memory work). This essay examines what it means for Radio Haiti to go digital. What is the role of the digital in doing memory work? How is the team at Duke Libraries rethinking and reimagining what \quotation{digital access} means, and how might it use the power of its institution to create a digital collection that is both robust enough to endure and light enough to be accessed on a telephone with a 3G network in rural Haiti? What concrete practices must be developed and employed to make sure that Radio Haiti's legacy reaches as many people in Haiti as possible? \stopnarrower}

\blank[2*line]
\blackrule[width=\textwidth,height=.01pt]
\blank[2*line]

The archive of Radio Haïti-Inter, currently being processed at Duke University's Rubenstein Rare Book and Manuscript Library, is perhaps the most comprehensive archive of late-twentieth-century Haitian history, society, and culture. The collection spans the mid-1960s through 2003 and comprises more than thirty-six hundred reel-to-reel and cassette tapes (which contain more than fifty-two hundred audio files), as well as a handful of DATs, VHS and Betacam tapes, and an estimated one hundred linear feet of paper. Under the direction of Jean Dominique and Michèle Montas, Radio Haiti broadcast the news, investigative reporting, interviews, and incisive editorials, exposing repression, corruption, impunity, and exclusion, from the days of the Duvalier dictatorship through military rule and coups d'état to the early promises and eventual disenchantments of the democratic era.

In 2013, Montas donated the archives of Radio Haiti to Duke, under the condition that the collection be made accessible to people {\em in Haiti}. The creation of this publicly accessible, multilingual, audio archive---this complex and inexact process of unsilencing, of bringing history and memory home via a digital space---demands that we rethink what a university library can do and whom it aims to serve.

\subsection[reference={the-story-of-radio-haiti},
bookmark={The Story of Radio Haiti},
title={The Story of Radio Haiti}]

When Jean Dominique, the outspoken and charismatic director of Radio Haïti-Inter, and his professional partner and wife, Michèle Montas, landed at the airport in Port-au-Prince on 4 March 1986, an estimated sixty thousand people were there to greet them. Photographs and video from that day show Dominique---a man ever unflappable---looking stunned.

They had been away, along with the rest of Radio Haiti's team, since Jean-Claude Duvalier famously declared, \quotation{Le bal est fini} (\quotation{The party is over}), and cracked down on the independent press and human rights activists on 28 November 1980. Dominique took refuge in the Venezuelan embassy, since Jean-Claude Duvalier had issued an order than he be killed on sight. The other journalists of Radio Haiti were arrested, imprisoned at the Casernes Dessalines, and then exiled with nothing more than the clothes on their backs. They were able to return to Haiti only after the regime fell in 1986. That March day at the airport, the crowd lifted Dominique onto their shoulders, and virtually carried the car to Radio Haiti's old station building, on Rue du Quai in downtown Port-au-Prince.

\placefigure{Crowd Greeting Jean Dominique and Michèle Montas, 4 March 1986}{\externalfigure[images/march1986_01.jpg]}
Duvalier's militia, the dreaded Tontons Macoutes, had destroyed the station. The equipment was smashed or stolen, the safe had been broken into, and Dominique's office had been ransacked. But Radio Haiti's tapes had not been destroyed. J. J. Dominique, Jean's daughter and Radio Haiti's station manager, \useURL[url1][https://www.youtube.com/watch?v=-18Ha_A-KuU][][describes]\from[url1] how the archive came to be, how it survived, and how the journalists salvaged all the tapes they could:

\startblockquote
Radyo Ayiti se youn nan kote enpòtans achiv egziste depi lontan. Pwoblèm peyi d Ayiti, poutèt yon pakèt rezon, anpil enstitisyon pa gen achiv. Donk, nan radyo a, ant 1970, lè Jean kòmanse radyo a kòm radyo pa l, epi lè Michèle kòmanse vin travay nan radyo a, yo te kenbe achiv. Sa vle di repòtaj, emisyon kiltirèl, editoryal---tout sa k ap fèt nan radyo a ki enpòtan, yo kenbe yo. E lè gouvènman Duvalier a voye kraze radyo a, nou toujou di, \quotation{Makout yo, yo konn kraze, men yo pa konn valè yon seri de bagay.} Sa vle di yo t ap chèche lajan. Yo kase kòf radyo a pou wè si te gen lajan. Men yo pa panse, yo pa reflechi, ke bagay ki te gen plis valè nan radyo a se travay achiv radyo a. Sa ki fè, an 1986 lè n al relouvri radyo a~.~.~. tout achiv yo te la. Tout anrejistreman, bann mayetik, kasèt, menm plak yo pa vole. Donk, nou rekipere tout bagay sa yo.~.~.~. Evidaman, yo te nan yon eta! Pousye, etsetera! Nou bije pase anpil tan ap netwaye! Nou te konn netwaye plak yo nan gwo kivèt! Yo netwaye plak yo grenn-an-grenn! Nou netwaye plak yo nan kivèt. Donk tout achiv yo, nou repran yo.~.~.~. Bon, genyen ki te tèlman nan yon move eta, nou pa t kapab kenbe yo. Dabò yo te twò vye. Yo te fin kase bout-bout. Nou pa t konn gwo teknik ki te ka pemèt nou kenbe yo. Genyen ki te nan yon move eta men nou reyisi kopye yo. Donk, nou fè sa nou te kapab pou achiv soti 1970 rive 1980, lè radyo a te kraze.
\stopblockquote

\startblockquote
(Radio Haiti was one of the few places that long recognized the importance of archives. The problem is that in Haiti, for many reasons, there are many institutions that don't have archives. So at the station, from 1970, when Jean took ownership of the station, and when Michèle came to work at the station, they kept their archives. That's to say reportage, cultural programs, editorials---everything important that was done at the station, they kept it. And when the Duvalier government came to destroy the station, as we always said, \quotation{The Macoutes, they may destroy, but they don't know the true value of so many things.} That is to say, they were looking for money. They broke the station's safe to see if there was money inside. But they didn't think, they didn't understand that the most valuable thing at the station was the work contained in the station's archive. So in 1986, when we went to reopen the station~.~.~. the whole archive was still there.~.~.~. We recovered all of it. Of course, they were in a state! Dust, etc.! We had to spend a lot of time cleaning them! We cleaned the disks in big basins, we cleaned the disks one by one, we cleaned the disks in those basins.~.~.~. Well, some were in such bad condition we couldn't keep them. First, they were too old, some of them were just falling apart, and we didn't know any fancy techniques that would allow us to keep them. Some were in bad shape, but we managed to copy them. So we did what we could, for the archive spanning 1970 to 1980, when the station was destroyed.)\footnote{J. J. Dominique, interview by Laura Wagner, Montreal, 22 October 2015, \useURL[url2][http://www.youtube.com/watch?v=-18Ha_A-KuU]\from[url2]. All translations are mine.}
\stopblockquote

Over the next seven months, Radio Haiti launched a \quotation{solidarity campaign} to raise money to rebuild the station---it collected small sums, cent by cent, donated by ordinary Haitian people, many of whom had very little, and all of whom felt that Radio Haiti was theirs.

\placefigure{Let Us Bring Haiti-Inter Back to Life}{\externalfigure[images/dove001.jpg]}
\placefigure{Radio Haiti Solidarity Campaign, 1986}{\externalfigure[images/solidarity.jpg]}
They reopened in October 1986, and Jean Dominique \useURL[url3][https://soundcloud.com/radiohaitiarchives/edito-jean-l-dominique-sur-la-reouverture-de-radio-haiti-inter-6-octobre-1986][][returned to the microphone]\from[url3]:

\startblockquote
Içi, Radio Haïti! Içi, Radio Haïti. Içi~.~.~. Radio Haiti. Mais oui, ce n'est pas un rêve. C'est pour de bon, c'est pour de bon cette fois-ci.~.~.~. Après six ans de silence forcé, nous voici à nouveau sur les ondes. Aujourd'hui, et pour de bon cette fois-ci. Aujourd'hui, c'est pour nous un grand jour. Un grand jour pour l'équipe de Radio Haïti reconstitué, un grand jour pour les milliers de cotisants, içi en Haïti, là-bas en diaspora, qui ont contribué à la grande chaîne de solidarité, un grand jour pour les dizaines de milliers de fanatiques qui ont toujours gardé à coeur notre présence, pour tous ceux qui n'ont jamais oublié notre travail. Un grand jour, enfin, pour la liberté de parole dans notre pays.
\stopblockquote

\startblockquote
(Radio Haiti here! Radio Haiti here! Radio Haiti~.~.~. here. No, this is not a dream. This time, it's for good.~.~.~. After six years of forced silence, we are at last back on the air. Today, and for good this time. Today is a great day for us, a great day for the reconstituted Radio Haiti team, a great day for the thousands of contributors here in Haiti and abroad in the diaspora who contributed to the great solidarity chain, a great day for the tens of thousands of fans who kept us in their hearts, for all of those who never forgot our work. And, finally, a great day for the freedom of speech in our land.)\footnote{\quotation{Édito sur réouverture Radio Haiti-Inter, Jean L. Dominique,} 6 October 1986, RL10059RR1362, Radio Haiti Collection, David M. Rubenstein Rare Book and Manuscript Library, Duke University, Durham, NC (hereafter RHC). Listen at \useURL[url4][http://soundcloud.com/radiohaitiarchives/edito-jean-l-dominique-sur-la-reouverture-de-radio-haiti-inter-6-octobre-1986]\from[url4].}
\stopblockquote

\placefigure{Jean L. Dominique's Editorial on the Reopening of Radio Haïti-Inter}{\externalfigure[images/Reouverture.jpg]}
What inspired such devotion, that sixty thousand people would welcome an exiled journalist home, that ordinary people, many of them very poor, would donate their money to bring a radio station back to life?

From the early 1970s until 2003, Radio Haïti-Inter was the voice of the Haitian people, especially those who had been historically and systematically excluded from public discourse and power: rural farmers, grassroots organizers, Vodou adherents, and the urban poor. Jean Dominique was a public intellectual with an unwavering commitment to social change---many of the dispossessed farmers with whom he collaborated referred to him as {\em avoka peyizan yo}, the peasants' lawyer. Under his leadership, Radio Haiti developed unparalleled democratic appeal and influence. Radio Haiti was democratic, in philosophy as well as practice, in no small part because radio itself is a relatively democratic medium in Haiti, something in which everyone can participate. The technology is relatively inexpensive, so an individual without a radio would certainly have a friend, relative, or neighbor who had one. Radio did not require traditional literacy---you did not have to read and write in order to participate as a listener or as a speaker. And Radio Haiti itself was in Haitian Creole in addition to French, demonstrating that the language spoken by all Haitian people could be used for serious topics and serious analysis.

Radio Haiti's journalists covered stories that no other media would or could cover, traveling to the far reaches of the country and into urban no-go zones to report on human rights abuses, repression, and popular resistance. From victims of police brutality in Cité Soleil to sugarcane planters in Léogâne, from {\em oungan} (Vodou priests) from the Artibonite to survivors of the massacre in Jean Rabel, from marketwomen in Port-au-Prince to Haitian braceros in the Dominican Republic---Radio Haiti was where they spoke. Radio Haiti was created for Haitian audiences, by Haitian journalists, and they understood that Haitian people, particularly the most historically marginalized Haitian people, could be the narrators of their own lives and experts on their own country. What Dominique and Radio Haiti believed in and represented, at their deepest level, was an end to exclusion. In this sense, the archive of Radio Haiti fills in some of the gaps in the archival record, the \quotation{silences in the historical narrative.}\footnote{See Michel-Rolph Trouillot, {\em Silencing the Past: Power and the Production of History} (Boston: Beacon, 1995), 53.} It vibrates with the voices of people who are represented nowhere else.

Early in the morning of 3 April 2000, Jean Dominique was shot dead in the courtyard of Radio Haiti along with station employee Jean-Claude Louissaint. Five days later, on 8 April, Charles Suffrard spoke in front of fifteen thousand people at Dominique's state funeral at Sylvio Cator stadium in downtown Port-au-Prince. Suffrard, a rice farmer and peasants' rights organizer from the Artibonite Valley and longtime friend of Dominique's, had appeared on Radio Haiti several times through the years, discussing issues concerning Haiti's long-dispossessed rural farmers:

\startblockquote
Jean, zanmi m. Jean, zanmi peyizan. Jean, pot vwa peyizan. Asasen yo touye w pou laverite. Laverite sou refòm agrè. Laverite sou siwo anpwazonnen. Laverite sou etanòl. Laverite sou kredi. Laverite sou Kazal. Laverite sou masak Janrabèl. Peyizan Latibonit, Pyat, Katye Moren. Laverite sou dechè pwazon Gonayiv. Laverite sou koudeta 30 septanm 1991. Sou koudeta elektoral. Laverite sou zak te konn fè pèp la soufri. Èske asasen konnen ki moun yo t ap touye a? Ki kalite ras kriminèl ki òganize asasina sa a? Fòk ou pa konnen ditou. Fòk ou pa renmen Ayiti ditou. Fòk ou pa gen okenn atachman pou peyi a.~.~.~. Jean bay tout sitwayen konsekan mikwo, pou yo bay opinyon yo libe- libè. Li klè ke tout verite Jean te konn di, ap di yo pou pale pou sa ki pa ka pale, pou sa ki pè pale---tout verite sa yo te menase lespri kriminèl yo.
\stopblockquote

\startblockquote
(Jean, my friend. Jean, friend of the peasants. Assassins killed you over the truth. The truth about agrarian reform. The truth about poisoned cough syrup. The truth about ethanol. The truth about credit. The truth about Cazale. The truth about the Jean Rabel massacre. Peasants of the Artibonite, Piatre, Quartier Morin. The truth about the toxic waste in Gonaïves. The truth about the 30 September 1991 coup d'état. About the electoral coup d'état. The truth about all the acts that made the people suffer. Did these assassins know who they were killing? What kind of criminal could have organized this assassination? You must not have known at all. You must not have loved Haiti at all. You must not have had any attachment to the country at all.~.~.~. Jean gave every citizen the microphone, to give their opinions freely. It is clear that all the truth Jean spoke was for those who could not speak, for those who were afraid to speak---and all that truth threatened the criminals.)\footnote{\quotation{Cérémonie funéraire I,} 8 April 2000, RL10059CS1072, RHC.}
\stopblockquote

A few days later, at the wake for Dominique held by peasant farmers in the Artibonite Valley, Suffrard emphasized the egalitarian nature of his relationship with Dominique, and the democratic nature of knowledge and expertise on Radio Haiti: \quotation{Mwen menm, Charles Suffrard, ki se yon bon patnè Jean Dominique, ki se bon zanmi Jean Dominique---ki se pwofesè Jean Dominique. E fò nou pa sezi tande, Jean Dominique agwonòm, Jean Dominique jounalis, Charles Suffrard peyizan se pwofesè Jean Dominique.}(\quotation{Myself, I am Charles Suffrard. I was a partner of Jean Dominique, I was a good friend of Jean Dominique---and I was Jean Dominique's teacher. You shouldn't be shocked to hear this. Jean Dominique the agronomist, Jean Dominique the journalist---and Charles Suffrard, the peasant farmer, was Jean Dominique's teacher.})\footnote{\quotation{Veye peyizan pou Jean Dominique nan Latibonit,} 15 April 2000, RL10059CS1082, RHC.}

For nearly three years after Dominique's murder, Radio Haiti endured, redoubling their commitment to justice and their opposition to impunity. It was about {\em jistis pou Jando}, in part, but not only justice for Jean Dominique. It was also about truth and justice for the prisoners who died at Fort Dimanche, the peasant farmers killed in Jean Rabel in 1987, the Jean-Bertrand Aristide partisans slaughtered in Raboteau in 1994, the families of the children who died of kidney failure after consuming cough syrup containing antifreeze, the people who died from drinking ethanol masquerading as {\em kleren}---and for all the victims of human rights violations too numerous to cite. Though the precise circumstances of the injustice varied, it was all the same impunity.

On Christmas Day 2002, there was an attempt on Michèle Montas's life in which her bodyguard, Maxime Seïde, was killed. Amid escalating threats to the lives and safety of all the station's journalists, she returned to the studio on 20 February 2003 for what would be her \useURL[url5][https://soundcloud.com/radiohaitiarchives/le-point-fermeture-de-radio-haiti-20-fevrier-2003][][final editorial]\from[url5]: \quotation{Nous avons payé très cher pour cette crédibilité. Nous avons déjà perdu trois vies. Nou refusons d'en perdre advantage.} (\quotation{We have paid dearly for this credibility. We have already lost three lives. We refuse to lose any more.}) She explained that the Radio Haiti team was \quotation{paralysée par des menaces constantes et des dangers evidents} (\quotation{paralyzed by constant threats and obvious dangers}). She continued: \quotation{Sans cette équipe, Radio Haïti n'existe pas.} (\quotation{Without this team, Radio Haiti does not exist.})

\startblockquote
A partir du samedi 22 février, Radio Haïti s'arrêtera temporairement d'émettre, car nous croyons primordial de protéger d'abord des vies. Cela aurait été la décision de Jean Dominique. Ne pouvons aujourd'hui savoir combien de temps durera cette pause nécessaire de silence, de réflexion, et d'évaluation. Nous la souhaitons de courte durée. Nous savons que nos auditeurs et nos commanditaires en comprendront la nécessité. Radio Haïti ne sera pas fermée. La station continuera de fonctionner en silence, travaillant à la production d'émissions de fonds qui seront diffusées ailleurs ou sur nos ondes lorsque les circonstances nous permettront de reprendre notre droit à la parole.~.~.~. Radio Haïti restera ce qu'en avait fait Jean Dominique: un média libre à l'image d'un homme libre.~.~.~. Nous n'abandonnerons jamais notre combat de justice pour Jean Dominique, un combat mené pouce par pouce depuis trois ans. Radio Haïti, même silencieuse, ne sera pas fermée. Nous ne prendrons pas une fois de plus l'exil pour la troisième fois---non. Cette terre et ce rêve de démocratie ont été abreuvées de notre sang. Il sont à jamais notres.
\stopblockquote

\startblockquote
(As of 22 February, Radio Haiti will temporarily cease broadcasting, as we believe the protection of lives is paramount. We cannot know today how long this necessary pause will last, this pause of silence, reflection, and evaluation. We hope it will be brief. We know that our listeners and our sponsors will understand why it is necessary. Radio Haiti will not be closed. We will continue to work in silence, producing in-depth programming that will be broadcast elsewhere, or else on our own airwaves when circumstances again allow us to exercise our right to free speech.~.~.~. Radio Haiti will remain what Jean Dominique made it: a free medium, in the image of a free man.~.~.~. We will never abandon our fight for justice for Jean Dominique, a fight led inch by inch over three years. Radio Haiti, even in silence, will not be closed. Neither will we go into exile once again, for the third time---no. This land, and this dream of democracy, were watered by our blood. They will remain, forever, ours.)\footnote{\quotation{{\em Le point}: Fermeture de Radio Haiti,} 20 February 2003, RL10059CS1859, RHC. Listen at \useURL[url6][http://soundcloud.com/radiohaitiarchives/le-point-fermeture-de-radio-haiti-20-fevrier-2003]\from[url6].}
\stopblockquote

Radio Haiti never reopened.

\subsection[reference={the-project},
bookmark={The Project},
title={The Project}]

\quotation{Archives assemble,} wrote Haitian anthropologist and historian Michel-Rolph Trouillot. \quotation{Their assembly work is not limited to a more or less passive act of collecting. Rather, it is an active act of production that prepares facts for historical intelligibility. Archives set up both the substantive and formal elements of the narrative.}\footnote{Trouillot, {\em Silencing the Past}, 52.} During its first life, Radio Haiti was the radical counternarrative that represented the majority of the Haitian people. Its archive---Radio Haiti's digital afterlife---should reflect that history and likewise endeavor to be radically accessible in terms of both technology and description. What happens when a major US academic library is tasked with processing and rendering \quotation{historically intelligible} the single most important archive of twentieth-century Haitian history? What happens when traditional archival practice collides with a massive non-English audio collection? Can an institution with an emphasis on preservation and research create a multilingual digital archive that is truly accessible, first and foremost, to Haitian people in Haiti? This section explores some of the practical and symbolic implications of the Radio Haiti project---a project that is still underway and riddled with challenges still unresolved.

Radio Haiti had always kept its own archives, as J. J. Dominique explained, and those archives were a living, dynamic part of the station's programming. Over the decades, Radio Haiti's journalists drew on these archives regularly to commemorate anniversaries, to pay tribute to the dead, to draw parallels between the present and the past, and to hold a mirror up to injustice (as I will discuss at greater length momentarily). Each tape was labeled, some in more detail than others, and Radio Haiti had kept a painstaking inventory of their contents. This conscious keeping of archives was unusual for a radio station. \quotation{The sounds of radio are ephemeral,} writes historian Alejandra Bronfman. \quotation{Broadcasts were not meant to be saved, and while many stations did record their broadcasts, they often recorded over them or discarded old recordings to make way for new ones.}\footnote{Alejandra Bronfman, {\em Isles of Noise: Sonic Media in the Caribbean} (Chapel Hill: University of North Carolina Press, 2016), 153.} Indeed, because of financial constraints, Radio Haiti usually kept their daily news broadcasts for only a week before taping over them. But they kept their field reports, editorials, investigative reports, in-studio interviews, and special programming (often copying them from reel-to-reel tapes to cassette tapes, which were smaller and less expensive). The only daily news broadcasts that they kept for good are the ones from the week preceding Jean Dominique's murder.

Archiving Radio Haiti is a project of unprecedented complexity and scope. Audio archives, particularly those in less common languages, are notoriously time consuming and costly to process.

\placefigure{Radio Haiti Reels}{\externalfigure[images/RHreels.jpg]}
We first created a detailed inventory of all the tapes and assigned each tape a unique identifier. This inventory contains the title or contents of the tape (as recorded by Radio Haiti), the date or dates, and preliminary notes on the physical condition (e.g., visible mold, vinegar syndrome, sticky-shed syndrome). We then shipped the tapes to our external vendor, Cutting Corporation, in Maryland, for remediation and digitization. (Of the 3,639 tapes, only 88 of them were damaged to the point of requiring additional intervention---a miracle, given that so many of them sat unguarded in the destroyed station for five and a half years.) Cutting Corporation sent these back to us in batches on hard drives, in WAV format---the standard format for archival preservation. We then converted all the files to mp3 for standard listening (the \quotation{use}copy), because mp3 is the format compatible with most media players, and its small file size is more easily downloaded and shared in Haiti. We also store high-resolution WAV files for professional usage (the \quotation{preservation} copy). The library will retain the physical tapes, with Radio Haiti's original labels, as evidence of the history of the objects, their formats, and their use; however, due to their fragile condition, they cannot be played.

Once the digitized audio returned to Duke, we began writing descriptions---in Haitian Creole, French, and English---of each and every recording and creating other metadata (also trilingual, when necessary): dates, duration, program type, program name, topics, individual and corporate names, and place names. It is imperative to describe each individual recording in detail so that researchers and members of the general public will be able to search the contents. The original Radio Haiti tape title might consist only of the name of a person or an event; Radio Haiti's archiving system, though thorough, was easily usable only to the station's own journalists and technicians, who lived and breathed that history and possessed a shared set of referents. For example, one recording is labeled simply \quotation{Marchaterre 1988.} Our description, in English below, contains significantly more information about the content of the tape, including the speakers involved, in order to make the collection as searchable as possible:

\startblockquote
Special broadcast commemorating the fifty-ninth anniversary of the Marchaterre massacre, drawing on research by historians Kethly Millet and Roger Gaillard. In the 1920s, the US occupation implemented laws that crippled the Haitian sugar and clairin economy and restricted land ownership by Haitian people, which led many people to leave Haiti for Cuba or the Dominican Republic. There was inflation and widespread student strikes. On December 6, 1929, peasant farmers assembled in Marchaterre to protest high taxation by the US marine occupation; the marines fired upon the crowd, killing several people (historians do not agree on how many people were killed). This was one of the events that precipitated the end of the US occupation of Haiti. In Marchaterre in 1988, grassroots peasant groups commemorate the massacre and oppression of peasant farmers from 1929 to the present. They describe their opposition to repressive chefs de section, and the importance of the Marchaterre monument.
\stopblockquote

As we describe the recordings, the trilingual metadata, along with streaming links to each recording, are ingested into two platforms: a traditional Duke University Libraries \useURL[url7][http://library.duke.edu/rubenstein/findingaids/radiohaiti/][][collection guide]\from[url7] and a \useURL[url8][https://repository.duke.edu/dc/radiohaiti][][Duke Digital Repository (DDR) digital collection]\from[url8].\footnote{\quotation{Preliminary Guide to the Radio Haiti Audio Recordings, 1957--2003,} Duke University Libraries, \useURL[url9][http://library.duke.edu/rubenstein/findingaids/radiohaiti/][][library.duke.edu/rubenstein/findingaids/radiohaiti,]\from[url9] and \quotation{Radio Haiti Archive Digital Collection,} Duke University Libraries Digital Repository, \useURL[url10][https://repository.duke.edu/dc/radiohaiti][][repository.duke.edu/dc/radiohaiti]\from[url10].}

\placefigure{Radio Haiti Archive Home Page, Duke Digital Repository}{\externalfigure[images/DDR_page_2.png]}
The collection guide functions as a searchable inventory of the entire archive. The DDR, which is built on a Fedora 3 repository system using a Hydra framework, provides the Radio Haiti audio with a robust and safe permanent digital home. The Radio Haiti digital collection on the DDR offers a more customizable experience than the standard guide, as it is facetable by subject, speaker, program time, program name, and location. The descriptions will all be searchable on Google. Without a high level of description and the creation of trilingual metadata, the Radio Haiti archive would be far less accessible and usable. Because our objective is to make the archive as accessible as possible to as many people as possible, particularly in Haiti, \quotation{overdescribing} the audio is necessary.

Because it contains real-time, on-the-ground accounts of events that are not documented elsewhere, and the voices and experiences of people who are not represented in other sources, the archive of Radio Haiti fills in some of the gaps in the historical record. Still, as Trouillot explains, \quotation{Silences are inherent in the creation of sources, the first moment of historical production.}\footnote{Trouillot, {\em Silencing the Past}, 51.} Moreover, questions of the politics of knowledge are not limited to Radio Haiti. They extend to the power of archival production as well.\footnote{See Trouillot, {\em Silencing the Past}; and Ann Laura Stoler, {\em Along the Archival Grain: Epistemic Anxieties and Colonial Common Sense} (Princeton, NJ: Princeton University Press, 2009).} Detailed, trilingual description is better than no or minimal description, but it is necessarily incomplete and necessarily shaped by the archivist's understanding of what is important. There is a fundamental tension between understanding and reflecting on the epistemological dilemmas of knowledge production and doing this work every day, of having a finite amount of time to work through more than fifty-two hundred recordings. There is little time to linger.

The mere existence of the descriptive information does not mean that that information is usable. This is why social media outreach is crucial, and also why, in the summer of 2016, we distributed a \useURL[url11][http://blogs.library.duke.edu/rubenstein/2016/07/01/12259/][][thousand flash drives]\from[url11] with a small sample of Radio Haiti content to schools, community radio stations, grassroots groups, and cultural institutions in Haiti.\footnote{Laura Wagner, \quotation{Bringing Radio Haiti Home, One Step at a Time,} 29 June 2016, {\em H-Net}, \useURL[url12][http://networks.h-net.org/node/116721/blog/h-haiti-blog/132330/bringing-radio-haiti-home-one-step-time]\from[url12]; reposted 1 July 2016, {\em The Devil's Tale}, \useURL[url13][http://blogs.library.duke.edu/rubenstein/2016/07/01/12259]\from[url13].} To encourage public interest in and awareness of the project, particularly in Haiti, and to continue to share material even though the collection is not completely processed, I maintain a Facebook page and a Twitter account, and I also blog periodically. I regularly post recordings and occasionally photographs, texts, ephemera, and other items, especially to commemorate major anniversaries and important dates. This was something Radio Haiti themselves did, as part of their ongoing commitment to {\em devoir de mémoire} (memory work, the duty to remember), and it is something we seek to do as well in creating this archive.

\subsection[reference={memory-work},
bookmark={Memory Work},
title={Memory Work}]

As Barbie Zelizer argues, \quotation{Memory's work on journalism does not reflect journalism's work on memory.} Scholars of memory rarely draw on journalistic sources, and by the same token, \quotation{the relevance of journalists' work to understanding the past~.~.~. is not necessarily admitted by journalists, who neither explicitly speak of the past nor consider the past as part of their obvious purview.}\footnote{Barbie Zelizer, \quotation{Why Memory's Work on Journalism Does Not Reflect Journalism's Work on Memory,} {\em Memory Studies} 1, no. 1 (2008): 80.}

In the case of Radio Haiti, however, journalism and memory work were explicitly intertwined. For more than thirty years, Radio Haiti was on the frontlines of the battle against forgetting and injustice. Its journalists not only reported on but consciously revisited, year after year, the violent legacy of Duvalierism, the repression of peasants and the urban poor under successive military regimes, the human rights violations committed by the army and the paramilitary during the coup years, and corruption and insecurity during the democratic era. These events were not isolated horrors but part of a pattern: the machinery and mentality of oppression, of Macoutism, was not confined to any time period nor to any single form of government. \quotation{While we continued to give a voice to the victims of the past, it became obvious that covering the ongoing news was in itself building memory, since what we had inherited, with the army in power, was a veiled form of Duvalierism without Duvalier,} Michèle Montas reflects.\footnote{Michèle Montas, \quotation{Unearthing Haiti's Buried Memories,} in Merilee S. Grindle and Erin E. Goodman, eds., {\em Reflections on Memory and Democracy} (Cambridge, MA: David Rockefeller Center for Latin American Studies, 2016), 58.} It was a grim national calendar, and Radio Haiti's staff invoked it time and again, and the dates themselves became infused with meaning---what J. J. Dominique calls {\em le poids des dates}, the \quotation{weight of dates.}\footnote{Jan J. Dominique, {\em Mémoire errante} (Montreal: Éditions du Remue-ménage, 2008), 150.}

\useURL[url14][http://blogs.library.duke.edu/rubenstein/2016/04/26/duvalierism-without-duvalier-radio-haiti-commemorates-massacres-april-26-1963-1986/][][26 April.]\from[url14] The day Duvalier {\em père} unleashed a wave of violence on military officers, opponents, and supposed opponents, in 1963. And then, twenty-three years later, at a march to commemorate the 1963 massacre, the army fired on the crowd assembled peacefully in front of the infamous political prison Fort Dimanche, killing several people, among them adolescent boys who were still in high school.\footnote{Laura Wagner, \quotation{Duvalierism, with and without Duvalier: Radio Haiti Commemorates the Massacres of April 26, 1963 and 1986,} 26 April 2016, {\em The Devil's Tale}, \useURL[url15][http://blogs.library.duke.edu/rubenstein/2016/04/26/duvalierism-without-duvalier-radio-haiti-commemorates-massacres-april-26-1963-1986]\from[url15].}

\useURL[url16][http://blogs.library.duke.edu/rubenstein/2015/11/20/radio-haiti-you-are-the-rain-if-you-didnt-fall-we-could-not-bloom-repression-and-remembrance-on-november-28/][][28 November.]\from[url16] The day of the 1980 crackdown and then, five years later, the day the army killed three schoolboys, the Twa Flè Lespwa, in Gonaïves, setting off widespread resistance to the regime.\footnote{Laura Wagner, \quotation{\quote{Radio Haiti, You Are the Rain. If You Didn't Fall, We Could Not Bloom}: Repression and Remembrance on November 28,} 20 November 2015, {\em The Devil's Tale}, \useURL[url17][http://blogs.library.duke.edu/rubenstein/2015/11/20/radio-haiti-you-are-the-rain-if-you-didnt-fall-we-could-not-bloom-repression-and-remembrance-on-november-28]\from[url17].} (The next day, 29 November, would be remembered as the date the army violently shut down Haiti's first attempt at democratic elections in 1987.)

11 September. The day of the 1988 St.~Jean Bosco massacre, when armed aggressors under the orders of Port-au-Prince mayor and former Macoute Franck Romain slaughtered parishioners at Father Jean-Bertrand Aristide's church in the shantytown of La Saline. Among the victims was a baby girl stabbed while still inside her mother's womb; miraculously, both the baby (who was given the pseudonym \quotation{Esperancia} by Radio Haiti) and her mother survived. Five years later, on another 11 September in the heart of the coup years, political activist Antoine Izméry, while attending a mass at commemorating the 1988 massacre, was dragged out of Église Sacre-Coeur and shot dead in the street.

\placefigure{Esperancia}{\externalfigure[images/esperancia.jpg]}
30 September. The day General Raoul Cédras overthrew the democratically elected government of Jean-Bertrand Aristide. Haiti was ruled for the next three years by a bloody military junta that violently repressed pro-Aristide partisans (particularly the very poor) and used systematic rape as a weapon. The date itself was so powerful that when activist \useURL[url18][https://soundcloud.com/radiohaitiarchives/sets/face-lopinion-lovinsky-pierre][][Lovinsky Pierre-Antoine]\from[url18] created an organization to advocate for the rights of victims of the coup years, he christened it Fondation 30 Septembre.\footnote{\quotation{{\em Face à l'opinion}: Lovinsky Pierre-Antoine sou jistis, enpinite ak memwa aprè koudeta,} 16 December 1997, RL10059CS0527, RHC. Listen at \useURL[url19][http://soundcloud.com/radiohaitiarchives/sets/face-lopinion-lovinsky-pierre]\from[url19].}

Just as certain dates were heavy, certain place names were as well. In 1999, while conducting an interview about the legacies of Duvalierism, Jean Dominique described the importance of {\em devoir de mémoire}:

\startblockquote
Genyen yon travay ki te fèt, ou yon travay ki pa janm fèt, ki fè ke moun pa sonje. Se sèlman trant an! Se trant an sèlman! Sa vle di ke genyen yon seri de bagay ke moun pa konnen.~.~.~. Rejim Duvalier a, omwen li mete karant a senkant mil ayisyen atè, li kraze karant a senkant mil moun ki asasinen pandan tout tan rejim nan te kanpe nan peyi a. O, ou rann kont ke divalyeris yo pa wont pou yo parèt figi yo sou nou. Y ap fè yon kalkil: \quotation{Ah! Pèp ayisyen bliye! Pèp ayisyen bliye. Nou mèt konnen ke nou menm nou te lèd, men yo menm, yo pa konn sa. Donk, nou la!} Donk, sa vle di ke kesyon memwa se yon bagay ki enpòtan {\em konkrètman}. Lò, an '86--'87, tout moun t ap di \quotation{Jamais plus! Jamais plus!} Men, {\em jamais plus} a te vle di \quotation{ann kondane sa k te pase a, pou l pa ka rekòmanse.} Donk memwa genyen yon rezon {\em pratik}, pou salopri sa yo pa rekòmanse.
\stopblockquote

\startblockquote
(Something happened, or something didn't happen, that causes people not to remember. It's only been thirty years! It's only thirty years! That means that there's a series of things that people don't know.~.~.~. The Duvalier regime struck down at least forty to fifty thousand Haitians, it destroyed, killed forty to fifty thousand people during the whole time the regime ruled the country. And you see that, now, the Duvalierists are not ashamed to reappear, right in front of our eyes. They are calculating: \quotation{Ah, the Haitian people have forgotten! The Haitian people have forgotten! Ourselves, we may know that we were monsters, but as for them, they don't know that! So, here we are!}\footnote{The original Haitian Creole is \quotation{Nou mèt konnen ke nou menm nou te lèd, men yo menm, yo pa konn sa! Donk, nou la!} Dominique is playing with and twisting the well-known saying, \quotation{Pito nou lèd, men nou la!}(\quotation{We may be ugly, but we're here!}), which means, more or less, \quotation{We will not give up, even if others declare us ugly, even if others say we aren't people.} The playfulness is sadly lost in translation, and the term {\em lèd} (which literally means \quotation{ugly,} but here refers to cruel and vicious acts rather than appearance) has been adapted.} This means that the question of memory is important, in a {\em concrete} way. In '86--'87 {[}after Duvalier fell{]}, everyone was saying, \quotation{Never again! Never again!} But that \quotation{never again} meant, \quotation{Let's condemn what happened in the past so it cannot begin again.} Memory has a {\em practical} purpose, so these terrible things do not begin again.)\footnote{\quotation{Face à l'opinion: Nathalie Lamaute-Brisson,} 5 August 1999, RL10059CS0973, RHC.}
\stopblockquote

Over the years, \quotation{Nou p ap manje manje bliye} became one of Radio Haiti's slogans: \quotation{We refuse to eat the fruit of forgetfulness.} And as time passed, it had more and more to remember---including, in its final years, the assassination of the station's own director. Memory work can be dangerous work. Assuming her husband's role of editorialist, Montas began each broadcast by stating how many days, weeks, months, and eventually years had passed since Dominique's murder---how many days, weeks, months, and years had passed without justice. Dominique, like so many others, \useURL[url20][http://lenouvelliste.com/lenouvelliste/article/126501/Affaire-Jean-Dominique-les-dates-les-gens-et-les-faits-connus.html][][never found justice]\from[url20], not officially.\footnote{\quotation{Affaire Jean Dominique, les dates, les gens et les faits connus.} 20 January 2014, {\em Le Nouvelliste,} \useURL[url21][http://lenouvelliste.com/lenouvelliste/article/126501/Affaire-Jean-Dominique-les-dates-les-gens-et-les-faits-connus.html]\from[url21].} When the justice system itself is sick, the only way to combat impunity is to remember.

Because it contains both a real-time record of so many key moments in twentieth-century Haitian history---including some that appear nowhere else---as well as analysis of events past, Radio Haiti's archive continues to be an irreplaceable tool in the fight against erasure and impunity. Creating this archive and bringing it home, to Haiti, is a secondary ​{\em devoir de mémoire}.

\subsection[reference={square-pegs-round-holes},
bookmark={Square Pegs, Round Holes},
title={Square Pegs, Round Holes}]

To believe in this project is to force it into existence, for the very university that took on the Radio Haiti project is unable to do everything necessary for it to be a publicly accessible, trilingual archive. This is a weakness not of personal or institutional will but of technology, a fundamental shortcoming of the tools available to us at present.

Our first duty was to preserve and digitize the recordings---to keep them safe, to grant them enduring life. This we have accomplished. Often, with traditional archive projects, preservation is the final goal. This is not the case for Radio Haiti. The preservation of the Radio Haiti archives was never an end unto itself but a necessary intermediate step in creating a fully accessible public collection. If the Radio Haiti archives cannot reach Haitian people, this project has failed. Among Radio Haiti's core principles were the end of exclusion and the persistence of memory; we must do the same: \quotation{There is no political power without control of the archive, if not of memory,} Jacques Derrida argues. \quotation{Effective democratization can always be measured by this essential criterion: the participation in and the access to the archive, its constitution, and its interpretation.}\footnote{Jacques Derrida, \quotation{Archive Fever: A Freudian Impression,} trans. Eric Prenowitz, {\em Diacritics} 25, no. 2 (1995): 11.} We aspire to Derrida's ideal imperfectly, since the day-to-day function of a university library is by its nature hierarchical, placing the power of description and naming in the hands of very few.

The Radio Haiti project shares some traits with what Abigail De Kosnik terms \quotation{rogue archives}: the project prioritizes accessibility to all online users (no fees, no paywalls, no required institutional affiliation) and the ability to download and stream content in its entirety. The Radio Haiti project is, however, located in what De Kosnik terms a \quotation{traditional memory institution}---in this case, a major university library.\footnote{Abigail De Kosnik, {\em Rogue Archives: Digital Cultural Memory and Media Fandom} (Cambridge, MA: MIT Press, 2016), 18, 2.} In that key sense, we are not rogue at all. We are a nontraditional project in a very traditional institution, endeavoring to democratize a crucial piece of Haitian cultural memory. Though rogues we are not, we must at times think like them.

And so---straining against the standard expectations and pushing the boundaries of the university library that houses it, while enjoying the stability and permanence that only a major, well-funded institution can provide---the Radio Haiti archive project lays bare many of the assumptions that academic libraries make about who resources are intended for and how they access them and why these questions matter.

Radio Haiti's canon is not the standard canon of traditional archives; Library of Congress--authorized headings are ill-suited to this project. Many central Haiti-specific terms do not have authorized headings, and so we have had to create a massive local, controlled vocabulary, filled with major events and the names of grassroots leaders, Vodou {\em lwa}, Tontons Macoutes and members of the 1990s paramilitary, and towns and villages, as well as terms such as {\em dechoukaj} (the uprooting of Duvalierists) that exist only in Haitian Creole. These terms are central toHaitian history but are unknown to the Library of Congress authorities. When an omission is truly untenable, we submit it for the laborious process of inclusion in the Library of Congress. Before we began this project, for example, \quotation{Radio Haïti-Inter} did not have an authority record.

Some terms do appear in the Library of Congress headings, but they are antiquated, misinformed, and, to put it plainly, racist. When existing Library of Congress headings are objectionable, we simply refuse to include them and instead create local headings, while concurrently petitioning the Library of Congress to change the headings, following the example of the scholars who successfully argued for \quotation{Voodooism} to be changed to \quotation{Vodou.}\footnote{See Kate Ramsey, \quotation{From \quote{Voodooism} to \quote{Vodou}: Changing a US Library of Congress Subject Heading,} {\em Journal of Haitian Studies} 18, no. 2 (2012): 14--25.} Among the most objectionable are the headings for the 1937 massacre of Haitian people by the Dominican army, which is currently \quotation{Dominican-Haitian Conflict, 1937,} and the heading for the Haitian Creole (Kreyol) language, which is \quotation{Creole dialects, French---Haiti.} As for the former, it was no \quotation{conflict} between two equally matched powers but the targeted and racist slaughter of unarmed civilians by the Dominican military under the direct orders of President Rafael Trujillo. This semantic sleight of hand allows people to conflate and forget who is the oppressor and who is the oppressed. As for the latter, Haitian Creole is not a dialect of French but a language in its own right. Haitian Creole possesses its own grammar, vocabulary, and official orthography. Designating Haitian Creole a \quotation{dialect} further marginalizes an already-stigmatized language. Since the colonial era, French has been the language of power in Haiti, a tool that has allowed social and economic elites to oppress the majority of people and exclude them from public discourse.\footnote{See Michel DeGraff, \quotation{Creole Exceptionalism and the (Mis)Education of the Creole Speaker,} in~Jo Anne Kleifgen and George Bond, eds., {\em The Languages of Africa and the Diaspora: Educating for Language Awareness} (Bristol: Multilingual Matters, 2009), 124--44.} Only after the fall of the Duvalier regime was Haitian Creole recognized as one of the two official languages of Haiti, and many people, both in Haiti and beyond, still consider Haitian Creole to be inferior to French. Despite this, Haitian Creole is increasingly a language of serious discourse and literary and pedagogical importance. As the first radio station to broadcast news, investigative journalism, serious analysis, and interviews with scholars, politicians, and writers in Haitian Creole, Radio Haiti was one of the pioneers. This was an activist stance, an act of social and linguistic inclusion in a country built, from its very conception, on exclusion.

The \quotation{more product, less process} (MPLP) approach, as advocated by Mark Greene and Dennis Meissner, has been influential among research libraries in recent years. They assert that traditional archival description is inefficient, and that, in order to prevent backlogs, processing archivists should provide only the minimum amount of information required to access collections.\footnote{See Mark A. Greene and Dennis Meissner, \quotation{More Product, Less Process: Revamping Traditional Archival Processing,} {\em American Archivist} 68, no. 2 (2005): 208--63.} For the Radio Haiti project, the MPLP logic and methodology are antithetical to our goals. Audio, unlike print material, is virtually unskimmable. The more detail we include in our descriptions, the more searchable and accessible the archive becomes. As Mary Caton Lingold, Darren Mueller, and Whitney Anne Trettien put it, \quotation{While digital media thus create a space of possibility for the study of sound, it is critical, interpretive labor that fulfills this potential, not the technology itself.}\footnote{Mary Caton Lingold, Darren Mueller, and Whitney Anne Trettien, preface to Mary Caton Lingold, Darren Mueller, and Whitney Anne Trettien, eds., {\em Digital Sound Studies: A Provocation} (Durham, NC: Duke University Press, forthcoming), 7.} The longer we work on this project, the more accustomed we become to the contours and nuances of the archive and the quicker we can go. Still, it remains slow work. And the imperative to do slow, careful work comes into conflict with the exigencies of funding the project.

Translation of text is only one part of internationalization; there is translation of language as well as translation of technology. The products we are using to create the eventual Radio Haiti Web framework remain unequal to the task. We are translating thousands of subjects (such as topics, individual and corporate names, place names) to be facetable search terms in each of the three languages, but the Hydra platform of Duke University's digital repository is unable to separate the terms by language. The translations are, as far as we know, going to be an undifferentiated blob.

Robust though Duke's digital repository may be, it is not accessible to the average Haitian user, who is likely to have a smartphone on a 3G network with a data plan that uses minutes bought daily and who uses apps instead of browsers. In fact, a standard Duke collection guide loads so slowly, it is prohibitive on most Internet connections in Haiti. In theory, the beauty of digitization is that the collection should be available anywhere in the world. But what good is digitization if the collection is available in name only? This project presents a challenge to Duke: the imperative to think of audiences far beyond the English-speaking researchers that form the majority of the library's target audience. If someone in a rural community or an urban {\em katye popilè} in Haiti can access and use the archive, then US-based researchers will be fine.

We also continue to face funding challenges, not in the least because the current US regime is threatening to eliminate the National Endowment for the Humanities, which funds the Radio Haiti project, as well as other \quotation{wasteful} institutions, such as the National Endowment for the Arts and the Corporation for Public Broadcasting---this, despite the fact that \useURL[url22][https://www.nytimes.com/2017/03/15/arts/nea-neh-endowments-trump.html?_r=0][][these programs represent but 0.0625\letterpercent{} of the federal budget]\from[url22].\footnote{See Sopan Deb, \quotation{Trump Proposes Eliminating the Arts and Humanities Endowments,} 15 March 2017, {\em New York Times}, \useURL[url23][http://www.nytimes.com/2017/03/15/arts/nea-neh-endowments-trump.html][][http://www.nytimes.com/2017/03/15/arts/nea-neh-endowments-trump.html?_r=1]\from[url23].} It is not an austerity measure; it is a slap in the face.

\subsection[reference={coda-rezistans},
bookmark={Coda: *Rezistans*},
title={Coda: {\em Rezistans}}]

Let us return to the moment of crisis and uncertainty with which this story began. In the late 1970s, in the years just preceding the November 1980 crackdown, Radio Haiti lost its on-air sponsors. A new censorship law was in effect, ostensibly inspired by the \quotation{licentious} nature of Frankétienne's play {\em Pèlin Tèt};\footnote{\quotation{Communiqué sur la censure,} 10 May 1979, RL10059RR0160, RHC; and \quotation{Censure et théâtre,} May 1979 RL10059CS0016, RHC.} in truth, it was an attempt to muzzle the independent media. Radio Haiti was hit by a spurious lawsuit, supposedly brought against it by the widow of Ricardo Widmaïer, the former owner of the station; it was, in fact, a Duvalierist ploy to drive Jean Dominique out of business. The businessmen of Port-au-Prince did not want to be associated with a station that was in such open and dangerous opposition to the regime, so Radio Haiti had to be resourceful. \useURL[url24][http://www.alterpresse.org/spip.php?article21161\#.WRES8RLyuu4][][Richard Brisson]\from[url24] (who, unable to bear exile, would be executed in January 1982 in a failed and quixotic attempt at invasion) famously used his car as a taxi.\footnote{See Roody Edmé, \quotation{Haïti-Mémoire: Il était une fois Richard Brisson,} 16 January 2017, {\em AlterPresse}, \useURL[url25][http://www.alterpresse.org/spip.php?article21161\#.WS7zO9y1thF]\from[url25].} Radio Haiti held a raffle of paintings by renowned Haitian artists; each ticket sold for \$3.

\placefigure{Raffle Ticket, December 1979}{\externalfigure[images/raffle004.jpg]}
By October 1980, Radio Haiti knew their days were numbered. Jean-Claude Duvalier had outlawed all but the official government media. Dominique responded with one of his \useURL[url26][https://soundcloud.com/radiohaitiarchives/sets/bon-app-tit-messieurs][][most famous broadcasts]\from[url26], a mordant editorial addressed to the \quotation{messieurs de la presse officielle} (\quotation{gentlemen of the official press}):

\startblockquote
Dans le journal officiel, on nous a dit hier, \quotation{Le bal est fini, messieurs.} {\em Bal la fini, ebyen, bal la fini!} Nous n'étions pas au bal, quant à nous. Nous n'avions pas de masques, nous. Pour nous, ce n'était pas un Mardi Gras. Pour vous, peut-être. . . . Pour vous, le banquet va reprendre. Ah, oui, ça: le banquet va reprendre. Cette fois-ci, vous n'entendrez pas de sons discordants, des bruits qui vous empêcheraient de manger. Vous n'entendrez pas, pour vous distraire votre festin plantureux, les cris des malheureux, les hurlements des boat people dévorés par les requins, les coups de fouet ou les coups de pistolet attaignant nos frères braceros à Santo Domingo ou bien à Nassau, à La Romana. Non, nous n'entendrez plus les bruits discordants et désagréables pour troubler votre repas, pour vous empêcher de festoyer. Ce sera le silence. Le silence rassurant. . . . Là aussi, il y aura silence, silence total. Vous pourrez donc festoyer à l'aise, messieurs! A l'aise! Et dans le silence recueilli: Bon appétit, messieurs!
\stopblockquote

\startblockquote
(In the official newspaper, they told us yesterday, \quotation{The party is over, gentlemen.} {\em The party is over, well, the party is over!} We weren't at the party, ourselves. We had no masks, ourselves. For us, it was no Mardi Gras. For you, perhaps. . . . For you, the banquet shall resume. And you will not hear any discordant sounds, any noise that might disturb your appetites. You will not be distracted from your plentiful feast by the cries of the poor, the screams of the boat people devoured by sharks, the lashes of the whip, the gunshots upon our cane-cutter brothers in Santo Domingo or in Nassau or in La Romana. No, you will not hear those discordant, disagreeable noises that might trouble your meal, that might prevent you from celebrating---it will be only silence. A reassuring silence. . . . There will be silence, total silence. So you may celebrate at ease, gentlemen! At ease! And in that profound silence: {\em Bon appétit, messieurs!})\footnote{\quotation{{\em Le point}: Bon appétit, messieurs!,} 20 October 1980, RL10059CS0978, RHC. Listen at \useURL[url27][http://soundcloud.com/radiohaitiarchives/sets/bon-app-tit-messieurs]\from[url27].}
\stopblockquote

And for the six years that followed, silence reigned. Radio Haiti's team returned in 1986 and reopened the station, with the assistance of the grassroots solidarity campaign, while the word on the street was, \quotation{Baboukèt la tonbe!} (\quotation{The gag has fallen!}). In short, Radio Haiti persevered the face of a hostile system that was constantly at its throat. This puts our own financial, technological, and institutional challenges in perspective. And at times, it serves as a warning.

Radio Haiti remained tenacious in the face of dictatorship and privation, through arrests, imprisonment, torture, exile, and assassination. The Radio Haiti archive is a reminder of repression, but its survival and continued resonance is a testament to the ultimate failure of that repression. We are poised at a bleak and dangerous crossroads, presided over by an antitruth, antifact, antimedia American regime. It is a moment in which forces of cruelty, hatred, and fear threaten to overtake this so-called democracy, a moment in which the most powerful sectors of society are endeavoring, by legislation and by force, to bulldoze the human and civil rights of the most powerless and marginalized, a moment in which the president of the United States has declared journalists the \quotation{enemy of the American people.} It is a moment in which truth appears no longer to matter, a moment of \quotation{alternative facts,} where the most blatant and odious sorts have lies appear to have triumphed. The relevance of Radio Haiti's example is a lesson, a warning, a comfort, and a call to action: {\em yes} to investigative journalism, {\em yes} to human rights, {\em yes} to facts, {\em yes} to truth, {\em yes} to resistance.

\thinrule

\subsection[reference={useful-links},
bookmark={Useful Links},
title={Useful Links}]

\startitemize[packed]
\item
  Facebook: \useURL[url28][http://www.facebook.com/radiohaitiinter][][www.facebook.com/radioHaitiinter]\from[url28]
\item
  Twitter: \useURL[url29][https://twitter.com/achivradyoayiti][][@AchivRadyoAyiti]\from[url29]
\item
  Pilot site: \useURL[url30][http://www.radiohaitilives.com][][www.radiohaitilives.com]\from[url30]
\item
  Preliminary collection guide: \useURL[url31][http://www.library.duke.edu/rubenstein/findingaids/radiohaiti][][www.library.duke.edu/rubenstein/findingaids/radiohaiti]\from[url31]
\item
  Radio Haiti Archive---Duke Libraries: \useURL[url32][https://repository.duke.edu/dc/radiohaiti][][repository.duke.edu/dc/radiohaiti]\from[url32]
\stopitemize

\thinrule

\page
\subsection{Laura Wagner}

Laura Wagner, PhD, is the Radio Haiti project archivist at the Rubenstein Rare Book and Manuscript Library at Duke University. She received her BA from Yale University and her PhD in anthropology from UNC--Chapel Hill, where her research focused on the aftermath of the 2010 earthquake in Haiti. She is also a fiction and nonfiction writer.

\stopchapter
\stoptext