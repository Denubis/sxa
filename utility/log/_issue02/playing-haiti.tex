\setvariables[article][shortauthor={Lauro}, date={July 2017}, issue={2}, DOI={doi:10.7916/D8DB8D8S}]

\setupinteraction[title={Digital Saint-Domingue: Playing Haiti in Videogames},author={Sarah Juliet Lauro}, date={July 2017}, subtitle={Digital Saint-Domingue}]
\environment env_journal


\starttext


\startchapter[title={Digital Saint-Domingue: Playing Haiti in Videogames}
, marking={Digital Saint-Domingue}
, bookmark={Digital Saint-Domingue: Playing Haiti in Videogames}]


\startlines
{\bf
Sarah Juliet Lauro
}
\stoplines


{\startnarrower\it This article comes out of a longer project looking at digital commemorations of slave rebellion. In this excerpt from that work, the author considers the issues at stake in videogamic representations of colonial Saint Domingue and its denizens, particularly for their depiction of the prehistory of the Haitian Revolution. In two mainstream videogames, both part of the Assassin's Creed franchise, the history of Saint Domingue, its legacy of slave resistance, and the Haitian Revolution are made into fodder for an interactive entertainment experience that intervenes in and reshapes history in a complex manner. There are several issues at stake, which the author focuses on exclusively in terms of the commodification of Saint Domingue. First, the games place the history of slave revolt into the hands of game players of diverse ancestry, allowing for a redistribution of ownership over narratives of emancipation and empowerment. Second, the games identify themselves as tampering with history, and their mélange of fictional characters and real personages seems to risk rewriting the history of Saint Domingue's legacy of slave revolt and---by extension---of the Haitian Revolution itself. Given recent events in the United States, and increased attention to strategies of black resistance such as the Black Lives Matter movement, it seems all the more imperative that our depiction of slave revolts in popular culture be handled with care. And yet, the author finds a subversive maneuver visible in the games: the use of untranslated language, especially Haitian Kreyol, may work to preserve and limit the player's mastery over these histories. This article provides a tour of this complex territory of digital Saint Domingue. \stopnarrower}

\blank[2*line]
\blackrule[width=\textwidth,height=.01pt]
\blank[2*line]

My students' reaction to discovering (belatedly, they feel) the facts of the Haitian Revolution never ceases to surprise me. Many of them have never heard of this miracle, in which the slaves of the French colony of Saint-Domingue rose up against their masters in 1791---often armed only with tools from the canefields---and, forging alliances with various parties, waged an eleven-year-long war against the major European militaries fighting to keep them under the heel of colonial servitude, to declare in 1804 the sovereignty of their own country, the first black republic of Haiti. This gap in my students' education is sometimes met with wonder, hurt, and even {\em anger}---an anger directed at the US-American high school history curriculum. But what is perhaps even stranger than this lacuna in their schooling, or their reaction to its revelation, is the fact that many of them may have encountered the Haitian Revolution, or at least its prehistory, without even realizing it, perhaps not in the AP History classroom but in playing X-box on a Saturday morning: the Haitian Revolution has provided the context for at least two recent games in videogame manufacturer Ubisoft's Assassin's Creed franchise---{\em Liberation} and {\em Freedom Cry}.

In brief, the texts this article engages are at once mainstream entertainment videogames and interactive tales of defiant people of color in the colonial era. The playable protagonists are the descendants of Africans imported to the Caribbean sugar colonies, working to free themselves and others, fighting alongside, and sometimes disguised as, slaves. My concern, as I began to investigate these AAA titles, was to determine whether Haiti's legacy of slave resistance was being reduced to an entertainment commodity and what kind of experience was being offered to the player in allowing him or her an experiential role in that history. In considering what kind of depiction of slave revolt is fostered by such games, several issues came to the foreground. First, does the developers' act of drawing on Haitian history truly expand a gamer's knowledge of historical resistance in the Atlantic world, or is this enfolding of Haitian history into the fictional world of the game merely an act of what theorist bell hooks might call \quotation{eating the other}?\footnote{bell hooks, \quotation{Eating the Other: Desire and Resistance,} in {\em Black Looks: Race and Representation} (Boston: South End, 1992), 21--39, \useURL[url1][http://genius.com/Bell-hooks-eating-the-other-desire-and-resistance-annotated][][genius.com/Bell-hooks-eating-the-other-desire-and-resistance-annotated]\from[url1].} Second, does the game, whether historically accurate or not, appear to foster empathy for the historical rebel slave, as some educational and art games about runaway slaves arguably do?\footnote{I am currently working on a book project that will address more fully the range of games in which slave resistance is rendered playable. Some are mainstream entertainment games like those described here, but there are also art games and educational games aimed at elementary school children that depict runaway slaves, which I include in my study. See, for example, the art game {\em Thralled}, a side-scrolling puzzle game about a runaway slave, created by students at the USC Games Lab, and {\em Mission US: Flight to Freedom}, an educational game about the Underground Railroad. I argued in \quotation{Kill the Overseer! Playing the Rebel Slave in Videogames,} a recent talk given at the Experimental Media and Performing Arts Center, Rensselaer Polytechnic Institute, Troy, NY, on 1 February 2017, that both these games use emotional stress (the persistent sound of a crying baby in the first, the demand to click through a series of chores before the overseer metes out punishment in the second) to heighten the player's experience, fostering empathy for the playable character as a representative of historical slave resistance. See \useURL[url2][http://web.archive.org/web/20170904034530/http://empac.rpi.edu/events/2017/spring/kill-overseer][][empac.rpi.edu/events/2017/spring/kill-overseer]\from[url2].} Or does it instead prioritize the playability of this subject position in a problematic manner, bringing the rebel slave of Saint-Domingue under the control of the contemporary gamer, chiefly, for fun? None of these questions proved simple to answer. In this piece, I'll adumbrate the various complexities of the games' representations of Caribbean slave revolt.

As this article emphasizes, the history of Saint-Domingue, in particular, plays a key role in both games. {\em Liberation} and {\em Freedom Cry} make problematic use of the prehistory of the Haitian Revolution as a kind of backdrop or subtext to the action at hand, potentially reducing the historical subject of Saint-Domingue into a present-day commodity, or allowing a kind of digital collaboration in a fictive version of---and perhaps then, a kind of narrative rewriting of---the slave revolts of Saint-Domingue. Yet despite the games' flaws, there are at once subversive and productive aspects of their construction that resist the gamer's attempt to master the art of slave rebellion.

The complexity of this issue can best be introduced by performing a brief visual analysis. The Assassin's Creed games include a recurring device in which, at certain moments of the gameplay, the player, having climbed to a designated \quotation{viewpoint} (a high vantage above the city or countryside), hits the button to achieve \quotation{synchronization}---meaning that, in the fictional world of the game, the player is on track to completion of the game's challenges. When the player hits the button to synchronize, the perspective pulls back from the protagonist, as if there were a camera floating in mid-air---a shot that in film could only be accomplished by a helicopter or drone---and does a 360-degree rotation of the character, revealing the panorama of the surrounding space. Accompanied by the swelling music of the game's soundtrack, these are beautiful vistas that exhibit the designers' artistry, the lavish detail of the game's architecture of the city, and the historical accuracy of the engineered setting. But thematically, such moments do double duty: first, they present the world that is laid out at the player's feet, as if to say, \quotation{Behold, this is your digital playground,} but, second, it pulls back from the playable character, both visually and ludically (a term that in the context of game studies means concerning the mode of play): after hitting the button, the player is rendered immobile until the panoramic tour is finished---such devices emphasize the construction of the text, the player's ensconcement in a digital world over which he, really, has little control. Such moments, then, in all the Assassin's Creed games, emphasize the duality of the player's control and lack of control.

These kinds of viewpoints, a hallmark of the game franchise, may take on a special significance when the context of the game's drama is a narrative of colonization or imperialism, as in both of the games I discuss here, as the panorama is a visual apparatus historically connected to conquest, dating back to its development in scenes of French battles and World Exhibitions.\footnote{Katie Hornstein connects official battle paintings with the nineteenth-century French justification of the Algerian conquest. See her \quotation{Horace Vernet's Capture of the Smalah: Reportage and Actuality in the Early French Illustrated Press,} in Jason E. Hill and Vanessa R. Schwartz, eds., {\em Getting the Picture: Visual Culture of the News} (London: Bloomsbury, 2015), 245--51.} Here I am thinking of Vivian Sobchak's discussion of the visual in terms of possession. \quotation{In his phenomenological description of human vision,} Sobchak writes, \quotation{Merleau-Ponty tells us, \quote{To see is to {\em have} {\em at a distance}.}}\footnote{Vivian Sobchak, \quotation{The Scene of the Screen: Envisioning Cinematic and Electronic \quote{Presence,}} in John Thornton Caldwell, ed., {\em Electronic Media and Technoculture} (New Brunswick, NJ: Rutgers University Press, 2000), 142. Sobchak quotes Maurice Merleau-Ponty, \quotation{Eye and Mind,} trans. Carleton Dallery, in James Edie, ed., {\em The Primacy of Perception} (Evanston, IL: Northwestern University Press, 1964), 166 (italics in original).} On the history of the panorama in World Exhibitions, Lieven de Cauter writes, \quotation{In the panorama the capitalist need for spatial expansion was, as it were, transferred to perception: the \quote{all-encompassing view} becomes an aesthetic experience of the first order. The viewer of a panorama takes enjoyment from a distant reality that can be possessed, that is always on the verge of being annexed or colonized.}\footnote{Lieven de Cauter, \quotation{The Panoramic Ecstasy: On World Exhibitions and the Disintegration of Experience,} {\em Theory, Culture, and Society} 10, no. 4 (1993): 3.} The panoramic view of colonial territory should always be thought in terms of the bodies and space that are claimed as a part of the act of viewing. But it is worth noting that these viewpoints actually operate like panoramas in reverse: rather than illustrating the space from the point of view of the character in a 360-degree rotation of his or her gaze, the \quotation{camera} swings out to show the protagonist atop his or her perch. In performing this perspective gyre, the protagonist is diminished in comparison to the space around---she or he becomes a part of the space to be claimed, and by aligning the gamer's perspective with a disembodied, imaginary camera, the viewer is reminded that it is the game designer, and not the gamer, who has the ultimate control.

\placefigure{Assassin's Creed: Liberation}{\externalfigure[images/lauro-vid1.png]}
\placefigure{Freedom Cry}{\externalfigure[images/lauro-vid2.png]}
In these viewpoint moments, then, we find a spectacle containing the various complications inherent in the games I wish to examine. Do they offer players, many of whom may well be students encountering this history for the first time, some type of ownership of these rebel slave characters, or of the history of slave revolt, or of the colonial spaces in which the narratives are set (here, French New Orleans and colonial Saint-Domingue), or do they merely tease the player with the possibility that she or he might, in playing the game, come to possess such narratives and histories but ultimately foreclose this annexation? I argue that it is in moments such as these, which disrupt the player's absorption in the game---as if to remind her or him of the fact of the engineered fiction---that we can locate a subversive gesture that ultimately denies the player ownership of the historical rebel slave's subject position. We will begin our study with a game that, set in French-controlled New Orleans, depicts various characters with ties to Saint-Domingue, and thus draws on Haitian history, before addressing a game that is explicitly set in the colony of Saint-Domingue during the prehistory of the Haitian Revolution.

{\em Assassin's Creed: Liberation}, a production of Ubisoft Sofia and Ubisoft Montreal, takes place in French New Orleans one hundred years before the American Civil War. In the game, the player incarnates a free woman of color, Aveline de Grandpré, who is the product of a female slave from Saint-Domingue and the slave's white master; Aveline is our titular \quotation{Assassin} and playable protagonist. Having lost her mother when she just a child, Aveline was raised by her white father and stepmother, wealthy merchants. The game begins with a flashback to the moment of her childhood trauma, in which the young Aveline sees slaves being auctioned on the block and then is separated from her mother in the marketplace. This \quotation{cutscene,} a cinematic interlude in which the player has no control, is in fact a dream, and Aveline awakens to find herself in the year 1763, where she is a young lady of privilege, dividing her time between elegant society and freedom-fighting missions as an assassin.

The Assassin's Creed franchise prides itself on its historical realism----its accurate representations of geographical spaces, use of languages other than the player's chosen setting, and inclusion of recognizable historical figures in supporting roles. As a testament to the designers' diligence, each game is prefaced with the following note: \quotation{Inspired by historical events and characters, this work of fiction was designed, developed, and produced by a multicultural team of various religions, faiths, and beliefs.}\footnote{Game scholar Adrienne Shaw has addressed problematic revision of history in another Assassin's Creed game in \quotation{The Tyranny of Realism: Historical Accuracy and Politics of Representation in {\em Assassin's Creed III},} {\em Loading} 9, no. 14 (2015): 4--24.} But aside from the purported historical realism of the in-game play, the Assassin's Creed franchise also features secret societies, time travel, and an alien race, though these are not my primary concern here. Suffice it to say that the intrigue framing all the games concerns a long-standing war between the Brotherhood of the Assassins and the secretive order of the Templars, who wage their battles throughout history by means of a device called the Animus, which allows characters to access their ancestors' memories and, in a virtual sense, to be transported back in time.

{\em Liberation}'s gameplay includes a series of missions, some of which are concerned with the treatment of slaves whom Aveline aids in their escape to the bayou, or the investigation of their disappearance and displacement to the colony of Chichen Itza in Mexico, but most of which are about battles for territorial control among corrupt colonial governors and conflict among smugglers in the swamplands. The theme of liberating the slaves is present here, though it seems subsidiary to Aveline's efforts to uncover the truth about her mother's involvement with the assassin syndicate and her subsequent disappearance. At the same time, somewhat mysteriously, Aveline becomes involved in her father's shipping business. In one mission, she buys out one of her father's merchant competitors in order to pay his slaves a living wage, thereby liberating them via capitalism. In short, despite Aveline's stated goals (which are, as she tells her mentor, Agaté, also from Saint-Domingue and a friend of her mother's, to \quotation{free the slaves, defeat {[}their{]} enemies, impose justice}), for most of the game there is an absence of a radical abolitionist message.

Troublingly, the game incorporates into its fiction the historical personage of François Makandal, maroon leader of Saint-Domingue, whose efforts at poisoning the white slaveholding community some historians consider to be the first chapter of the Haitian Revolution.\footnote{Laurent Dubois, for example, describes Makandal as a powerful precursor to the revolution and an iconic freedom fighter; see Laurent Dubois, {\em Avengers of the New World: The Story of the Haitian Revolution} (Cambridge, MA: Harvard University Press, 2004).} Aveline battles a bayou {\em houngan} who calls himself \quotation{François Mackandal}\footnote{I use two different spellings to make the distinction more apparent between the true Makandal and the imposter.} and is in league with the corrupt Spanish official Raphael Joachim de Ferrer, who seeks control of the colony. \quotation{The False Mackandal,} as the {\em houngan} character is called in the game (it is later revealed that his name is actually Baptiste) was formerly acquainted with both Agaté and Aveline's mother, Jeanne. Like the real Makandal, Baptiste appears to be missing an arm, though this is never explained; he simply wears a feathered wing over one shoulder. Baptiste turns out to be the disciple of the historical Makandal, as were both Agaté and Jeanne, when they worked as assassins for the brotherhood. As a part of Aveline's training, Agaté gives her Makandal's blowpipe, which she uses to dispense darts dipped in \quotation{fast poison} to kill enemies silently and from afar. This incorporation of the history of Haiti, the game's use of both Vodou and poison as strategies of slave subversion, might at first glance seem a successful homage to a proud legacy of resistance. For example, in a critical manner, there is a chapter called \quotation{The Power of Voodoo,} in which Agaté and Aveline exploit the Europeans' fears of the slaves' religion: Aveline thins out a patrol of soldiers crossing the bayou by using poison darts at points where Agaté has placed some sort of fetishes he refers to as \quotation{Voodoo signs}; Aveline's stealthy poisoning makes the soldiers think they are cursed and they flee in terror. Using poison to play up the belief that the slaves had a command of supernatural forces references one effective strategy of slave resistance, but for every moment that seems productively critical of the colonial hegemony, there is a misstep that reinforces stereotypes and diminishes the game's authenticity.\footnote{On the powers of Vodou and its uses as a tactic of intimidation during slave revolts and the revolution, see Joan {[}Colin{]} Dayan, {\em Haiti, History, and the Gods} (Berkeley: University of California Press, 1995).}

When Aveline seeks to kill the False Mackandal, the ally of her Spanish enemies, she must first reduce his forces; the acolytes with whom he surrounds himself are men of African descent who wear kilts and body paint similar to our ersatz prophet, including a feathered wing over one shoulder, creating a dramatic but inauthentic portrait of Vodou practitioners. At one point, the game's directions read, \quotation{Kill the thugs. Remain undetected,} in a use of the term {\em thug} that, although it may be appropriate here for its historical connection to religious assassins, has become increasingly problematic in recent years when applied to young black men, as these characters are. During the False Mackandal's dying confession, Aveline comes to understand her enemy's true identity and his relationship to the real Makandal. At this point, upon his death by her hand, Baptiste also discloses that he knows both Agaté and her mother, which he conveys by recognizing the locket she wears. If this were the extent of the historical Makandal's inclusion in the narrative, it might merely raise an eyebrow as a questionable attempt at preserving historical realism in the game, but there is an underlying layer that must be revealed.

\placefigure{Assassin's Creed: Liberation}{\externalfigure[images/lauro-fig1.png]}
As mentioned above, the Assassin's Creed franchise involves a storyline about a company, Abstergo Industries, that facilitates the transportation back in time by means of a machine called the Animus device. Most of the games include a frame narrative about a white male character named Desmond Miles who relives his ancestors' memories via the Animus. Scholars have made a point of noting that {\em Liberation} is the first game of the series in which Abstergo Industries offers its services to the general public. That is, in the world of the game, Aveline is {\em for sale}: her story is available not only to her ancestors but to anyone who will pay to play.\footnote{As Amanda Phillips and Soraya Murray note, it is indeed troubling that the first black protagonist depicted in the franchise is offered up as a commodity. See Amanda Phillips, \quotation{Nothing is True: Racial Hybridity, Manipulated Memory, and White Innocence in {\em Assassin's Creed III},} lecture, Society for Cinema and Media Studies (SCMS) 2016 conference, Atlanta; and Soraya Murray, \quotation{The Visual Poetics of Videogames,} lecture, 16 February 2016, Texas Tech.} Therefore, implicit in both {\em Liberation} and {\em Freedom Cry} (the latter also absents the frame narrative about Desmond Miles) is the idea that some other person, perhaps {\em you} as the game player, incarnate the playable protagonist. The player has a more direct channel to the protagonist absent the frame narrative, making the playable character seem inhabited by the player him- or herself, a device that here parallels the slave's appropriation by the master.

There is a subplot to {\em Liberation} in which a hacker called Erudito makes contact at various points in the narrative and offers \quotation{you} (the player incarnating our eighteenth-century heroine) revealing glimpses into the way the fictional foe Abstergo Industries has tampered with or redacted Aveline's storyline. In such moments, an electronic voiceover tells you to find a character called \quotation{Citizen E} and assassinate him; if you do so, you then replay a scene you have seen before, this time with modifications or extra lines that reveal \quotation{the truth} Abstergo presumably does not want you to know. The player may experience déjà vu, as previously seen cutscenes repeat, this time with some distortion, as if a visual input cable had come loose on the back of the television. Each of these moments reveals more dialogue than was previously seen and offers a supposedly more satisfying picture: for example, the third reveal replays a conversation between Aveline and Gérald Blanc, a business associate, would-be suitor, and sometime accomplice of our heroine, in which she expresses dissatisfaction that she cannot do more for the slaves: \quotation{A small gesture, hardly enough. I can offer them a wage, but what good is money without freedom?} Here she acknowledges that paying the slaves she frees does not undo the problems of the society at large. She also expresses feelings of powerlessness going up against the Templars, who will \quotation{never allow the slaves to be free.} It may be that it is only these references to the Templars that are meant to be key revelations, but it is interesting to note that Erudito's recoveries in this game consistently reveal more radical content, expressing the wider need for slave emancipation, for instance. We might say that such moments in the game are \quotation{meta,} tacit acknowledgements of the way the programmers themselves have tampered with history.\footnote{The resonances of this part of the game to our contemporary political climate, with its references to hacks, conspiracies, and alternative facts, is jarring, and scholars are currently working to unpack the political relevance of such moments. At SCMS's annual meeting in 2016 in Austin, Kimberly Bain presented an excellent paper on glitching and radical black activism, \quotation{Glitches/Black Bodies: \quote{Alright,} Digital Games, and Newly Imagined Emancipations,} in which she discussed such moments for their aesthetics of the glitch.} Yet there is a potential hazard embedded in this aspect of the game's narrative architecture and in the role the co-option of Haitian history and culture plays in this game, and one of the Erudito hacks speaks to this directly. Even as this game packages itself as progressive, and the Erudito hacks do reveal a more revolutionary narrative, the game's construction of its own alternative history is not without complications.

The game's first reference to its own manipulation of the history of Saint-Domingue is made in the governor's mansion, heard in the Templar Raphael Joachim de Ferrer's claim to his conspirator, the colonial French Governor Abbadie: \quotation{We'll ensure the errors of Santo Domingo are not repeated here.} At first it might seem that he is referencing the historical Spanish loss of part of the island to French control, formally acknowledged in 1697's Treaty of Ryswyck, but more likely this is a nod to a specific (fictional) instance related to the game's characters.\footnote{Signed on 20 September 1697, to settle the Nine Years' War between France and the alliance of Spain, England, the Holy Roman Empire, and the United Provinces, the Treaty of Ryswyck solidified French control of Tortuga and the western third of the island of Hispaniola, the French colony of Saint-Domingue.} It may even be read as a proleptic call-back to the game {\em Freedom Cry}, to which I will turn presently, made by Ubisoft later but taking place in Saint-Domingue in the year 1735, some thirty years {\em before} the action depicted in {\em Liberation}. Whatever this incident was that happened on the island, it is obviously important, for it is the subject of more than one of the cover-ups orchestrated by Abstergo and revealed to the player by Erudito.

Baptiste (the False Mackandal) admits to Aveline that he is not the real Makandal, but it is only by exposing the tampered-with history provided by Erudito, in the first hack of the game, that we discover what Baptiste's plan was: he tells de Ferrer, \quotation{The nobles of New Orleans shall perish by poison, and the slaves shall be avenged. My mentor's work will be complete.} Of course, coming as this revelation does, {\em after} Aveline has assassinated the False Mackandal, implies that she has obstructed his goals for direct, violent action on the slave-master class and thereby thwarted the historical Makandal's strategy. Aveline has unwittingly protected the status quo. In this moment, then, the separation between Aveline and the player is underscored, since the player necessarily knows far more than does Aveline.

If this tampering with Haiti's history offends, one might take comfort in the fact that one can play through the game without learning much more about the historical Makandal. But if the player decides to collect the thirty pages of Aveline's mother's diary that are scattered throughout New Orleans, a more complete (and destructive) picture emerges. The leaves of paper, like Erudito's hacks, clearly signify as histories that have to be uncovered.

In these pages, Jeanne gives us a portrait of her enslavement, of her \quotation{relationship} with the Master De Grandpré, Aveline's father, and of her involvement with the Brotherhood of Assassins. Most important for our purposes here, it is revealed that Jeanne was working in the resistance movement in Saint-Domingue with the true Makandal, but that she betrayed him when she refused to poison her kind mistress and the white children, setting a limit for what constitutes reasonable freedom fighting in the game. Jeanne, we learn, escaped Saint-Domingue after stealing an item precious to the brotherhood (presumably the locket Aveline has inherited, which is ultimately revealed as a key object in her quest), and she thereby impeded Makandal's efforts to emancipate the slaves of Saint-Domingue. Even, then, as the game purports to provide a heroic, emancipatory model, resistance is given limits: Aveline's mother attains freedom only when it is given to her by her lover-cum-slave-master, and in both Jeanne's story and, incidentally, in Aveline's assassination of Baptiste, white slaveholders are spared a grisly death by poisoning.

Further, the game implies that the historical slave rebel François Makandal was a member of the fictional Brotherhood of Assassins, a move that is a troubling co-option of historical slave resistance by the entertainment industry. But even stranger, in one of Erudito's hacks it may be implied that Makandal's poisonings were only ever a Templar plot gone awry: in an episode on Chichen Itza, de Ferrer reveals that they are having problems with runaways from this supposedly Edenic slave colony. Erudito reveals in a secret dialogue that de Ferrer had planned to \quotation{doctor the workers' ale} to keep them complacent, but he is reminded by his henchman that he previously has not had good luck with poison, which he acknowledges, saying, \quotation{Perhaps you are right, I don't want a repeat of the Mackandal debacle.} Does this imply that it was a poison engineered by the Templars and meant for the slaves that Mackandal planned to use against the masters (no matter which Makandal they mean to reference) or that they gave the historical Makandal the idea of using poison? Regardless, this edit seems to further rob Makandal of his legacy, whose prowess at developing subtle and sinister poisons is widely documented.\footnote{See, on this topic, C. L. R. James, {\em The Black Jacobins; Toussaint L'Ouverture and the San Domingo Revolution} (New York: Vintage, 1963); and Carolyn E. Fick, {\em The Making of Haiti: The Saint Domingue Revolution from Below} (Knoxville: University of Tennessee Press, 1990).}

The historical Makandal's revolutionary command is gelded by his incorporation into this game, but his power is also sublimated. In the game Makandal is thwarted by Jeanne, Aveline's mother, and his brand of emancipation demonized by her refusal to participate. The lord of poison is depicted explicitly as a failure, a man who died without accomplishing his mission. That the Templars seem to take credit even for his mastery of poison was, to me, exceptionally problematic. Historical revision is endemic to the Assassin's Creed series, often in productive ways, but these tamperings with the past take on dangerous resonances, given that it is a legacy of slave resistance that is nerfed.\footnote{Coming from the brand name Nerf, this gaming term signifies that something has been changed for the worse, most often indicating that it has been deprived of its power, as in the uselessness of a Nerf bat, made of soft foam, to inflict damage. See www.urbandictionary.com/define.php?term=nerfed.}

While the game's appropriation and undercutting of Makandal's legacy and its representation of Saint-Domingue more broadly are certainly troubling, it is impossible to say definitively whether {\em Liberation} is ultimately successful as a socially conscious project drawing attention to the history of black resistance. Addressing this issue would demand a detailed study of player response, whereas my concern here is to read the game as a text, especially insofar as it translates a narrative of historical resistance into an interactive medium. As a text that is played, the game may offer black resistance to a wide audience of gamers, a possibility that has been thoroughly investigated by scholars such as Kishonna Gray and Jordan Mazurek, who address the \quotation{problem of white gaming} in their forthcoming essay \quotation{Visualizing Blackness---Racializing Gaming: Social Inequalities in Virtual Gaming Communities.}\footnote{Kishonna L. Gray and Jordan Mazurek, \quotation{Visualizing Blackness---Racializing Gaming: Social Inequalities in Virtual Gaming Communities,} in Michelle Brown and Eamonn Carrabine, eds., {\em The Routledge International Handbook of Visual Criminology} (London: Routledge, forthcoming).} In the absence of a definitive answer to this question, I merely want to describe some of the ways that the complexity of participation in a narrative of slave resistance is treated within this particular game's structure, and to draw attention to the high stakes of allowing a player to incarnate the historical slave or slave-in-revolt within the context of an interactive game.

The visual and ludic operations of the game work in a complex manner to dramatize player identification with the character. More than just moving bodies across a digital landscape, the videogames I consider here exhibit, in some sense, the \quotation{second-person} conundrum, in that they compel the player to {\em become} the rebel slave, in much the same way that the inclusion of the second-person address in a literary text will draw the reader into the passage with the pronoun {\em you}.\footnote{Jill Walker, \quotation{Do You Think You're Part of This? Digital Texts and the Second-Person Address,} in Markku Eskelinen and Raine Koskimaa, eds. {\em CyberText Yearbook 2000}, Research Center for Contemporary Culture (Saarijärvi, Finland: Gummerus, 2001), 44.} Although these are not first person perspective games, and neither are they written in the second person, the player is nonetheless invited to identify with, to control, and to {\em be} a slave in revolt, which naturally has ramifications from the standpoint of \quotation{identity tourism,} as the contemporary player dons a historical role that is not properly his or her own.\footnote{Lisa Nakamura articulates this concept in the context of online personae in \quotation{Race in/for Cyberspace: Identity Tourism and Racial Passing on the Internet,} {\em Works and Days}, nos. 25--26 (1995): 181--93. Identity tourism on the Internet and in cyberspace allows a person to take on a gender or racial role other than their own.} It is crucial to note, however, that visually, both {\em Liberation} and {\em Freedom Cry}, discussed below, are third person perspective games, meaning that the viewer sees the player-character on screen at all times, as opposed to in a first person perspective game, in which the player sees through what is called in cinema studies \quotation{subjective camera,} as if she or he were inhabiting the body of the character. The player of the Assassin's Creed games may feel narratively, because of the frame narrative about the Animus machine, and ludically, because she mostly retains control over the body's movements on screen, that she has stepped into the protagonist's body, but visually she is given a somewhat conflicting vantage point. Therefore, some of the stickiness around identity occupation may be avoided by virtue of the fact that the \quotation{camera} is situated some feet above the playable character, thus holding open a space between the player and the character she or he pilots.\footnote{This might be contrasted with a game like 3D Realms's {\em Shadow Warrior}. See Jeffrey Ow, \quotation{The Revenge of the Yellowfaced Cyborg: The Rape of Digital Geishas and the Colonization of Cyber-Coolies in 3D Realms' {\em Shadow Warrior},} in Kolko, Nakamura, Rodman, eds. {\em Race in Cyberspace} (New York: Routledge, 2000), in which Ow describes Shadow Warrier as a first-person videogame that, he argues, explicitly identifies the ideal player as a \quotation{white male, middle class, cultural colonizer} (54) who plays the game as ninja Lo Wang.}

Because of the subject matter, too, we might link the occupation of the player-character in these games to discourse on the function of empathy in slave narratives. The issue of empathy has long been a fraught one in depictions of slavery, inasmuch as it redirects the slave's suffering for the \quotation{enjoyment} of others.\footnote{Discussions of empathy are often a major component of narratives about slavery and slave revolt, dating back to the sentimentalist tradition of both Harriet Beecher Stowe's {\em Uncle Tom's Cabin} and her {\em Dred}, which T. W. Higginson described as \quotation{dim and melodramatic} in \quotation{Nat Turner's Insurrection} (1861), in {\em Black Rebellion: Five Slave Revolts} (San Bernardino, CA: CreateSpace Independent Publishing Platform, 2013), 66.} As Saidiya Hartman has written, \quotation{It becomes clear that empathy is double-edged, for in making the other's suffering one's own, this suffering is occluded by the other's obliteration.}\footnote{Saidiya Hartman, {\em Scenes of Subjection: Terror, Slavery, and Self-Making in Nineteenth-Century America} (New York: Oxford University Press, 1997), 19.} In this narrative experience of the enslaved, the player is not concerned, really, with the slave himself but with the player's own suffering in imagining her- or himself in his place. The danger is that the historical person has been effaced by the act of empathizing in this manner.

This issue of empathic occupation becomes even more complicated when we turn from the subjugation of the slave to the slave's revolt as an experience to be recreated, in that it risks assimilating, commandeering, and even unwittingly subverting legacies of resistance. In addition to potentially fostering a problematic empathy, such games may allow for the assimilation of the radical emancipatory drive into the inventory of the entertainment economy. More broadly, and in a similar spirit to Hartman's caveat, I remain skeptical of the potential of what Alison Landsberg calls \quotation{prosthetic memory} to allow the consumer to take on ancestral memories that properly belong to other groups. Landsberg reads the potential of \quotation{prosthetically appropriating memories} as positive: it encourages empathy across social groups when, for example, a Gentile visits a Holocaust museum. \footnote{Alison Landsberg, {\em Prosthetic Memory: The Transformation of American Remembrance in the Age of Mass Culture} (New York: Columbia University Press, 2004), 34.} Though Landsberg's term fits the Assassin's Creed games, with their emphasis on reliving the memories of others, there is nonetheless a certain \quotation{empathy trouble} produced by the spectator's taking the place of the historical victim of oppression.\footnote{For an example of such \quotation{ironies of empathy,} see Alisha Gaines, \quotation{A Secondhand Kind of Terror,} in Claire Oberon Garcia, Vershawn Ashanti Young, and Charise Pimentel, eds., {\em From \quote{Uncle Tom's Cabin} to \quote{The Help}: Critical Perspectives on White-Authored Narratives of Black Life} (New York: Palgrave, 2004), 159--69.}

Videogames that capitalize on the history of slave resistance produce a potentially problematic act of occupation of the historically subjugated person, particularly when the player is invited to play {\em as} the slave or rebel slave. Although, as noted above, an exploration of the Assassin's Creed games in this respect would necessitate an in-depth reception study taking into account player demographics and responses, we can nevertheless acknowledge here that commoditizing the rebel slave as a playable protagonist risks \quotation{eating the {[}historical{]} Other,} to once again invoke hooks's description of cultural appropriation: \quotation{Where the desire is not to make the Other over in one's image but to {\em become} the Other.}\footnote{hooks, 25 (emphasis mine).} However, I would like to propose that these games' interruption of their own \quotation{immediacy,} that is, the media's invisibility as a medium, works against this problematic consumption.\footnote{Immediacy \quotation{dictates that the medium itself should disappear and leave us in the presence of the thing represented: sitting in the race car or standing on a mountaintop.} Jay David Bolter and Richard Grusin, {\em Remediation: Understanding New Media} (Location: MIT Press, 2000), 6.} Most simply stated, certain aspects of the game interrupt the immediacy of the gameplay. Absorption in the narrative is defied, and this serves to remind the gamer that she or he is merely a player. Like a defiance of continuity-editing in cinema---a jumpcut, for instance---these devices insist on the medium of the game and thereby interrupt the illusion that the player has become the character. This emphasis on the game's media is on display in {\em Liberation} within Aveline's presentation as a character with triple personae.

\placefigure{Assassin's Creed: Liberation}{\externalfigure[images/lauro-fig2.png]}
Aveline has three costumes that she dons: the assassin's garb, the slave's rags, and the lady's gown. Each costume comes with different skill sets and elicits different treatment by the community as Aveline walks through the city. As Soraya Murray has suggested, we see in the diversity of this avatar and her ability to maneuver through society an illustration of the diversity of roles taken on by people of color in New Orleans.\footnote{See Murray, \quotation{Visual Poetics of Video Games.}} That the manipulation of the same buttons on the controller will have different effects depending on whether the avatar is wearing the costume of the slave, the assassin, or the lady also undergirds the themes of slavery and freedom as they are manifested in the haptic dimensions of the game, since the player is aware of a range of her abilities and deficiencies. In the guise of the lady, for example, Aveline's mobility is somewhat restricted. On the other hand, she is swiftest and strongest in her assassin's garb. Aveline's polyvalence as a character, too, raises the specter of the collaborative effort of the slaves, free blacks, and wealthy mixed-race landowners to throw off the colonial yoke during the Haitian Revolution, as she alternately plays slave, assassin (and sometime Vodou initiate), and free lady of color. Significantly, Aveline's role as a fictional character in a deeply important history of resistance, which draws on her Saint-Dominguan heritage and is concretized by her possession of Vodou fetishes and poison blow darts in her arsenal, reveals the deeper {\em techne} of the game as rooted in role-playing. I want to suggest, then, that Aveline's triple nature also represents the player trying on different social and historical roles and serves as a reminder of the game's fiction and of the player's limitations within the game designer's algorithm.

At the same time that it depicts the diversity of colonial New Orleans, in part through Aveline's triple personae as signaling to the gamer's role play, so too {\em Liberation}'s use of language may remind the anglophone gamer of the medium and of her or his own limitations within an engineered digital space. Like the other games in the Assassin's Creed franchise, {\em Liberation} is multilingual, with important and even revelatory details spoken in French and left untranslated for the English speaker. For example, the comments of the passersby in New Orleans reveal how successfully or unsuccessfully Aveline is \quotation{passing} in her various guises.\footnote{ibid.} There may also be snippets of African dialects included here; there were several times slaves spoke in a language I could not comprehend. Flagging the use of language as important in the game, in one instance a smuggler named Roussillion tries to pronounce a Creole word---\quotation{houng . . . haung . . .}---and then opts for its (insufficient) translation, \quotation{witchdoctor.} The word he was looking for was, of course, {\em houngan}. This word and several other untranslated, unexplained concepts, such as {\em loas}, the Vodou divinities, point to the way language and culture makes communities, even within the society of gamers. The multilingualism of the game strikes me as an acknowledgement of the fact that players will have dissimilar experiences interacting with a game's narrative. That a game changes depending on who is playing it is intrinsic to the medium, since a player's skills and how much time she or he invests affect the end result, but the issue of player identity takes on specific complications in games about ownership of racial history and heritage, something the makers of the Assassin's Creed series, ostensibly about people reliving their ancestors' memories, surely took into consideration. While not understanding all the dialogue in the game may not obstruct the enjoyment of the game for all players, it is certainly possible that encountering nonplayer characters that speak exclusively in a foreign tongue might interrupt full immersion in the narrative and even be frustrating---as if the game means to withhold something. The multilingualism certainly lends to the realism of the game's world, yet it seems also to serve to remind the player of her or his incomplete absorption of the narrative.\footnote{One developer I spoke with said that the choice to leave language untranslated in the game was likely just a matter of budget concerns. Personal communication with Maurice Suckling.} Here one might read together a line from Stuart Hall's famous essay \quotation{Encoding, Decoding} with hooks's \quotation{Eating the Other}: \quotation{If no \quote{meaning} is taken, there can be no {\em consumption}.}\footnote{Stuart Hall, \quotation{Encoding, Decoding,} in Simon During, ed., {\em The Cultural Studies Reader} (New York: Routledge, 2007), 91 (emphasis mine).} When the player encounters indecipherable content, she or he is faced with an uncompleted circuit, in Hall's parlance, and the sense that someone else, a player who speaks that language or the character whom she or he is playing, would likely be able to complete that circuit. These uncompleted circuits act like aporia that productively disrupt the player's consummation of the character and the narrative. More concretely, language acts as a key element in the next game I discuss, which is set directly in colonial Saint-Domingue: {\em Freedom Cry}.

Whereas Haiti and its legacy of slave revolt is backdrop in {\em Liberation}, it is front and center in {\em Freedom Cry}, another game within the Assassin's Creed series. This is not a major release but what is called a DLC (downloadable content) and is an add-on for the game {\em Assassin's Creed 4:} {\em Black Flag}, about pirates in the Caribbean. In {\em Freedom Cry}, the player incarnates a character called Adewalé, a former slave from a sugarcane plantation in Trinidad (and quartermaster to protagonist Edward Kenway in the pirate adventure) who journeys to Saint-Domingue, frees slaves, and falls in league with a band of maroons, led by a man called Augustin, who are waging a war against the colonial {\em gouverneur}. Set in 1735, the game draws explicitly on the prehistory of the Haitian Revolution, especially with its detailed depiction of an emancipation campaign carried out by the maroons, alongside whom Adewalé fights.\footnote{The role of the maroons, escaped slaves who lived in communities in the mountains and often organized attacks, was pivotal to the success of the revolution. See, for example, Brown University's web resource on the Haitian Revolution: \useURL[url3][http://web.archive.org/web/20170904034539/http://library.brown.edu/haitihistory/2frt.html][][library.brown.edu/haitihistory/2frt.html]\from[url3].}

\placefigure{Freedom Cry}{\externalfigure[images/lauro-fig3.png]}
Even more pronouncedly than {\em Liberation}, {\em Freedom Cry} makes the liberation of slaves the player's chief labor in the game. Recruiting maroons along the way, Adewalé liberates plantations and slave ships, ultimately winning a ship for the cause and teaching Augustin to man it for the coming \quotation{revolution,} the objective of which is stated to be \quotation{maroon independence} but which also necessarily alludes to the revolution that would lead to the creation of Haiti. As {\em Liberation} co-opts the historical Makandal, {\em Freedom Cry} coopts the maroons, implying that the fictional Brotherhood of Assassins had a role in laying the groundwork that would lead to the Haitian Revolution. This game, then, does not wholly manage to avoid problematic commodification of black resistance and a troubling appropriation of the history of Saint-Domingue. Yet at the same time, the game's use of language ironically works to safeguard that history as something that is not available to anyone at the push of a button.

{\em Freedom Cry} incorporates dialogue not only in French, the language of the colonists present in Saint-Domingue, but also in Haitian Kreyol; there are even some phrases in a Trinidadian dialect. As in {\em Liberation}, foreign languages in gameplay are untranslated for the gamer, even if one enables subtitles. Sometimes the slaves whom Adewalé frees thank him in English, sometimes in French, and sometimes in Kreyol. In the cutscenes, the use of foreign languages is more spare, but with captions, one can read a translation in brackets (such as \quotation{{[}Bully!{]}}), though no transliteration of the word heard is given (in this case, \quotation{Baa John}) nor any indication of what language the character has spoken (here, Trinidadian Creole), lending a kind of mystery to the game, like a door through which one is not permitted to walk. In games about slave revolt, this trademark of the franchise works to particular effect in that such linguistic doors, to stick with the metaphor, allow only certain players to pass through---in many instances in {\em Freedom Cry}, for example, the very few players who speak Haitian Kreyol.

The first instance of Kreyol is heard from a slave whom Adewalé saves from mutilation by an overseer. After taking hold of a package during a battle with a Templar fleet, Adewalé shipwrecks just off the coast of Saint-Domingue. His first challenge comes when he stumbles upon a white overseer attacking a female slave and hissing threats (spoken only in French) that he will cut off her ears if she refuses to submit. After chasing down and killing the overseer, thus saving the slave woman, Adewalé asks the spared victim in a cutscene how to find \quotation{Bastienne Josèphe,} the intended recipient of the package he carries. The slave utters a phrase that I would transliterate as \quotation{Mwen memn pa konnen!} (\quotation{Me, I don't know!}), but this is translated in the closed captioning as \quotation{{[}\quote{Of course not!}{]}.} We come to find out that Josèphe is the madam of a brothel, so perhaps this testy response intimates that the slave woman is not in the habit of keeping company with prostitutes. There is a gap evident here between the woman's spoken words and the translation given that may more accurately depict the subtext than the dialogue itself. This distortion seems like a kind of synecdochal representation of what the games themselves do, in offering a version of history that is not wholly accurate and in creating moments that may divide the gamership into those who spot the strangeness of the narrative and those who cannot, much the way the hacker Erudito reveals certain elements only to the gamer who can catch Citizen E in {\em Liberation}.

\placefigure{Freedom Cry}{\externalfigure[images/lauro-vid3.png]}
The significance of the game's use of unidentified, untranslated language is suggested to me by the inclusion, early on in the game, of a password in a critical scene between Adewalé and Bastienne Josèphe. The latter turns out to be a wealthy and influential businesswoman with connections to both the leader of the maroon renegades and to the colonial governor. At the start of the game, Josèphe's presumed connection to the Templars causes Adewalé to distrust her, and he withholds the package he had planned to deliver. Josèphe is also wary of Adewalé, and so on first meeting him, she gives him a task to prove his trustworthiness. Josèphe teaches Adewalé to sing a snippet of song to the field slaves as a passcode to gain access and information as he makes his way to the maroon hideout. The song is a line from a traditional Haitian lullaby, \quotation{Si~vlé pa~dodo crab la va~manger,} which roughly translates to \quotation{If you don't go to sleep, the crab will eat you.} While it is true that recognizing that the song is a lullaby and knowing what it means do not necessarily further one's understanding of the game, the song's inclusion here suggests that the use of language in the game acts similarly to a kind of shibboleth, holding some gamers at bay and admitting others.

\placefigure{Freedom Cry}{\externalfigure[images/lauro-fig4.png]}
There are other instances in which the use of Kreyol explicitly divides the gamership into those who can understand the true meaning of certain exchanges and those who cannot. There are two separate occasions in which a freed slave and Madame Josèphe refer to Adewalé as {\em Blanc}, the Haitian word for stranger, which translates literally to \quotation{white person.} Although the gamer who reads the translation provided (if captions are enabled) may understand that this indicates that the characters do not fully trust Adewalé, he or she would likely miss the irony of their referring to the black Adewalé as {\em white} and all that this term implies in reference to the culture's sense of community, since it is a word freighted with history.

Playing further on the polysemy of language, when Adewalé first meets the maroon Augustin, he almost seems to speak in traditional Haitian aphorisms, saying things like \quotation{To give little is not cheap} (in Kreyol, \quotation{Bay piti pa chich}) and \quotation{White people say without doing; the good Lord does without saying} (\quotation{Blanc di san fe. Bondye fe san di}). As a means of telling Adewalé to be patient, for example, Augustin says, \quotation{Little by little the bird builds its nest.}\footnote{The game is not without flaws: at one point the actor playing Augustin Dieufort pronounces the word {\em lwa} (elsewhere spelled {\em loa}), meaning the Vodou pantheon of gods, as {\em iwa}, mistaking the lowercase {\em l} for a capital {\em I}.} This type of code talking may thrill the Haitian gamer, who recognizes her or his own culture represented in the game. Games journalist Evan Narcisse, for example, writes of the pleasure of playing {\em Freedom Cry}, a game that feels as if it was meant for him and in which he recognizes his mother's culture.\footnote{See Evan Narcisse, \quotation{A Game That Showed Me My Own History,} {\em Kotaku}, 19 December 2013, \useURL[url4][http://web.archive.org/web/20170904034549/http://kotaku.com/a-game-that-showed-me-my-own-black-history-1486643518][][kotaku.com/a-game-that-showed-me-my-own-black-history-1486643518]\from[url4]. In his review, Narcisse makes several good points about the game: its deft avoidance of Vodou, a topic that is too often used to ignorantly demonize the culture (and as I have already indicated, used to mixed effect in {\em Liberation}), and its significant use of Haitian music.} I nevertheless believe that it might be productive for gamers of non-Caribbean descent to feel explicitly excluded in encountering undecipherable moments in the game. For my part, there were times at which I could not tell if what I was hearing was French, French spoken with a highly African accent, modern Kreyol, or an older Creole.\footnote{For a discussion of the evolution of Creole in the Caribbean, see Jane Etienne's lecture \quotation{Bannzil Kreyol: La literature en langue creole de 17e siècle a nos jours} (\useURL[url5][http://web.archive.org/web/20170904034610/http://www.potomitan.info/bannzil/litterature.html][][www.potomitan.info/bannzil/litterature.html]\from[url5]), in which the author reminds us not only that Kreyol changed substantially, but that there were multiple creoles spoken by people belonging to different social levels in the seventeenth and eighteenth centuries.} For one scene in particular, in the maroon hideout, I enlisted the help of other French speakers and of native Kreyol-speaking Haitians and was unable to get any consensus on which language or languages were being spoken in the passages in question. Perhaps few gamers would be bothered by this opacity, but my own experience suggested that, in some capacity, the game was holding me at arm's length, reminding me of my own powerlessness within the digital system. This dynamic is, of course, a part of every videogame-playing experience. It is felt whenever a player fails to advance to the next level, is killed, or \quotation{desynchronized} and must begin again. And though untranslated language appears in diverse videogames, when the content is historical slave revolt, such formal devices act as a productive partition that calls to mind the broader semiotics of slave resistance, like a lamp in a window, the monkey-wrench quilt, a song about a drinking gourd---symbols to some but not legible by all.\footnote{These are examples of ciphers used in the communication of the Underground Railroad in the continental United States.} As such, moments like these may act like the statue to the maroon slave that stands in downtown Port-au-Prince.

\placefigure{Statue to the Maroon}{\externalfigure[images/lauro-fig5.jpg]}
The lone figure is depicted blowing the conch, the secret call to arms, one that the viewer, of course, cannot hear in this form. As such, it emphasizes the historical rebel slave's need to talk of revolution in ways that would allow only her or his compatriots to hear the message and that would not raise the alarm of the white colonial masters or others unsupportive of the cause. Though anyone might see the statue, the silence of the medium emphasizes that this is not an icon that belongs to everyone. It is perhaps tempting to read games such as {\em Liberation} and {\em Freedom Cry} as little more than a bit of easy shoot-'em-up entertainment---a genre which is always on the hunt for a way to make human victims of the protagonist's wrath unsympathetic (for example, as Nazis, or as zombies, or as Nazi zombies). And it is perhaps in this spirit that my students, who have encountered the history of Caribbean slave resistance only in this form, consume these entertainment commodities. But I have tried to show here that while the game series might seem to appropriate historical slave resistance by inviting the gamer to inhabit the role of the rebel emancipating slaves, it also defies the gamer's complete absorption in that narrative. Through the game's optics and in the use of untranslated language, an impassable boundary is established, reminding us that the game is only a game. Like the lullaby that warns the sleepy child of the hungry crab, these devices ensure that historical slave resistance cannot be devoured entirely, despite the consumable nature of these entertainment commodities. Therefore, even as (and perhaps, {\em because}) these games misrepresent history (as in the case of Makandal), they tacitly preserve a distinction between history and play: even with a game controller in hand, the history of slave resistance remains out of reach.

\thinrule

\subsection[reference={acknowledgments},
bookmark={Acknowledgments},
title={Acknowledgments}]

I benefitted from presenting parts of this article at the Southern American Studies Association 2015 conference, the Modern Language Association 2016 conference, the Society for Cinema and Media Studies 2016 conference; and the Rensselaer Polytechnic Institute's Experimental Media and Performing Arts Center in early 2017, as well as from the ensuing conversations with panel attendees and from discussing it informally with participants at the Caribbean Studies Association 2016 conference onboard a bus from Port-au-Prince to Jacmel. I also wish to thank the following people: the Dean of the College of Arts and Letters at the University of Tampa, Haig Mardirosian, for funding this project with a purchase of supplies and a budget for an undergraduate research assistant; David Restrepo, who was a fantastic and enthusiastic research assistant and who captured the video clips and still images included here; Renam Ocean, Oriane Delfosse, Joseph Mai, Vanessa Rukholm, and Laure-Alissa Jean, who consulted on some of the thornier language issues; Caleb Smith, whose comments on a different article also improved my thinking here; and my colleagues and article group at the University of Tampa, Kacy Tillman, Dan Dooghan, Joe Letter, and Daniel Wollenberg, who provided valuable input on drafts.

\thinrule

\page
\subsection{Sarah Juliet Lauro}

Sarah Juliet Lauro is an assistant professor of hemispheric literature at the University of Tampa. She is the author of many works that address the figure of the living-dead zombie, including the book {\em The Transatlantic Zombie: Slavery, Rebellion, and Living Death} (Rutgers University Press, 2015), and is the editor of the forthcoming collection {\em Zombie Theory: A Reader} (University of Minnesota Press, 2017). Her next book project turns from zombies as a figuration of slavery and slave revolt, which is her central interest in the monster, to literal commemorations of slave rebellion in literature, art, film, and digital culture.

\stopchapter
\stoptext