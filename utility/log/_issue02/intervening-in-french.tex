\setvariables[article][shortauthor={Broughton, Corlett-Rivera, Dize, de Gail}, date={July 2017}, issue={2}, DOI={doi:10.7916/D88K7NG5}]

\setupinteraction[title={Intervening in French: *A Colony in Crisis*, the Digital Humanities, and the French Classroom},author={Abby Broughton, Kelsey Corlett-Rivera, Nathan Dize, Brittany M. de Gail}, date={July 2017}, subtitle={Intervening in French}]
\environment env_journal


\starttext


\startchapter[title={Intervening in French: {\em A Colony in Crisis}, the Digital Humanities, and the French Classroom}
, marking={Intervening in French}
, bookmark={Intervening in French: *A Colony in Crisis*, the Digital Humanities, and the French Classroom}]


\startlines
{\bf
Abby Broughton
Kelsey Corlett-Rivera
Nathan Dize
Brittany M. de Gail
}
\stoplines


{\startnarrower\it This essay explore​s​ the use of {\em A Colony in Crisis: The Saint-Domingue Grain Crisis of 1789} in the French literature classroom and how it helps address gaps in digital humanities and French language pedagogy while interrogating the colonial positionality of the French Revolution's digital archive. In 2015, the Newberry Library received a Digitizing Hidden Special Collections and Archives grant from the Council on Library and Information Resources (CLIR) to digitize 30,000 French language pamphlets, a portion of which pertains to the period before, during, and after the French Revolution. As the digital archive of the French Revolution rapidly grows, the need to draw attention to the broader context of revolution in the French Empire--particularly in the Caribbean--has become even more urgent. ​One of the most effective ways of addressing the marginalization of the Caribbean in colonial archives is through pedagogical interventions and course design. While digital humanities pedagogy has become somewhat normalized in the anglophone literature classroom, the French language classroom has been slow to adapt to the use of digital tools and pedagogy beyond the introductory language course. \stopnarrower}

\blank[2*line]
\blackrule[width=\textwidth,height=.01pt]
\blank[2*line]

\subsection[reference={introduction},
bookmark={Introduction},
title={Introduction}]

In 2015, the Newberry Library received a Digitizing Hidden Special Collections and Archives grant from the Council on Library and Information Resources (CLIR) to digitize thirty thousand French-language pamphlets, a portion of which pertains to the periods before, during, and after the French Revolution.\footnote{See \quotation{About,} {\em Voices of the Revolution}, \useURL[url1][http://bit.ly/2lrnnEZ][][bit.ly/2lrnnEZ]\from[url1] (accessed 8 January 2017); and \quotation{2015 Funded Projects,} {\em Council on Library and Information Resources}, \useURL[url2][http://bit.ly/2kxQGXu][][bit.ly/2kxQGXu]\from[url2] (accessed 8 January 2017).} As the digital archive of the French Revolution rapidly grows, the need to draw attention to the broader context of revolution in the French Empire---particularly in the Caribbean---has become even more urgent.\footnote{There are digital archives of the Haitian Revolution, such as the Haiti Collection at the John Carter Brown Library, Brown University, \useURL[url3][http://bit.ly/2lDN1GX][][bit.ly/2lDN1GX]\from[url3], and other digital exhibitions of Haitian history, such as Malick Ghachem's 2014 \quotation{The Other Revolution: Haiti, 1789--1804,} at the John Carter Brown Library, \useURL[url4][http://bit.ly/2kxQtDs][][bit.ly/2kxQtDs]\from[url4] (accessed 9 January 2016), and Adam Silvia's 2014 \quotation{Haiti: An Island Luminous,} at the gallery at the Green Library, Florida International University, \useURL[url5][http://bit.ly/2lsyNZO][][bit.ly/2lsyNZO]\from[url5] (accessed 16 January 2017).} In her recent article \quotation{Refashioning Caribbean Literary Pedagogy in the Digital Age,} Leah Rosenberg rightfully states that the digital age \quotation{places significant responsibility on scholars to redress the marginalization of Caribbean literature and to ensure its future.}\footnote{Leah Rosenberg, \quotation{Refashioning Caribbean Literary Pedagogy in the Digital Age,} {\em Caribbean Quarterly} 62, nos. 3--4 (2016): 423.} Haitianist scholars are all too familiar with the processes of occlusion that ultimately led Michel-Rolph Trouillot to refer to Haiti's past, its revolution, and its place in the Age of Revolutions as \quotation{silenced} within the realm of Western historiography and Eurocentric events. In the two decades that have passed since Trouillot penned {\em Silencing the Past: Power and the Production of History}, scholars have helped place Haiti at the forefront of the global and Atlantic Age of Revolutions, which has provided critical insights into the history of slavery, human rights, and citizenship.\footnote{Michel-Rolph Trouillot, {\em Silencing the Past: Power and the Production of History} (Boston: Beacon, 1995). Recent titles include Laurent Dubois, \quotation{Atlantic Freedoms: Why Haiti Should Be at the Centre of the Age of Revolution,} {\em Aeon}, 7 November 2016, \useURL[url6][http://bit.ly/2kHtPUm][][bit.ly/2kHtPUm]\from[url6]; Elizabeth Maddock Dillon and Michael Drexler, eds., {\em The Haitian Revolution and the Early United States: Histories, Textualities, Geographies} (Philadelphia: University of Pennsylvania Press, 2016); James Alexander Dun, {\em Dangerous Neighbors: Making the Haitian Revolution in Early America} (Philadelphia: University of Pennsylvania Press, 2016); and Julia Gaffield, {\em Haitian Connections in the Atlantic World: Recognition after Revolution} (Chapel Hill: University of North Carolina Press, 2015).}

The digital turn, though, gives reason for pause, planning, and action to ensure that with the introduction of digital tools and pedagogy into university classrooms that the Haitian Revolution is not, once again, subsumed by the French Revolution.\footnote{The digital archive in French poses major questions relating to power and the production of history, especially since North American institutions possess and have digitized massive collections of French language materials. Whether or not these institutions are equipped to handle the burden of curating and critically engaging with these digital archives is often an afterthought in the race to digitize collections, which deserves further study.} As Rosenberg and others suggest, one of the most effective ways of addressing the marginalization of the Caribbean in colonial archives is through pedagogical interventions and course design. Digital scholars, including Luke Waltzer, Stephen Brier, and Kathleen Harris, highlight that pedagogy and course design are often at odds with the research-driven praxis of the digital humanities,\footnote{This seems to be shifting since the MLA Commons released the \quotation{keywords} project on the {\em Digital Pedagogy in the Humanities: Concepts, Models, and Experiments} site, which gives scholars and teachers a curated list of digital projects along with suggested implementation tactics. See Rebecca Frost Davis et al., \quotation{Digital Pedagogy in the Humanities \letterbar{} MLA Commons,} an open-access, born-digital publication that aggregates digital experiments by practitioners and presents pedagogical projects in their original forms: \useURL[url7][http://bit.ly/2lkwiaQ][][bit.ly/2lkwiaQ]\from[url7].} which renders pedagogy and assignment design \quotation{invisible labor.}\footnote{Luke Waltzer, \quotation{Digital Humanities and the \quote{Ugly Stepchildren} of American Higher Education,} in {\em Debates in the Digital Humanities} (Minneapolis: University of Minnesota Press, 2012), 342.} What is more, even though the use of digital tools and research practices have become somewhat normalized in the English-literature classrooms at the 300 and 400 levels, the undergraduate French-language classroom at the same levels has been slow to adopt the use of digital tools and pedagogy beyond the introductory language course.\footnote{Many use cases are found in the literature, such as Matthew K. Gold, ed., {\em Debates in the Digital Humanities} (Minneapolis: University of Minnesota Press, 2012); and Sarah H. Ficke, \quotation{From Text to Tags: The Digital Humanities in an Introductory Literature Course,} {\em CEA Critic} 76, no. 2 (2014): 200--210, doi:10.1353/cea.2014.0012; and anglophone digital humanities is well-represented at disciplinary conferences such as AHA (\useURL[url8][http://bit.ly/2kyac5V][][bit.ly/2kyac5V]\from[url8]) and MLA (\useURL[url9][http://bit.ly/2l360tM\%5D][][bit.ly/2l360tM{]}]\from[url9]). For an example of digital humanities pedagogy in the French-language classroom, see Virginia M. Scott, Cara L. Wilson, and Todd Hughes, \quotation{Digital Tasks for Advanced Learners: The Case of {\em La princesse de Clèves},} {\em French Review} 90, no. 4 (2017).} This is perhaps because of the positioning of the digital humanities in English rather than modern language departments, the periodized framing of modern language departments that ideally grants a faculty position to each century from the Middle Ages to the Renaissance to the twenty-first century, or the lack of synergy between departments in the twenty-first century university.\footnote{Matthew Kirschenbaum, \quotation{What Is the Digital Humanities and What's It Doing in English Departments?,} in Gold, {\em Debates in the Digital Humanities}, 3--15; Abby Broughton, Kelsey Corlett-Rivera, and Nathan H. Dize, \quotation{(De)Constructing Boundaries through the Digital Humanities: Collaborative Pedagogy and {\em A Colony in Crisis},} conference presentation, Caribbean Digital III, Maison Française, Columbia University, 2 December 2016 (\useURL[url10][http://bit.ly/2lduRsv][][bit.ly/2lduRsv]\from[url10]); Charles Forsdick, \quotation{What's French about French Studies?,} {\em Nottingham French Studies} 54, no. 3 (2015): 312--27.}

Given this environment, it was imperative for us, the authors of \useURL[url11][https://colonyincrisis.lib.umd.edu/][][{\em A Colony in Crisis}]\from[url11]{\em : The Saint-Domingue Grain Shortage of 1789}, to introduce this digital document reader in translation into the French-language classroom as a means of directly intervening at the intersection of digital humanities, French studies, and foreign-language pedagogy.\footnote{\quotation{The Project,} {\em A Colony in Crisis: The Saint-Domingue Grain Shortage of 1789} 9 July 2014, \useURL[url12][http://bit.ly/2mj7mhB][][bit.ly/2mj7mhB]\from[url12].} By partnering with Dr.~Sarah Benharrech at the University of Maryland, we brought {\em A Colony in Crisis} into the classroom in order to put the digital archive directly into the students' hands rather than relying on improved access alone. In what follows, we seek to elaborate the possibilities for {\em A Colony in Crisis} and other critical digital humanities tools, projects, and assignments in the undergraduate French classroom to expand the scope of French language, literary, and historical studies in our digital age. We will introduce {\em A Colony in Crisis} along with its long-term pedagogical intentions and aspirations. Next, we will present the first guided use of {\em A Colony in Crisis} in the classroom, offering an evaluation of the successes and failures of our pedagogical intervention and assignment design.\footnote{Abby Broughton, Kelsey Corlett-Rivera, and Nathan H. Dize, \quotation{Lessons from {\em A Colony in Crisis}: Collaborative Pedagogy and the Digital Humanities,} {\em Age of Revolutions}, 15 July 2016, \useURL[url13][http://bit.ly/2kWWjd3][][bit.ly/2kWWjd3]\from[url13].}

\subsection[reference={background-the-origins-of-a-colony-in-crisis},
bookmark={Background: The Origins of *A Colony in Crisis*},
title={Background: The Origins of {\em A Colony in Crisis}}]

The University of Maryland (UMD) Libraries' Special Collections holds nearly ten thousand French pamphlets, dating from the seventeenth through the twentieth centuries. Many of the French pamphlets were stored in boxes according to subject; perhaps they arrived in College Park in said boxes, or perhaps this system was implemented during earlier attempts to catalogue this collection. Regardless, while the pamphlets focusing on mainland France were organized thematically in uniform boxes, those focusing on the French colonies had been preserved in separate boxes, distinct from the rest of the collection.

\placefigure{French Pamphlet Storage Boxes}{\externalfigure[images/colonial-boxes.jpg]}
The colonial materials were not included in a finding aid developed with the help of the UMD French Department in the 1990s, whereas the items from France were broken down into very specific subjects (Series).

\placefigure{French Pamphlet Finding Aid}{\externalfigure[images/finding-aid.jpg]}
Despite these earlier efforts to improve access to the collection, awareness was still low around campus, especially when it came to the colonial materials. Cataloguing and digitization can increase awareness of such materials by making them accessible to search engines and browsing from remote locations. With that in mind, a team of library staff led by the French subject specialist librarian, Kelsey Corlett-Rivera, began to plan out a project to facilitate online access. The team, along with French faculty members, obtained funding from UMD's College of Arts and Humanities to hire French graduate and undergraduate students to inventory the pamphlets, which would allow them to be fully cataloged and digitized through existing workflows, overcoming a lack of French-speaking library staff. The student workers were able to inventory more than four thousand pamphlets, leading to the digitization of nearly one thousand through the \useURL[url14][http://bit.ly/2lsDToP][][{\em Revealing la Révolution}]\from[url14]initiative.\footnote{\quotation{Revealing la Révolution,} Special Collections and University Archives at the University of Maryland Libraries, \useURL[url15][http://bit.ly/2lsDToP][][bit.ly/2lsDToP]\from[url15].} While one of the project's original goals was to encourage archival research by the UMD undergraduate community, improved access alone did not notably increase online usage.\footnote{The digitized pamphlets are hosted by the Internet Archive, which gathers usage statistics as described in \quotation{Frequently Asked Questions,} \useURL[url16][http://bit.ly/2qey3K7][][bit.ly/2qey3K7]\from[url16].} The team employed various strategies to increase awareness, such as press releases, and also published regular blog posts written by team members and students, several of which featured particularly interesting pamphlets.\footnote{On {\em Special Collections and University Archives at UMD} (blog), see Nathan Dize, Danica Hawkins, and Annie Rehill, \quotation{Revealing la Révolution: An Update from the Trenches,} 28 March 2013, \useURL[url17][http://bit.ly/2ocfWjN][][bit.ly/2ocfWjN]\from[url17]; Nathan Dize and Danica Hawkins, \quotation{French Pamphlets, Education, Thermometers, and Goodbyes,} 17 May 2013, \useURL[url18][http://bit.ly/2pevWW6][][bit.ly/2pevWW6]\from[url18]; and Marie Laure Flamer, \quotation{Agriculture, Illustrations and Prophecies,} 10 March 2014, \useURL[url19][http://bit.ly/2peNfqa][][bit.ly/2peNfqa]\from[url19].} Unfortunately, those publicity efforts designed to spark interest did not address two of the underlying causes for low uptake by students: many UMD undergraduates studying Revolutionary France did not speak French, and so while they had improved access, were still not well-served by an online version of an original French pamphlet; and many of those students with the necessary French-language skills were perhaps not as familiar with the historical context behind these pamphlets that frequently focused on very specific issues (such as customs regulations at the port of Marseille).\footnote{Ralph Bauer, e-mail message to College of Arts and Humanities, supporting initial funding request, 4 November 2012; Etienne-David Meynier de Salinelles, {\em Rapport fait a l'Assemblée nationale au nom du Comité d'agriculture et de commerce, sur le régime à donner au port et au territoire de Marseille, quant aux droits de douane} (Paris: De l'Imprimerie nationale, 1791).}

In order to address these barriers to meaningful use, the {\em Revealing la Révolution} team envisioned a pedagogical project aimed at providing a ready-made resource for faculty to implement colonial pamphlets in their undergraduate classrooms, which would also increase students' exposure to primary sources and archival research. In focusing on pedagogical uses of the French pamphlets, the project team identified three parallel, but independent, goals: assisting UMD instructors by contextualizing the pamphlets for classroom implementation and use; supporting the UMD Libraries' goals of educational outreach within the institution; and gesturing toward partnerships with teachers and institutions outside of UMD by providing an easily accessible aggregated resource that could serve as a model for further collaboration.

Nathan Dize, who worked on {\em Revealing la Révolution} as an undergraduate student in 2013, began his graduate coursework in UMD's Department of French and Italian. As his interest in Caribbean literature grew, he considered how to further curate the colonial pamphlets in UMD's Special Collections. The project team approached Dize to work on the aforementioned pedagogical project, and he suggested a digital document reader in translation, similar to the book {\em Slave Revolution in the Caribbean}, a resource he himself had used as an undergraduate.\footnote{Laurent Dubois and John D. Garrigus, {\em Slave Revolution in the Caribbean, 1789--1804: A Brief History with Documents}, 1st ed. (Basingstoke: Bedford/St.~Martin's, 2006).} Dize, along with fellow French graduate student Abby Broughton and librarian Kelsey Corlett-Rivera, began work on {\em A Colony in Crisis}. By creating this document reader in an online environment, they could provide students with open access to both historical contexts and excerpts of original primary sources---in this case, translations of colonial French pamphlets identified during {\em Revealing la Révolution}.

Faced with limited time and funds, and striving to stay within the undergraduate attention span, the {\em Colony in Crisis} team consulted with Dr.~Jennifer Guiliano, then an assistant director at the Maryland Institute for Technology in the Humanities (MITH), to establish specific parameters for the document reader. Her experience teaching undergraduate students and developing digital resources was invaluable as she worked with us to set concrete, achievable goals: each translation should be approximately one thousand words and ten to twelve pamphlets would be sufficient. While our model, {\em Slave Revolution in the Caribbean}, includes forty-five documents, ours could not be compared to a multiyear effort by leaders in the field with publisher support. By focusing on a particular time period, place, or incident, we could more effectively narrow down our list of over four thousand possible pamphlets. We toyed with the idea of focusing on a pivotal year, such as 1789 (the storming of the Bastille) or 1791 (the beginning of the Haitian Revolution), both seemingly logical starting points. But we had hundreds of pamphlets from those years, many of which had been curated in online exhibitions and had received considerable attention in literature.\footnote{The John Carter Brown Library's Remember Haiti project (\useURL[url20][http://bit.ly/2kNlXFH][][bit.ly/2kNlXFH]\from[url20]) is an example of one of these online exhibits. Translations of crucial documents relating to the Society of the Friends of Blacks ({\em Société des amis des noirs}; \useURL[url21][http://bit.ly/2kIYMMU][][bit.ly/2kIYMMU]\from[url21]) and others digitized at UMD appear in Dubois and Garrigus, {\em Slave Revolution in the Caribbean}.} Focusing our limited number of pamphlets around an event that had not been well-documented would add valuable content to the eighteenth-century digital archive and would also reduce the likelihood that the historical snapshots presented in each document would come across as disconnected, requiring extensive explanation to lead the reader from one text to the next. Dize, having spent the most time with these documents throughout his undergraduate and graduate career, proposed a single incident discussed in the pamphlets: the Saint-Domingue grain shortage of 1789, in which the importation of grain and flour was heavily disputed between the colony and mainland France. Keeping the reader focused on a short time period would allow for easier connections between pamphlets within the brief historical introduction presented at the beginning of each translated excerpt.

Upon first reading the twelve chosen pamphlets that became Issue 1.0, the {\em Colony in Crisis} team focused on the plight of the inhabitants of Saint-Domingue. These documents, presenting colonial voices accusing mainland France of neglect leading to a famine, and counterarguments insisting the famine was nothing more than a deception, appeared to be proof that both locals and slaves were on the brink of starvation because France would not allow them to break the exclusive laws to obtain grain from the United States of America or other sources.\footnote{\quotation{Succinct Response from the Saint-Domingue Deputies, Regarding the Merchants of Sea Ports, Distributed in the Offices of the National Assembly, October 9, 1789,} {\em A Colony in Crisis}, \useURL[url22][http://bit.ly/2mjyOfd][][bit.ly/2mjyOfd]\from[url22].} Consequently, the website was titled {\em A Colony in Crisis}. However, as the team did more research and began receiving feedback from advisory board members, it became apparent that perhaps the situation was not quite as dire as initially thought.\footnote{The following scholars have served on the advisory board for {\em A Colony in Crisis}: Sarah Benharrech, Manuel Covo, Marlene Daut, Carolyn Fick, John Garrigus, David Geggus, Jennifer Guiliano, Erica Johnson, Mariana Past, Alyssa Sepinwall, Chelsea Stieber, and Gina Athena Ulysse. For more details, visit the board of advisors page (\useURL[url23][http://bit.ly/2m5D4mi][][bit.ly/2m5D4mi]\from[url23]).} Slaves, the majority of Saint-Domingue's population in 1789, consumed little to no flour in their regular diets, so really only a small percentage of upper-class French citizens likely felt the effects of the grain shortage.\footnote{Joseph Horan, \quotation{The Colonial Famine Plot: Slavery, Free Trade, and Empire in the French Atlantic, 1763--1791,} {\em International Review of Social History} 55, supplement S18 (2010): 103--21, doi:10.1017/S0020859010000519; Bertie Mandelblatt, \quotation{How Feeding Slaves Shaped the French Atlantic: Mercantilism and the Crisis of Food Provisioning in the Franco-Caribbean during the Seventeenth and Eighteenth Centuries,} in {\em The Political Economy of Empire in the Early Modern World} (Houndmills: Palgrave Macmillan, 2013), 192--220; François Barbé-Marbois and France, Assemblée Nationale Constituante (1789--91), {\em Memoire et observations du Sieur Barbé de Marbois, intendant des Isles-sous-le-vent en 1786, 1787, 1788, et 1789: Sur une dénonciation signée par treize de MM. les députés de Saint-Domingue, et faite à l'Assemblée nationale qa nom d'un des trois comités de la colonie} (Paris: Chez Knapen et fils, libraires-imp. de la Cour des Aides, pont S. Michel, 1790), 35 (\useURL[url24][http://bit.ly/2lkJRHp][][bit.ly/2lkJRHp]\from[url24]).} In reality, the rhetoric-laden diatribes against France's policies were geared toward lifting the barrier presented by the exclusive laws restricting trade with countries other than France so that Saint-Domingue's merchants could trade freely, thereby seeing significantly increased profits.\footnote{Anne Eller, \quotation{A Review of {\em A Colony in Crisis},} {\em sx archipelagos}, no. 1 (2016), doi:10.7916/D8833S3J.} While the authors came to terms with the interpretative framing employed by the French and colonial administration, it is not to say that {\em A Colony in Crisis} failed to recognize the inhumanity of the plantation, slavery, or the denial of full citizenship rights to the {\em gens de couleur libre} (free people of color) in Saint-Domingue. Issue 3.0, for example, addresses the systemic mistreatment of both enslaved and \quotation{free} people of color surrounding the grain shortage of 1789.\footnote{See Marlene L. Daut, \quotation{Issue 3.0: Introduction,} {\em A Colony in Crisis}, 31 October 2016, \useURL[url25][http://bit.ly/2mi2Uk3][][bit.ly/2mi2Uk3]\from[url25].} While the project team used the previously discussed criteria in identifying and expanding on the initial topic, the major impetus behind the many years of work to improve access to UMD's French pamphlets, this project included, was to facilitate classroom use of these rare materials. Therefore, the effectiveness of our work could be measured only through implementation in the undergraduate classroom. Thus we partnered with Dr.~Sarah Benharrech in order to integrate {\em A Colony in Crisis: The Saint-Domingue Grain Crisis of 1789} into \quotation{Riots, Rebellions, and Revolution: Cultures of Dissent,} a Fall 2015 upper-level French undergraduate course on eighteenth-century French civilization

\subsection[reference={saint-domingue-french-language-pedagogy-and-the-digital-humanities},
bookmark={Saint-Domingue, French Language Pedagogy, and the Digital Humanities},
title={Saint-Domingue, French Language Pedagogy, and the Digital Humanities}]

The French-language classroom, whether upper-level or introductory, makes use of digital tools to provide students with the necessary linguistic reinforcement in the classroom environment as well as at home. These are often textbook companion sites that feature drills, glossaries, or interactive interfaces that help to prolong in-class discussions or expand assignments related to cultural materials. However, in the upper-level French civilization or literature classroom, students begin to grow into proficient users of French at the same time they begin to hone their skills in humanities-based research practices. For this reason, students of French in upper-level courses are poised to benefit from the introduction of digital humanities research methods and models so that they can begin to mobilize both their language abilities and their research skills.

The digital humanities can serve as a vast canvas for French-language students to create their own work, define their own research interests, and explore the boundaries of their interests in francophone culture, broadly defined. The digital space has a wealth of benefits for the French teacher as well because it broadens the source base from which instructors can draw---particularly as it relates to the digital archive of the eighteenth century in French. {\em A Colony in Crisis} attempts to provide students with curated historical content ready for classroom use, but the site also introduces a replicable model for digital humanities research in French for students, teachers, and scholars alike. Thus we argue that a thematic approach to teaching the Haitian Revolution in concert with the French Revolution in the French-language classroom will allow students to grapple with the history of slavery, empire, and revolution in a more global sense.\footnote{Christie McDonald and Susan Suleiman, eds., {\em French Global: A New Approach to Literary History} (New York: Columbia University Press, 2011). This influential volume builds on developments in French and francophone studies also found in Charles Forsdick, \quotation{Mobilising French Studies,} {\em Australian Journal of French Studies} 51, nos. 2--3 (2014): 250--68; and Alec G. Hargreaves et al., {\em Transnational French Studies: Postcolonialism and Littérature-Monde} (Liverpool: Liverpool University Press, 2012). This is also part of a larger frame of transnational literary and cultural studies; see Shu-mei Shih and Francoise Lionnet, eds., {\em Minor Transnationalism} (Durham, NC: Duke University Press, 2005).} Christie McDonald and Susan Suleiman suggest that by reading and interacting with literature in French in the world, scholars can begin to \quotation{challenge the notion of a seamless unity between French as language, as literature, and French as a nation.}\footnote{McDonald and Suleiman, {\em French Global}, xix.} As part of this intervention into the teaching of the Haitian and French Revolutions, we contend that the digital humanities and the use of digital projects offer ways for eighteenth-century French scholars to pivot from the Hexagon to a broader francophone context.\footnote{France is often referred to as the \quotation{hexagon,} referring to the physical shape of the country. In French studies, this term is employed mainly to differentiate between metropolitan France and France's overseas departments in the Caribbean, the Indian Ocean, and the Pacific as well as regions of the world that also speak French. It is employed linguistically (hexagonal French) to distinguish between the French spoken in France versus, for instance, Canadian French or Québécois.}

In the past decade, historians have benefitted greatly from using comparative methods for teaching the Haitian and French Revolutions in the US-history classroom, providing students with a wider scope of history as well as encouraging nuanced discussions about race in the Americas. As John Garrigus argues in his article \quotation{White Jacobins/Black Jacobins: Bringing the Haitian and French Revolutions Together in the Classroom,} \quotation{The case of Saint-Domingue/Haiti raises a topic of great importance for U.S. students---the interaction of racial and national identities.} By teaching the Haitian Revolution alongside the French Revolution, Garrigus contends that we are allowed to ask \quotation{at what point were events no longer defined in Paris but in the Caribbean?}\footnote{John D. Garrigus, \quotation{White Jacobins/Black Jacobins: Bringing the Haitian and French Revolutions Together in the Classroom,} {\em French Historical Studies} 23, no. 2 (2000): 269, 260.} This decentering of the metropole in favor of the colonial periphery shifts explorations of identity, authoring, and influence away from traditionally French roots and towards a greater understanding of how the Haitian Revolution shaped the early history of the Americas. In the anglophone history classroom, these questions can be addressed in a number of ways. First, document readers such as {\em The Haitian Revolution: A Documentary History}; {\em Slave Revolution in the Caribbean, 1789--1804: A Brief History with Documents}; or {\em Race and the Enlightenment: A Reader} help draw students to historical realities through a layer of curation---specific historical context allowing readers to enter into an isolated and excerpted document---giving nonspecialist readers the tools to reconstruct events. Second, the development and persistence of the fields of Haitian revolutionary studies and Atlantic history have allowed scholars to identify the connections between Haiti and the broader Atlantic world in the Age of Revolutions and beyond.\footnote{Just a few examples of recent studies in the field of history include Elizabeth Maddock Dillon and Michael Drexler, eds., {\em The Haitian Revolution and the Early United States: Histories, Textualities, Geographies} (Philadelphia: University of Pennsylvania Press, 2016); James Alexander Dun, {\em Dangerous Neighbors: Making the Haitian Revolution in Early America} (Philadelphia: University of Pennsylvania Press, 2016); Caitlin Fitz, {\em Our Sister Republics: The United States in an Age of American Revolutions} (New York: W. W. Norton, 2016); Malick W. Ghachem, {\em The Old Regime and the Haitian Revolution} (Cambridge: Cambridge University Press, 2012); Ashli White, {\em Encountering Revolution: Haiti and the Making of the Early Republic} (Baltimore: Johns Hopkins University Press, 2012).} By introducing these approaches into a French-language context, the need to focus on the francophone world beyond France becomes even more apparent.

\placefigure{Historical Introduction}{\externalfigure[images/intro.jpg]}
The historical context for each document on the site is currently available only in English. However, each translation is juxtaposed with images of its French-language original, allowing teachers of French to control the implementation of the sources by making use of the contextual précis and bibliographical notes embedded in each translation page.

\placefigure{French Original and Translated Excerpt}{\externalfigure[images/originalandtranslation.jpg]}
Teachers of French can also easily trace the original pamphlets back to important digital collections of French-language pamphlets in the John Carter Brown Library's \useURL[url26][http://bit.ly/2lDN1GX][][Haiti Collection]\from[url26] and the University of Maryland, College Park's \useURL[url27][http://bit.ly/2mjtbO7][][French Pamphlet Collection]\from[url27], in the Internet Archive, and at the \useURL[url28][http://bit.ly/2l3z2JF][][Bibliothèque Numérique Caraïbe Amazonie Plateau des Guyanes]\from[url28] in order to guide their students in conducting original archival research in French.\footnote{See \quotation{John Carter Brown Library---Haiti Collection,} \useURL[url29][http://bit.ly/2lDN1GX][][bit.ly/2lDN1GX]\from[url29]; \quotation{University of Maryland, College Park---French Pamphlet Collection,} \useURL[url30][http://bit.ly/2mjtbO7][][bit.ly/2mjtbO7]\from[url30]; and “Manioc: Bibliothèque Numérique Caraïbe Amazonie Plateau des Guyanes, \useURL[url31][http://bit.ly/2l3z2JF][][bit.ly/2l3z2JF]\from[url31].} By providing links to the archived originals, {\em A Colony in Crisis} invites students, teachers, and scholars to replicate the type of document curation featured on the site so that other episodes such as the so-called \quotation{Saint-Domingue grain shortage} can be recovered from the archive of the early Caribbean. Replicating the curatorial aspect of {\em A Colony in Crisis} would implement a number of skills taught in the foreign-language classroom, such as translation, citation styles, and secondary-source reading, while also adding to the wealth of digital history projects on the web. For undergraduate students as well as graduate students and junior scholars, digital projects also often serve as crucial publications that prove useful when pursuing further professional opportunities or employment. For teachers, implementing these aspects of humanistic inquiry and digital literacy into the classroom provides students with critical thinking and the creativity to work within the digital space.

In Caribbean studies, a number of early digital projects highlight the importance of research and publishing within the digital realm that are ready for multilingual classroom settings.\footnote{These projects are often oriented toward pedagogy and predate the Modern Language Association's recent efforts for usable indices of digital humanities projects ready for classrooms. See Davis et al., \quotation{Digital Pedagogy in the Humanities.}} The \useURL[url32][http://bit.ly/2lOWOdL][][Digital Library of the Caribbean]\from[url32], the John Carter Brown Library's \useURL[url33][http://bit.ly/2lDN1GX][][Haiti Collection]\from[url33], and \useURL[url34][http://bit.ly/2lsyNZO][][Haiti: An Island Luminous]\from[url34] are just three examples of digital spaces where scholars, librarians, and digital humanists have collaborated and compiled digital projects ready for use in the French-language classroom.\footnote{For instance, the Digital Library of the Caribbean (DLoC) archives lesson plans, such as Erin Zavitz's work on pedagogical approaches to teaching Émeric Bergeaud's {\em Stella} and Pierre Faubert's {\em Ogé, ou Le préjugé de couleur}. Each of these texts, along with Zavitz's guide, are available in DLoC; see Erin Zavitz, \quotation{Literary Representations of the Haitian Revolution: A Teaching Resource for Pierre Faubert's {\em Ogé, ou Le préjugé de couleur} and Émeric Bergeaud's {\em Stella},} {\em Digital Library of the Caribbean}, 2012, \useURL[url35][http://bit.ly/2moZ9bS][][bit.ly/2moZ9bS]\from[url35].} These sites offer pedagogical suggestions for implementation or provide historical and cultural context that can be easily adapted to francophone classrooms. Furthermore, blog spaces such as \useURL[url36][http://bit.ly/2kS7kwY][][{\em Black Perspectives}]\from[url36], \useURL[url37][http://bit.ly/2lg0275][][{\em Age of Revolutions---A HistorioBLOG}]\from[url37], and \useURL[url38][http://bit.ly/2lTHSeA][][{\em The Junto: A Group Blog on Early American History}]\from[url38] often offer pedagogical approaches to teaching comparative perspectives in the humanities in our digital age.\footnote{See, for example, Brandon R. Byrd and Nathan H. Dize, \quotation{Black Lives in {\em A Colony in Crisis}: An Interview with Nathan H. Dize---AAIHS,} {\em Black Perspectives}, 13 November 2016, \useURL[url39][http://bit.ly/2luTLGD][][bit.ly/2luTLGD]\from[url39]; Broughton, Corlett-Rivera, and Dize, \quotation{Lessons from {\em A Colony in Crisis},} \useURL[url40][http://bit.ly/2kWWjd3][][bit.ly/2kWWjd3]\from[url40]; and Jessica Parr, \quotation{Reflecting on Digital History,} {\em The Junto}, 26 January 2017, \useURL[url41][http://bit.ly/2lJDhuF][][bit.ly/2lJDhuF]\from[url41].} In a blog post for {\em Age of Revolutions}, Haitian historian Erin Zavitz delves into the use and implementation of timeline software in the US-history survey course.\footnote{Erin Zavitz, \quotation{Revolutions in the Classroom: Digital Humanities and the U.S. History Survey,} {\em Age of Revolutions}, 13 June 2016, \useURL[url42][http://bit.ly/2mbFuQK][][bit.ly/2mbFuQK]\from[url42].} The above-listed cases are just some of the many ways the digital humanities can, as Leah Rosenberg suggests, avoid and contest the marginalization of (early) Caribbean literature and history.\footnote{Rosenberg, \quotation{Refashioning Caribbean Literary Pedagogy.}} More important, this contestation does not have to take place solely in the Caribbean history or literature classroom; rather, it can happen via a comparative and thematic approach. The case study below proves that other disciplines, in this instance French departments, are equipped to make the multivalent digital and Caribbean leap.

\subsection[reference={case-study-a-colony-in-crisis-in-the-undergraduate-classroom},
bookmark={Case Study: *A Colony in Crisis* in the Undergraduate Classroom},
title={Case Study: {\em A Colony in Crisis} in the Undergraduate Classroom}]

In the fall of 2015, the authors behind {\em A Colony in Crisis} first implemented the document reader in the undergraduate classroom, working with Dr.~Sarah Benharrech, a member of the site's board of advisors. This case study may be considered both a classroom intervention and a pedagogical experiment, since it provides invaluable insight into how current students can and cannot access {\em A Colony in Crisis} and, more broadly, primary sources in general. The students who participated in our study are privileged to be both French and English speakers who were able to use not only the Internet in their research but all of UMD's library holdings and resources in their work. Right away, we realized we needed to use the intervention to teach students about \quotation{authoritative sources,} since the undergraduates we encountered were unequipped to discern dated and inaccurate sources from valuable ones. This discovery is especially significant when it concerns the perpetuation of colonial narratives in a time when scholars are attempting to reintegrate stories of those disenfranchised by colonialism and subsequent dominant discourses. What is posted on the site reflects this realization, since most students do cite modern sources from respected historians, though this proved not always to be the case. It is our duty as educators to equip students with the tools they need to pursue informed scholarship, since even upper-level students still need reinforcement regarding citations and research methods. This is a task for instructors, professors and librarians alike, who must seize opportunities to work together to educate today's undergraduates.

In the spirit of collaborative pedagogy, Dr.~Behnarrech and the authors of {\em A Colony in Crisis} worked with the undergraduate French-literature course \quotation{FREN 449A: Studies in Eighteenth-Century French Literature and Culture; Riots, Rebellions, and Revolution: Cultures of Dissent} to produce the section \quotation{Background Notes,} which is featured in an independent tab on the website alongside the translations themselves. As we look back on this first pedagogical intervention, we are now faced with evaluating and reimagining how our digital pamphlet reader can and will function productively within the framework of a university syllabus. Our pedagogical goals originally focused on introducing a new generation of scholars to reading primary archival texts---the preferred evidentiary basis of historical analysis. As a means of introducing students to the history of Saint-Domingue and the Haitian Revolution, Dr.~Benharrech and the site's team worked together to create the parameters for her students. Once we conceived the assignment, we listed it in the syllabus as one of the main projects of the semester, with the goal of writing biographical columns to \quotation{enrich the website.}\footnote{Sarah Benharrach, syllabus for \quotation{FREN 449A: Studies in Eighteenth-Century French Literature and Culture; Riots, Rebellions, and Revolution: Cultures of Dissent,} Fall 2015.} Dr.~Benharrech introduced this collaborative project halfway through the semester in week six, corresponding to the unit on the Haitian Revolution. Two weeks prior, students had studied the 1775 Flour War ({\em la guerre des farines}) and popular revolts before focusing on the French Revolution (1789) for three class periods. This organization provided students with the foundational knowledge necessary to set the scene for their foray into the grain crisis of 1789.

The goal of the project was threefold: to teach students about events leading up to the Haitian Revolution, to facilitate their navigation of primary sources and support their finds with subsequent secondary sources, and to work at developing a written academic voice in both French and English. The students were tasked with reviewing the site, reading several translations and the accompanying introductions, and choosing an actor highlighted in either Issue 1.0 or 2.0 (at the time, 3.0 had not yet been published). Actors included, for example, administrators, various social structures, key cities, or any other aspects of life in colonial Saint-Domingue signaled in the documents. In the end, students wrote notes on six individuals, three geographic locations, four groups of people, and one organization. Dr.~Benharrech adopted the motto \quotation{Go beyond Wikipedia} and assigned students to write informational \quotation{background notes} of approximately two hundred words to be included on the official site.\footnote{For a librarian's perspective on {\em Wikipedia}, see Johnny Snyder, \quotation{Wikipedia: Librarians' Perspectives on Its Use as a Reference Source,} {\em Reference and User Services Quarterly} 53, no. 2 (2013): 155--63, doi:10.5860/rusq.53n2.155.} These notes, written in both French and English (self-translation by the students), describe the actors' roles within the greater historical context of the time period. Students worked in pairs and used sources contemporary to the eighteenth-century pamphlets and modern academic works that reflect on the period, rather than current encyclopedia-style sources easily accessible on the Internet. As a literature course, \quotation{Riots, Rebellions, and Revolutions} allowed students to read primary sources as \quotation{literature.} Though these sources were indeed treated as pieces of historical evidence, Dr.~Benharrech's required reading of the speeches, addresses, and letters found in {\em A Colony in Crisis}'s archives enabled students to confront questions regarding what constitutes \quotation{literature} and how interdisciplinary humanities scholarship can contribute to holistic pictures of history.

To introduce students to the site, the project team took advantage of physical proximity with an in-person visit to the class by Dize and Corlett-Rivera. Corlett-Rivera introduced students to UMD's resources and later guided students through the website technology, while Dize, still a graduate student at UMD, visited the class to give students a tutorial on how to approach primary sources from both historical and literary perspectives. Because these two authors were local, they were able to provide in-person services that will rarely be replicated at such a personal level. While the physical intervention in the classroom was a \quotation{typical \quote{one-shot,}} we would have preferred a more \quotation{embedded} approach, integrating librarians into the university classroom on multiple levels.\footnote{Frances Devlin and E. Bruce Hayes, \quotation{A Faculty/Librarian Collaboration to Restructure a Graduate Research Methods Class for French Literature Students,} {\em French Review} 89, no. 2 (2015): 146--63. The \quotation{one-shot} instruction session has been explored extensively in library and information science literature; see the following recent publications and their bibliographies for further insight: Heidi E Buchanan and Beth A McDonough, {\em The One-Shot Library Instruction Survival Guide} (Chicago: ALA Editions, 2014); Sarah Cisse, {\em The Fortuitous Teacher: A Guide to Successful One-Shot Library Instruction} (Cambridge, MA: Chandos, 2016); and Jill Markgraf et al., {\em Maximizing the One-Shot: Connecting Library Instruction with the Curriculum} (Lanham, MD: Rowman and Littlefield, 2015).} When librarians are \quotation{embedded} in the classroom, they become an obvious advantageous resource for student scholarship. Not only do librarians possess infinite tools for research, including knowledge of where to look for scholarly sources as well as access to the holdings of the university library and a whole host of affiliated libraries, they serve as intermediaries between the researcher and the archive.

Libraries hold treasures, yet few students know how to access them. {\em A Colony in Crisis} exists as both a repository of historical objects and as a vehicle for interpretative engagement in the undergraduate classroom. The librarian and the scholars could have become embedded in the physical and virtual classroom to both lead and assist with learning, which likely would have reduced some of the challenges we faced when completed assignments were received. Though the site includes the documents in both French and English, the specific scope of the project demands guidance for students with limited contact with primary sources; the historical objects are constrained by their limited number and status as part of a temporarily static repository. While our introductions to the pamphlets do render them accessible to a first-time reader, it is important pedagogically to guide students through the \quotation{first encounter} with the repository in order to give them the necessary tools to continue their journey into the archives throughout their careers.\footnote{To this end, we call for educators to walk their students through the site as an introduction to historical document reading as well as for content purposes regarding the grain crisis in Saint-Domingue.} These tools range from library literacy (how to navigate library catalogues and search scholarly databases), how to approach original documents materialistically, and where to turn for secondary sources that shed light on the primary documents' implications.

The experiment yielded productive, yet varied, results for the site, which required much more collaboration from the site authors than originally anticipated. Though the copy for the background notes was finalized on the students' end, we immediately realized that in order to upload it to the site we would have to significantly rework the text. Since we received the work at the end of the semester, after grades had been submitted, the students were no longer involved in the project. This meant that the reworking task fell to the project team alone. The primary problem we encountered was a tension between the undergraduate students' work and existing historical scholarship, namely, scholarship that centers on the delicate narrative historians have weaved since the Haitian Revolution that seeks to undermine colonial discourse from the perspective of the colonized and deemphasize dominant narratives that ignored colonized voices. Students worked with an imbalance of primary and secondary sources, most of which were dated and some of which were misleading. The students' work was linguistically accurate but erroneous in its conclusions. For example, in writing about Les Cayes, the student author cites historian M. L. E. Moreau de Saint-Méry's work {\em Description topographique, physique, civile, politique et historique de la partie française de l'isle Saint-Domingue}, volume 2, published in 1797.\footnote{\quotation{Les Cayes,} Historical Background Notes, {\em A Colony in Crisis}, 5 May 2016, \useURL[url43][http://bit.ly/2kSc1Hc][][bit.ly/2kSc1Hc]\from[url43].} While the student integrates Moreau's eighteenth-century description through paraphrasing and translation, no additional sources are cited to cross-reference the description. Not all students, however, fell into this historical trap. Some cited exclusively modern, credible scholarship yet failed to include any eighteenth-century sources; for example, the student authors of the note \quotation{The Slaves of Saint-Domingue} cite modern scholars David Geggus, Arlette Gautiers, and Franklin Midy (articles from 1990, 1989, and 2006, respectively).\footnote{\quotation{The Slaves of Saint-Domingue,} Historical Background Notes, {\em A Colony in Crisis}, 5 May 2016, \useURL[url44][http://bit.ly/2kHEPRQ][][bit.ly/2kHEPRQ]\from[url44].}

While not every background note lends itself to cross-references with both modern and period sources, it was only after the students submitted their work that we realized the inconsistencies within the ensemble. Though we had instructed students to use both primary and secondary sources, we did not provide specific guidelines that would create unified results, nor were we fully embedded in the class to be able to answer questions and address issues as they arose. The note \quotation{The French Colonists,} which comments on the ratio of slaves to colonists and their interactions, offers a thorough example of multi-generational scholarship. The student begins the note citing John D. Garrigus's 2006 book {\em Before Haiti: Race and Citizenship in French Saint-Domingue} and subsequently cites Daniel Livesay's 2015 article \quotation{Emerging from the Shadows: New Developments in the History of Interracial Sex and Intermarriage in Colonial North America and the Caribbean.} Next, to incorporate contemporary eighteenth-century sources, the student author refers to {\em Les Affiches américaines} from 1884, which advertises slaves alongside other goods for sale in Saint-Domingue.\footnote{\quotation{The French Colonists,} Historical Background Notes, {\em A Colony in Crisis}, 5 May 2016, \useURL[url45][http://bit.ly/2tyhTZC][][bit.ly/2tyhTZC]\from[url45].} This blend of scholarship produces a comprehensive background note that not only enriches understanding from points of view contemporary to {\em A Colony in Crisis}'s pamphlets but also reflects critically on history in an attempt to offer a global perspective to site visitors. The digital platform permits links to archived documents, such as {\em Les affiches américaines}, which establishes {\em A Colony in Crisis} as a curated, digital historical portal, where site visitors are only a click away from reaching beyond our work and into a variety of digital archives.

Primary problems with the background notes centered around the students' general lack of knowledge of historical discourse. While editing, the team encountered multiple entries that simply restated colonial ideologies, citing outdated sources instead of critically looking into the intersectional nature of oppression and contributing to contemporary conversations regarding Saint-Domingue.\footnote{For more on intersectionality, see Kimberle Crenshaw, \quotation{Mapping the Margins: Intersectionality, Identity Politics, and Violence against Women of Color,} {\em Stanford Law Review} 43, no. 6 (1991): 1241--99.} This misunderstanding is best represented, for example, when one student wrote that the colony's free people of color were the result of \quotation{relations between white landowners and their slaves,}\footnote{\quotation{Free People of Color,} Historical Background Notes, {\em A Colony in Crisis}, 5 May 2016, \useURL[url46][http://bit.ly/2l3ikKi][][bit.ly/2l3ikKi]\from[url46].} instead of directly calling the interactions \quotation{sexual coercion and rape.} Yet we cannot fault the inexperienced student for reproducing this narrative. How are undergraduates to know, without proper training, which sources are reliable? Looking back, we could have instructed students on themes such as race relations, gender studies, and identity politics in order to better situate them in their reading {\em before} sending them into library catalogue to retrieve quotable material. As Kimberlé Crenshaw writes, \quotation{In the context of violence against women, {[}the{]} elision of difference in identity politics is problematic, fundamentally because the violence that many women experience is often shaped by other dimensions of their identities, such as race and class.}\footnote{Crenshaw, \quotation{Mapping the Margins,} 1242.} Using the word {\em relations} in the background note is a result of ignorance of the colonial structure, one which hierarchized people based on gender, freedom status, and skin tone.\footnote{For more insight into the social dynamics of race and class in Saint-Domingue and Haiti, see Marlene L. Daut, {\em Tropics of Haiti: Race and the Literary History of the Haitian Revolution in the Atlantic World, 1789--1865}, 1st ed. (Liverpool: Liverpool University Press, 2015); and John D. Garrigus, {\em Before Haiti: Race and Citizenship in French Saint-Domingue} (Basingstoke: Palgrave Macmillan, 2006).} Future classroom interventions must pick up where these students left off, which will require the instructor to intentionally scaffold the unit to include conversations of race, gender, and economic disparity from the historical standpoint and through a twenty-first-century-critical lens.

As authors of the site, we are left to question how much responsibility lies with us, as creators, in managing how others read and interpret the documents. We have created an open-access source that has the potential to be used in a multiplicity of ways we cannot possibly survey. This first case study ended with Broughton and Dize spending the summer of 2015 reworking the students' contributions and taking note of how to minimize future complications. Because we agreed to collaborate on the assignment and committed to including the notes on our site, it was important to us that the information be as accurate and perceptive as possible.\footnote{See related articles on historicity, such as Patrick Hutton, \quotation{Recent Scholarship on Memory and History,} {\em History Teacher} 33, no. 4 (2000): 533--48.} To this end, we chose not to publish some of the notes. While the notes we excluded could have served as a teachable moment, the end of the semester subsequently ended the class's involvement in the project. Several of the students graduated and left campus, while others moved on to different classes and new research interests. This is, undoubtedly, one of the most crucial missed opportunities of our intervention.

After Broughton and Dize reworked and removed questionable scholarship, the team, including Corlett-Rivera and new site editor Brittany de Gail, went through each background note verifying sources, flagging notes too poorly researched to share, and copyediting the work. We additionally adjusted sentences and vocabulary to maintain consistency throughout the site; for example, when we intervened to change the aforementioned term \quotation{relations} to \quotation{sexual coercion and rape.}\footnote{Ibid.} Furthermore, de Gail spent far more time than anticipated correcting citations and formatting. More often than not, the footnotes referring to secondary sources contained incomplete or incorrect information. At worst, the citations were simply URL addresses to Google books; at best, the citations were complete but formatted incorrectly according to the style guide. Reworking these notes was a lengthy process of researching and verifying the publication details for each source, then subsequently rewriting the footnotes in the proper citation style. In order to keep the students' work at the forefront and as authentic as possible, the team attempted to keep content modifications to a minimum. Relinquishing control in this way proved difficult, however, and in order to create some distance between student work and {\em A Colony in Crisis}, we made the decision to flag each background note with the following annotation: \quotation{Note: This work is the result from a research assignment given to University of Maryland undergraduate and graduate students enrolled in Dr.~Sarah Benharrech's course \quote{Riots, Rebellions, and Revolutions,} taught in Fall 2015. Learn more,} with a link to the classroom collaboration description.

In the end, the background notes do greatly contribute to {\em A Colony in Crisis} as a whole, since they (now) document scholarly secondary sources and present source entries written in web-friendly prose accessible to a wide, interdisciplinary audience. It would have better served both our site and the students had we required each student to cite both eighteenth-century and modern sources, since this would have allowed students to rely on present-day research methods they are used to but also explore the archives. Future classroom applications will certainly use this first intervention as a starting point from which to evolve and improve. Thanks to the digital platform, the students created an interactive version of an encyclopedic research tool that accompanies the original documents and gives insight to the various actors surrounding the grain crisis of 1789. The flag that makes it immediately apparent that the background notes are a result of a collaboration with students and are not a product of the site authors themselves is not meant to silence the student authors. Digital humanities evokes collaborative methods, and credit is due to those who contribute. The collaboration with Dr.~Benharrech proves most valuable as a guide for where to steer future collaborations with undergraduate students. In hopes of inspiring consistency, the site team will develop rubrics and informational guidelines that will better guide students and instructors on their journey into the digital archives.

\subsection[reference={conclusion-future-directions-of-a-colony-in-crisis-in-the-classroom},
bookmark={Conclusion: Future Directions of *A Colony in Crisis* in the Classroom},
title={Conclusion: Future Directions of {\em A Colony in Crisis} in the Classroom}]

Further implementation of {\em A Colony in Crisis} in the undergraduate classroom will call on university educators to not only encourage students to work with primary sources but also conduct responsible research. The academy has repeated dominant discourses for too long. The future remains with our students, who will either unknowingly continue said discourse or will fight to correct stale and inaccurate sources. {\em A Colony in Crisis} serves in the classroom as a pedagogical tool beyond the content of its archives. In our work with Dr.~Benharrech's \quotation{Riots, Rebellions, and Revolution} French-literature course, we learned that students in all stages of their undergraduate careers struggle with basic principles of research and academic production. Teaching students to use the library and read primary sources reaches beyond what they learn from reading the documents themselves. Reading must be viewed holistically, as a means of extracting knowledge and discerning information about the time period from which it was produced. Our collaboration between graduate students and a librarian attempts to teach through interaction with the website, but it remains to be seen how pedagogues continue our work with students in their own classroom.

{\em A Colony in Crisis} is an open-access website that instructors are free to use. If, however, in the future we work on expanding the site through collaborations with students, we will take steps to create a more comprehensive feedback loop between the educator, students, and ourselves in order to envision as a group an assignment that benefits both the students and the site. As with any collaboration, communication is paramount from all sides. For students to effectively carry out the tasks they have been given, the creators must have a clear conceptualization of the intended outcome. Furthermore, we need to encourage students to take ownership of the documents. Captivating students' interest regarding a historic episode that took place more than two hundred years ago is no easy feat. Presenting these documents dynamically as samples of history will help students relate to the context and invest in its value. Additionally, in-class activities such as dramatic oral readings of the speeches in order to study differences between written and spoken register, along with task-based close readings of rhetoric, will anchor students in purposeful readings, which will prevent them from losing focus and becoming overwhelmed.

In Dr.~Benharrech's course, {\em A Colony in Crisis} was used both as content---delving into the rhetorical strategies of French planters in Saint-Domingue as they proliferated the guise of a grain famine in order to circumvent French trade laws like the {\em exclusive}---and as a digital project to which the students would eventually contribute as an assignment.\footnote{Eller, \quotation{A Review of {\em A Colony in Crisis}.}} Focusing on the theme of riots, rebellions, and revolution allowed Dr.~Benharrech to contrast particular moments in French history, such as the silk workers riots in Lyon in the early nineteenth century ({\em les canuts de Lyon}), and popular rebellions, like the 1775 Flour War under the French Old Regime, to revolts and revolutions in the Caribbean. In an eighteenth-century context, teaching the Flour War alongside the purported grain famine in Saint-Domingue in 1789 gave students the opportunity to engage with the theme of revolution through the lens of global francophone food cultures.\footnote{For more background into the Flour War that set a standard rate for flour upon its conclusion, see Cynthia Bouton, {\em The Flour War: Gender, Class, and Community in Late Ancien Régime French Society} (University Park: Penn State University Press, 2005); and Vladimir S. Ljublinskii, {\em La guerre des farines: Contribution à l'histoire de la lutte des classes en France, à la veille de la Révolution}, trans. Françoise Adiba and Jacques Radiguet (Grenoble: Presses Universitaires de Grenoble, 1979).} In France, grain has long been a staple crop and a major contributor to the French economy. Through {\em A Colony in Crisis}, scholars are able to explore how the cultural purchase of grain allowed planters to present a cogent argument that the colony of Saint-Domingue needed to trade with the United States and other Caribbean colonies to preserve French cultural ties and maintain the system of plantation slavery.\footnote{\quotation{Summary Given by M. Le Marquis de Gouy d'Arsy,} 9 September 1789, {\em A Colony in Crisis}, \useURL[url47][http://bit.ly/2lDP3Xo][][bit.ly/2lDP3Xo]\from[url47].} {\em A Colony in Crisis} allows viewers to follow links between documents that help illustrate the context and the stakes of the grain crisis of 1789.

Apart from the intersection of food culture and revolution, there are other ways of teaching {\em A Colony in Crisis} in the context of an eighteenth-century course. For instance, in Issue 3.0, introduced by Marlene L. Daut (University of Virginia, Charlottesville), numerous documents speak to the conditions of the {\em gens de couleur libre} in Saint-Domingue on the eve of and during the Haitian Revolution.\footnote{Daut, \quotation{Issue 3.0: Introduction.}} These documents highlight the shift of the {\em gens de couleur libre} from free, noncitizen members of Saint-Dominguan society to full-fledged citizens between 1789 and April 1793.\footnote{David Geggus, {\em The Haitian Revolution: A Documentary History}, 1st ed. (Indianapolis: Hackett, 2014); Dubois and Garrigus, {\em Slave Revolution in the Caribbean}.} A closer look at these documents, such as the \quotation{Speech given October 23, 1791, by the Mayor of Port-au-Prince,} reveals a nascent republican discourse in Saint-Domingue among the {\em gens de couleur libre}.\footnote{\quotation{Speech given October 23, 1791, by M. the Mayor of Port-Au-Prince, Following the Peace Treaty between the White Citizens and the Citizens of Color from the Western Province of the French Section of Saint-Domingue,} 23 October 1791, {\em A Colony in Crisis}, \useURL[url48][http://bit.ly/2m5xPTK][][bit.ly/2m5xPTK]\from[url48].} It was only after the {\em gens de couleur libre} received full citizenship rights that the French government, as well as the local government in Saint-Domingue, refer to this particular class of people as citizens. In light of this, {\em A Colony in Crisis} could help open the discussion of comparative republicanism in the global francophone context. By analyzing the deployment of republican discourse in France in contrast to its use in Saint-Domingue, students would be privy to a whole new range of questions relating to the intersection between class and race vis à vis republicanism. In pursuing such an endeavor, teachers of eighteenth-century French literature and specialists of the Enlightenment alike could entertain speculation about whether the French Revolution, or even the Haitian Revolution, fully achieved the ambitions of the Declaration of the Rights of Man. As John Garrigus notes, these types of discussions have significant import for undergraduate students in the United States---especially in an era in which freedom and citizenship are constantly called into question in relation to a person's country of origin, family history, or potential status as refugee.\footnote{Garrigus, \quotation{White Jacobins/Black Jacobins.}} Through comparative discussions of slavery, citizenship, and personhood in a globalized francophone frame, students will be exposed to the full purchase of the Enlightenment in the twenty-first century.

Since Issue 1.0 of {\em A Colony in Crisis} went live in 2014, we have added two more issues of translated pamphlets along with the background notes described in the case study here. The site has received over nineteen thousand page views by visitors from ninety different countries, and several digital humanities courses list {\em A Colony in Crisis} as required reading on their syllabi.\footnote{For example, University of Saint Andrews, \quotation{Politics, Culture and Society in the French Revolution, 1789--1815,} Resource List (2016--17), \useURL[url49][http://bit.ly/2kHEKNH][][bit.ly/2kHEKNH]\from[url49]; and New York University Abu Dhabi, \quotation{Introduction to Digital Humanities AHC--AD 139,} Fall 2016, \useURL[url50][http://bit.ly/2mcB1gn][][bit.ly/2mcB1gn]\from[url50].} Along with impressive online traffic reports, the site has received positive feedback from scholars within the fields of digital humanities and Caribbean studies, as evidenced by Anne Eller's 2016 \useURL[url51][http://web.archive.org/web/20170904034632/http://smallaxe.net/sxarchipelagos/issue01/review-eller-colony.html][][review]\from[url51] in {\em sx archipelagos}. Further developments are in the works, including a collaboration established in 2016 with students at Montclair State University, who are interacting with the site's primary sources to provide translations in Haitian Creole, aiming to further expand the project's reach.

Teaching language includes teaching culture, which by default requires a study of history. Historical documents serve to study rhetoric throughout the ages as well as provide students with primary source information in order to encourage studying history from within and outside the context. As {\em A Colony in Crisis} continues to grow, the site's editors look to the future for opportunities to collaborate with classrooms and \quotation{embed} the archives into student scholarship and student scholarship into the website of archives. We aim for this relationship to be symbiotic in nature as academic work on the colonial and postcolonial Caribbean is rewritten and reveals lost narratives. In order for the Caribbean to continue its push from the margins to the center, scholars will need materials that help their students participate in the process of decolonizing the history of the region---pedagogy and the digital humanities are proving, and will continue to prove, to be indispensable tools in this endeavor.

\thinrule

\page
\subsection{Abby Broughton}

Abby R. Broughton is a PhD student in the Department of French and Italian at Vanderbilt University, where she specializes in twentieth-century queer literature, body and identity politics, and the intersection of illustration and text. She is a co-author, a translator, and an editor of {\em A Colony in Crisis: The Saint-Domingue Grain Shortage of 1789.}

\subsection{Kelsey Corlett-Rivera}

Kelsey Corlett-Rivera is the head of the Research Commons and a librarian for the School of Languages, Literatures, and Cultures at the University of Maryland. She leverages emerging technologies to provide services for researchers on campus and is the site designer and an editor of {\em A Colony in Crisis: The Saint-Domingue Grain Shortage of 1789.}

\subsection{Nathan Dize}

Nathan H. Dize is a PhD student in the Department of French and Italian at Vanderbilt University, where he specializes in Haitian theater, poetry, and revolutionary poetics during the nineteenth and early twentieth centuries. Nathan is a content curator, a translator, and an editor of {\em A Colony in Crisis: The Saint-Domingue Grain Shortage of 1789.}

\subsection{Brittany M. de Gail}

Brittany de Gail is a site editor for {\em A Colony in Crisis} and an administrative assistant at the University of Maryland Libraries, where she also graduated with a B.A. in Chinese and Government and Politics.

\stopchapter
\stoptext